\documentclass[../../main]{subfiles}
\begin{document}

\paragraph{}
\begin{exercise}
	Following Grothendieck, define a \textbf{fiber space} \(\func{p}{E}{B}\) to
	be a morphism in \(\Top\). A map of fiber spaces is a commutative square.
	Thus the category of fiver spaces is isomorphic to the diagram category
	\(\Top^\two\). We are also interested in the non-full subcategory
	\(\Top/B\subset\Top^\two\) of fiber spaces over \(B\) and maps whose
	codomain component is the identity. Prove the following:
	\begin{enumerate}[(i)]
		\item A map
			\[\begin{tikzcd}
					E' \ar[r, "g"] \ar[d, "p'"] & E \ar[d, "p"] \\
					B' \ar[r, "f"] & B
			\end{tikzcd}\]
			of fiber spaces induces a canonical map from the fiber of a point
			\(b\in B'\) to the fiber over its image \(f(b)\in B\).
			\begin{proof}
				The fiber of \(b\) is a subset of \(E'\), so we may consider its
				image under \(g\). Given a point \(e\) in this set we have that
				\(p'(e)=b\) and \(pg(e)=fp'(e)\) by assumption, and thus
				\(pg(e)=f(b)\). This means that \(g(e)\) is in the fiber of
				\(f(b)\) and thus the restriction of \(g\) to the fiber of \(b\)
				lands in the fiber of \(f(b)\).
			\end{proof}

		\item The fiber of a product of fiber spaces is the product of the
			fibers.
			\begin{proof}
				Let \(\cI\) be some set and \(\func{p_i}{E_i}{B_i}\) be an
				\(\cI\) indexed set of maps over corresponding \(\cI\) indexed
				spaces, and \(\func{p}{E}{B}\) be the corresponding products.
				We claim that:
				\[\inv{\qty{\prod_{i\in\cI}p_i}}(b)=
					\prod_{i\in\cI}\qty{\inv{p_i}(b_i)}\qtext{for all
				\(b=(b_i)_{i\in\cI}\in B\)}.\]

				We then have a collection of diagrams for each \(i\in\cI\).
				\begin{equation*}\begin{tikzcd}
						E_b\subset E \ar[r, "\pi_i"] \ar[d, "p", shift left=4] &
						E_i\supset E_{b_i} \ar[d, "p_i", shift right=4] \\
						b\in B \ar[r, "\pi_i"] & B_i\ni b_i
				\end{tikzcd}\end{equation*}
				If we trace the path of \(b\) through this diagram it is clear
				that the bottom commutes by the definition of \(b\). Chasing
				\(b\) up the left side we have the fiber \(\inv{p}(b)\), and on
				the right we have the fiber \(\inv{p_i}(b_i)\). The previous
				part says that the restriction of \(\pi_i\) defines a map from
				the former fiber to the latter. However, the projections map
				\(\prod_{i\in\cI}\qty{\inv{p_i}(b_i)}\) onto each
				\(\inv{p_i}(b_i)\).
			\end{proof}
	\end{enumerate}
	A projection \(B\by F\to B\) defines a \textbf{trivial fiber space} over
	\(B\), a definition that makes sense for any space \(F\).
	\begin{enumerate}[(i),resume]
		\item Show that the fiber of a trivial fiber space \(B\by F\to B\) is
			isomorphic to \(F\).
			\begin{proof}
				Given an element \(b\in B\), its fiber is
				\(\inv{\pi_1}(b)=\set{(b,f)|f\in F}\). Note that if \(f=f'\) in
				\(F\) then \((b,f)=(b,f')\) in \(B\by F\). Thus
				\(\pi_2|_{\inv{\pi_1}(b)}\) is injective in addition to being
				surjective, and thus has an inverse map. Further, because the
				projection is an open map this inverse is also continuous. Thus
				\(\inv{\pi_1}(b)\) is isomorphic to \(B\).
			\end{proof}

			\pagebreak
		\item Characterize the isomorphisms in \(\Top/B\) between the two
			trivial fiber spaces (with a priori distinct fibers) over \(B\).

			\subparagraph{}
			Note that \(\Top/B\) is a slice category, so an isomorphism in this
			category is an isomorphism in \(\Top\) with the additional condition
			that it commutes with the projections onto \(B\).

			Explicitly, we require the following diagram to commute.
			\[\begin{tikzcd}
					B\by F \ar[rr, "\gm", shift left] \ar[dr, "\pi_1"] &&
					B\by F' \ar[ll, "\de", shift left] \ar[dl, "\pi_1"] \\ &B
			\end{tikzcd}\]
			Note first that the map \(\gm\) may be decomposed into
			\(\func{\pi_1\gm}{B\by F}{B}\) and \(\func{\pi_2\gm}{B\by F}{F'}\)
			and the universal property of products guarantees that this
			decomposition preserves all of the information of \(\gm\). Now, one
			of the identities we can pull from the diagram is that
			\(\pi_1\gm=\pi_1\). This means we can restrict our consideration to
			the other projection \(\pi_2\gm\).

			Given a homeomorphism \(\func{\tilde{\gm}}{F}{F'}\) we may define
			\(\func{\gm}{B\by F}{B\by F'}\) by
			\((b,f)\mapsto(b,\tilde{\gm}(f))\). If \(\tilde{\gm}\) is
			invertible, then it is clear that \(\gm\) satisfies all of the
			conditions imposed by the diagram. However, this is too restrictive.
			What if instead we had a family of maps
			\(\set{\func{\gm_b}{F}{F'}}_{b\in B}\) and defined \(\gm\) by
			\((b,f)\mapsto(b,\gm_b(f))\). In this case \(\gm\) will satisfy the
			condition that \(\pi_1\gm=\pi_1\) and if each \(\gm_b\) is
			invertible, then \(\gm\) will be as well, but it is not necessarily
			the case that \(\gm\) is continuous, i.e. it might be that \(\gm\)
			is not an arrow in our category.

			To see how we may impose continuity on \(\gm\), recall that a family
			is in fact a function so we have \(\func{\Gm}{B}{\Top(F,F')}\) where
			\(\Gm(b)=\gm_b\). Further, we may uncurry this to
			\(\func{\Gm}{B\by F}{F'}\) which is a function for which we have a
			well defined notion of continuity. Supposing then \(\Gm\) is
			continuous in this sense. The universal property of \(B\by F'\) then
			gives us that \(\pi_1\by\Gm=\gm\) is continuous and thus a
			legitimate map for our diagram.

			Piecing all of this together, let \(\func{\Gm}{B\by F}{F'}\) and
			\(\func{\De}{B\by F'}{F}\) be continuous maps such that the
			restrictions are inverses:
			\(\Gm(b,-)=\inv{\De(b,-)}\) for any \(b\in B\). Next define
			\(\gm=\pi_1\by\Gm\) so that \(\gm(b,f)=(b,\Gm(b,f))\) and like wise
			for \(\de\). Then \(\gm\) and \(\de\) are inverses and satisfy all
			the conditions imposed by the diagram.

		\item Prove that the assignment of the set of continuous sections of a
			fiber space over \(B\) defines a functor
			\(\func{\Sect}{\Top/B}{\Set}\).
			\begin{proof}
				We can define $\Sect$ by mapping each fiber space $p$ over $B$
				to its set of continuous sections, and by mapping each morphism
				$\func{f}{p}{q}$ to a function $\func{f \cdot -}{\Sect p}{\Sect
				q}$ defined by left composition of $f$. To verify that our
				proposed function is a map between set of sections let $s$ be a
				section of $p$, then $1_{B} = p \cdot s$. Since $p = q \cdot f$,
				then $1_{B} = q \cdot f \cdot s$, showing that $f \cdot s$ is a
				section of $q$. To verify that this is indeed a functor, note
				that the identity morphism on $\func{p}{E}{B}$ defined by
				$\func{1_{E}}{p}{p}$ gets mapped to left composition by $1_{E}$
				which induces the identity on $\Sect p$. Take morphism $g \cdot
				f$ .Then $\Sect \qty{g \cdot f} = g \cdot f \cdot - = g \cdot
				\qty{f \cdot -} = \Sect g \cdot \Sect f$. Thus $\Sect$ is a
				functor
			\end{proof}

		\item Consider the non-full subcategory \(\Top^\two_\pb\) of fiber
			spaces in which the morphisms are the pullback squares. Prove that
			the assignment of the set of continuous sections to a fiber space
			defines a functor \(\func{\Sect}{\qty{\Top^\two_\pb}^\op}{\Set}\).
			\begin{proof}
				We can define $\Sect $by mapping each fiber space $p$ over $B$
				to its set of continuous sections and each morphisms,
				$\func{\qty{f,g}}{p'}{p}$ is a pullback, then $\qty{f,g}$ get
				mapped to a function $\func{\sm_{\qty{f,g}}}{\Sect p}{\Sect p'}$
				defined as follows: each section $s$ of $p$ gets taken to a
				section $s'$ using the universal property of the pullback
				$\qty{f,g}$ shown in following diagram:
				\[ \begin{tikzcd}
						B' \ar[drr, bend left, "s \cdot g"]
						\ar[ddr, bend right, "1_{B'}"]
						\ar[dr, dotted, "s'" description] & & \\ &
						E' \ar[r, "f"] \ar[d, "p'"]
						\ar[dr, phantom, "\lrcorner", very near start] &
						E \ar[d, "p"] \\ &
						B' \ar[r,"g"] &
						B
				\end{tikzcd}\]
				Thus $s'$ is unique for each $s$ confirming that
				$\sm_{\qty{f,g}}$ is a function. To verify that $\Sect$ is a
				functor note that the identity morphism on $p$ induces the
				following commutative diagram:
				\[\begin{tikzcd}
					B \ar[drr, bend left, "s"]
					\ar[ddr, bend right, "1_{B}"]
					\ar[dr, dotted, "s" description] & & \\ &
					E \ar[r, "1_{E}"] \ar[d, "p"]
					\ar[dr, phantom, "\lrcorner", very near start] &
					E \ar[d, "p"] \\ & B \ar[r, "1_{B}"] & B
				\end{tikzcd}\]
				where $s$ is a section.

				For morphisms $\qty{f',g'}$  and $\qty{f,g}$ such that
				$\qty{f,g} \cdot \qty{f',g'}$ makes sense, if $s \in \Sect p$
				with $ \sm_{\qty{f,g}}\qty{s} = s'$ and $
				\sm_{\qty{f',g'}}\qty{s'} = s''$, then $s \cdot g = f \cdot
				s'$ and $s' \cdot g' = f' \cdot s''$ by the universal property
				of pullbacks $\qty{f',g'}$  and $\qty{f,g}$. We get that $s
				\cdot g \cdot g' = f \cdot s' \cdot g' = f \cdot f' \cdot s'' $.
				Since $s''$ is a section of $p''$ this allows the following
				diagram to commute:
				\[\begin{tikzcd}
					B \ar[drr, bend left, "s \cdot g \cdot g' "]
					\ar[ddr, bend right, "1_{B''}"]
					\ar[dr, dotted, "s''" description] & & \\ &
					E'' \ar[r, "f \cdot f'"] \ar[d, "p''"] \ar[dr, phantom,
					"\lrcorner", very near start] &
					E \ar[d, "p"] \\ & B'' \ar[r,"g \cdot g'"] & B
				\end{tikzcd}\]
				showing that $ s'' =
				\sm_{\qty{f',g'}}\qty{\sm_{\qty{f,g}}\qty{s}} =
				\sm_{\qty{f,g}\cdot \qty{f',g'}}\qty{s} $ for all $s \in
				\Sect p$. Thus, $\sm_{\qty{f,g}\cdot \qty{f',g'}} =
				\sm_{\qty{f',g'}} \cdot \sm_{\qty{f,g}}$. This shows that
				$\Sect$ preserves composition. Therefore, $\Sect$ is a functor.
			\end{proof}

		\item Describe the compatibility between the actions of the ``sections''
			functors just introduced with respect to the map \(g\) of fiber
			spaces \(p\) and \(q\) over \(B\) and their restrictions along
			\(\func{f}{B'}{B}\).
			\[\begin{tikzcd}
					E' \ar[rr] \ar[dd] && E \ar[dd] \ar[dr] \\ &
					F' \ar[rr] \ar[dl] && F \ar[dl] \\
					B' \ar[rr] && B
			\end{tikzcd}\]
			\begin{proof}
				Consider the pullbacks

				$$ \begin{tikzcd}
					F' \ar[r, "\pi_{1}"] \ar[d, "q'" {left}] \arrow[dr, phantom, "\lrcorner", very near start] & F \ar[d, "q"] \\
					B' \ar[r, "f"] & B
				\end{tikzcd} $$ and

				$$ \begin{tikzcd}
					E' \ar[r, "\pi_{2}"] \ar[d, "p'" {left}] \arrow[dr, phantom, "\lrcorner", very near start] & E \ar[d, "p"] \\
					B' \ar[r, "f"] & B
				\end{tikzcd} $$ and a morphism $\func{g}{E}{F}$ such that $q \cdot g = p$. Using the universal property of the first pullback, we can derive a new morphism $\func{g'}{E'}{F'}$ such that

				$$ \begin{tikzcd}
					E'
					\arrow[drr, bend left, "g \cdot \pi_{2} "]
					\arrow[ddr, bend right, "p'" {left}]
					\arrow[dr, dotted, "g'" description] & & \\
														 & F' \arrow[r, "\pi_{1}"] \arrow[d, "q'"{left}] \arrow[dr, phantom, "\lrcorner", very near start]
														 & F \arrow[d, "q"] \\
														 & B' \arrow[r,"f"]
														 & B
				\end{tikzcd}$$

				We can treat the triples $\qty{g,p,q}$ and $\qty{g',p',q'}$ as morphisms $\func{g}{p}{q}$ and $\func{g'}{p'}{q'}$ in $\Top/B$. We can also treat the two pullback diagrams as morphisms $\func{\qty{\pi_{1},f}}{q'}{q}$ and $\func{\qty{\pi_{2},f}}{p'}{p}$ in \(\Top^\two_\pb\). Applying the functor $\Sect$ defined in part 5 to $\func{g}{p}{q}$ and $\func{g'}{p'}{q'}$ we get the functions $\func{g \cdot -}{\Sect p}{\Sect q}$ and $\func{g' \cdot -}{\Sect p'}{\Sect q'}$. Applying $\Sect$ defined in part 6 to the pullback squares we get the functions
				$\func{\sigma_{\qty{\pi_{1},f}} }{\Sect q}{\Sect q'}$ and $\func{\sigma_{\qty{\pi_{2},f}} }{\Sect p}{\Sect p'}$. To show that two definitions of $\Sect$ are compatible, we will show that the following diagram commutes:
				$$ \begin{tikzcd}
					\Sect p \ar[r, "g \cdot -"] \ar[d, "\sigma_{\qty{\pi_{2},f}}" {left}]
					& \Sect q \ar[d, "\sigma_{\qty{\pi_{1},f}}"] \\
					\Sect p' \ar[r, "g' \cdot -"{below}] & \Sect q'
				\end{tikzcd} $$ which should be the result of applying a functor $\Sect$ overloaded with the definitions from part 5 and 6 to the following diagram:
				$$ \begin{tikzcd}
					E'\arrow[dr, phantom, "\lrcorner", very near start] \ar[dd, bend right , "p'"{left}] \ar[r, "\pi_{2}" ] \ar[d, "g' "] & E \ar[d,"g"{left} ] \ar[dd, bend left , "p"{right}]\\
					F' \arrow[dr, phantom, "\lrcorner", very near start] \ar[r , "\pi_{1}" ] \ar[d, "q'" ] & F \ar[d, "q"{left}] \\
					B' \ar[r, "f"{below} ] & B
				\end{tikzcd} $$ which is  the combination of the two pullbacks with the morphisms $g$ and $g'$ included. Let $s \in \Sect p$, then $g \cdot s \in \Sect q$ and $\sigma_{\qty{\pi_{2},f}}\qty{s} = s' \in \Sect p'$. To show that $g' \cdot s' = \sigma_{\qty{\pi_{1},f}}\qty{g \cdot s}$ we will use the universal property of the second pullback defined at the beginning of the proof. This gives us $\pi_{2} \cdot s' = s \cdot f$. By extension, $g \cdot \pi_{2} \cdot s' = g \cdot s \cdot f  $. By the diagram above, this gives us that $ \pi_{1} \cdot g'  \cdot s' = g \cdot s \cdot f  $. Furthermore $q' \cdot g' \cdot s' = p'  \cdot s' = 1_{B'}$. This allows the following diagram to commute:
				$$ \begin{tikzcd}
					B'
					\arrow[drr, bend left, "g \cdot s \cdot f"]
					\arrow[ddr, bend right, "1_{B'}" {left}]
					\arrow[dr, dotted, "g' \cdot s'" description] & & \\
																  & F' \arrow[r, "\pi_{1}"] \arrow[d, "q'"{left}] \arrow[dr, phantom, "\lrcorner", very near start]
																  & F\arrow[d, "q"] \\
																  & B' \arrow[r,"f"]
																  & B
				\end{tikzcd} .$$

				This confirms that $g' \cdot s' = \sigma_{\qty{\pi_{1},f}}\qty{g
				\cdot s}$ and that the diagram formed by the $\Sect$ functor
				commutes. This shows that the definitions of $\Sect$ in 5 and 6
				are compatible in the sense that $\Sect$ overloaded with both
				defintions, preserves commutativity of diagrams formed by
				morphisms of $\Top /B$  and \(\Top^\two_\pb\)
			\end{proof}
	\end{enumerate}
\end{exercise}

\end{document}
