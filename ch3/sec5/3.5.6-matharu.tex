\documentclass[main.tex]{subfiles}
\begin{document}

	\paragraph{}
	\begin{exercise}
		Define a contravariant functor $ \cat{Fin}_{\textrm{mono}}^\op \to
		\Top$ from
		the	category of finite sets and injections to the category of
		topological
		spaces that sends a set with $ n $ elements to the space $
		\cat{PConf}_n(X) $
		constructed in Example 3.5.4. Explain why the functor does not induce a
		similar functor sending an $ n $-element set to the space $ \cat{Conf}_n(X). $
	\end{exercise}

	\begin{proof}
		Recall that
		$$
		\cat{PConf}_n(X)=\{(x_1,\dots,x_n)\mid x_i\neq x_j \forall i\neq j
		\}.
		$$
		For simplicity we will define $ F\colon\cat{Fin}_{\textrm{mono}}^\op
		\to \Top $  on finite sets of the form $ \{0,\dots,n-1\} $ and then
		extend $ F $ to arbitrary finite sets. Define $ F $ on the objects by
		the map $\{0,\dots,n-1\}\mapsto \cat{PConf}_n(X)$ and on morphisms as
		$ (x_0,\dots,x_{n-1})\mapsto (x_{f(0)},\dots, x_{f(m-1)}). $
		Note that this functor is contravariant as we have $f$ in the
		subscripts not $f\inv.$ We also needed that $ f $ was injective else we
		could land on the ``fat diagonal.'' For example if $ f $ has sent every
		element to $ 0 $ we would be on the actual diagonal, not even the fat
		one.

		To extend this map to any finite set we first note that that every
		finite set $ A $ with $ n $ elements
		is pretty much the same as the set $ \{0,\dots,n-1\}, $
		since we can always find a bijection between them, so the
		extension should not be too bad. For any finite set $ A $ with $ n $
		elements we can fix\footnote{Notice to do we do need to invoke the
		Axiom of Choice in a larger universe as the collection of singletons is
		a proper class, so we are choosing a proper class worth of
		bijections.} a bijection $ \func{\phi_A}{A}{\{0,\dots,n-1\}}.$
		This gives a commutative diagram for finite sets $ A\AND B $ of
		sizes $ m\AND n $ respectively and a injective
		map $ f $ between them.
		$$
		\xymatrix{
			A\ar[r]^{f}\ar[d]_{\phi_A}&B\ar[d]^{\phi_B}\\
			\{0,\dots,m-1 \}\ar[r]_{\phi_B f\inv{\phi_A}}&\{0,\dots,n-1\}
		}
		$$

		Now we can extend
		$F\colon\cat{Fin}_{\textrm{mono}}^\op
		\to \Top $
		on morphisms as $ Ff\coloneq F(\phi_B f\phi_A\inv),$
		where $ F $ sends both $\phi_A\inv\AND\phi_B $ to the identity map
		in $ \cat{Conf}_n(X). $
		This is equivalent to $ (x_1,\dots,x_n)\mapsto (x_{f(1)},\dots,
		x_{f(m)}),$ while leaving the action on objects the same.

		If we try and induce a similar functor, but allowed for
		permutation we could have for a fixed	injective map $f$ that takes
		order tuples in the same equivalence class to tuples in different
		equivalence classes. Consider the ordered triples	$ (x_0,x_1,x_2)
		\AND (x_1,x_2,x_0)$ and the map $ \func{g}{\{0,1\}}{\{1,2,3\}} $ with $
		g(0)=1\AND g(1)=2. $ If we apply the functor we defined earlier we have
		$ (x_0,x_1,x_2)\mapsto (x_0,x_1) \AND (x_1,x_2,x_0)\mapsto (x_1,x_2).$
		Even though $ (x_0,x_1,x_2) \AND (x_1,x_2,x_0)$ were in the same
		equivalence class $ (x_0,x_1)\AND (x_1,x_2) $ are not.
	\end{proof}
\end{document}
