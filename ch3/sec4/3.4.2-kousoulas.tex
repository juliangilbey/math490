\documentclass[main.tex]{subfiles}

\begin{document}
\paragraph{}
\begin{exercise}
	Explain in your own words why the Yoneda embedding
	\({\sC}\hookrightarrow{\Set^{\sC^\op}}\) preserves and reflects but does not
	create limits.
\end{exercise}

Recall that the Yoneda embedding of a category \(\sC\) takes each object to the
presheaf which it represents and maps to natural transformations between these
presheaves. Explicitly, an object \(c\) becomes the hom-functor \(\sC(-,c)\),
and a map \(f\) becomes the natural transformation of post-composition with
\(f\). The Yoneda lemma states that this functor is fully faithful, and thus
\(\sC\) is realised as a full subcategory of \(\Set^{\sC^\op}\). Recall that
this means \(\Set^{\sC^\op}\) may contain objects not in \(\sC\), but if the
domain and codomain of a morphism are present in both categories then the
morphism must be as well. For the sake of brevity we will treat \(\sC\) as being
part of \(\Set^{\sC^\op}\) and ignore the renaming implicit in the isomorphism.
Correspondingly, a functor \(\func{F}{\sJ}{\sC}\) is also a functor
\(\func{F}{\sJ}{\Set^{\sC^\op}}\).

This makes it immediately clear why the Yoneda embedding reflects limits. Given
a cone over the diagram \(\func{F}{\sJ}{\Set^{\sC^\op}}\) with an apex
\(c\), first, this cone will exist in \(\sC\) if and only if \(c\) and \(Fj\)
where \(j\in\ob J\) are objects in \(\sC\). Second, if this is a limit cone in
\(\Set^{\sC^\op}\), then there is a unique map from any other cone over \(F\)
subject to naturality conditions. If the apex of this cone is in \(\sC\), then
so must be the unique map by the condition that the \(\sC\) is a full
subcategory. Thus \(K\) will still be a limit cone considering only the objects
and morphisms in \(\sC\).

Now, suppose that \(c\) is the limit of a functor \(\func{F}{\sJ}{\sC}\). \dots

However, given an arbitrary object \(c\) in \(\Set^{\sC^\op}\) there is no
guarantee that there is an object in \(\sC\) isomorphic to \(c\). Thus if \(c\)
happens to be the apex of a limit cone, it is impossible for that cone to have a
limit in \(\sC\) even if the base of that cone is entirely in \(\sC\).

\end{document}
