\documentclass[main.tex]{subfiles}
\begin{document}

\paragraph{}
\begin{exercise}
For a fixed diagram $F \in \sC^{\sJ}$, show that the cone functor $\Cone(-,F)$
is naturally isomorphic to $\Hom(\Delta(-), F)$, the restriction of the
hom-functor for the category $\sC^{\sJ}$ along the constant functor embedding
defined in 3.1.1.
\end{exercise}

\begin{proof}
First, we note what we must show that we can define for each $c \in \sC$, a
function $\eta_c: \Cone(c,F) \rightarrow \Hom(\Delta(c), F)$ that causes the
following diagram to commute for $f\colon c \rightarrow c'$.
\begin{align}
   \xymatrix{
\msf{Cone}(c', F)
\ar[r]^{\Cone(-,F)(f)}
\ar[d]_{\eta_c'}
& \msf{Cone}(c,F)
\ar[d]^{\eta_c}
\\
\Hom(\Delta(c'),F)
\ar[r]_{\Hom(\Delta(-), F)(f)}
& \Hom(\Delta(c),F)
\\
}
\end{align}

Now, we note definition 3.1.2, which defines a cone over $F$ with apex $c$ as a
natural transformation $\lambda\colon \Delta(c) \rightarrow F$. We also see
that $\epsilon \in \Hom(\Delta(c),F)$ is defined by the following commutative
diagram for a morphism $f\colon d \rightarrow e$.
\begin{align}
    \xymatrix{
    c
    \ar[r]^{1_c}
    \ar[d]_{\epsilon_d}
    & c \ar[d]^{\epsilon_e}
    \\
    Fd \ar[r]_{Ff}
    & Fe
    }
\end{align}
We can condense this diagram into the triangular diagram
\begin{align}
   \xymatrix
   { & c \ar[dl]_{\epsilon_d} \ar[dr]^{\epsilon_e}  & \\
     Fd \ar[rr]_{Ff} & & Fe
   }
\end{align}
Since we have a diagram like this for every $f \in \mor\sC$, we see that
$\epsilon$ defines a cone in $\Cone(c,F)$. So we see that elements $\Cone(c,F)$
and $\Hom( \Delta(c), F)$ describe the same class of natural transformations. So
the image of $c$ under the functors $\Cone(-,F)$ and $\Hom(\Delta(-), F)$ is
identical. So these functors behave identically on objects.

We know by exercise 3.1.i that $\Cone(-,F)(f)\colon \Cone(c',F)
\rightarrow \sC(c,F)$ takes a $\lambda_i\colon c' \rightarrow Fi$ to $\lambda_i
f\colon c \rightarrow F$. We also see that by definition 3.1.1, that  for an
$\epsilon \in \Hom(\Delta(c'),F)$ and $f\colon c \rightarrow c'$, that
$\Hom(\Delta(-), F)(f) = \epsilon * \Delta(f)$, where $\Delta(f)$ is the
constant natural transformation defined by $f$ between the functors
$\Delta(c')$ and $\Delta(c)$. Since we are vertically composing these natural
transformations, we see that $(\epsilon * f)_i = \epsilon_i f_i = \epsilon_i
f$. So these functors also behave identically on morphisms and so they are
identical and have a trivial natural isomorphisms between them.
\end{proof}

\end{document}
