\documentclass[../../main]{subfiles}
\begin{document}

\paragraph{}
\begin{exercise}
	Give a second proof of Proposition 3.1.7 by using the universal properties
	of each of a pair of limit cones $ \lm \colon l \Rightarrow F $ and $
	\lm' \colon l' \Rightarrow F $ to directly construct the unique
	isomorphism $ l \cong l' $ between their apexes.
\end{exercise}

\begin{proof}
Since $ \lm $ and $ \lm' $ are limit cones, for any cones $
\Lm\colon c\Rightarrow F $ and  $ \Lm'\colon c^\prime\Rightarrow F $ there are
unique morphisms $ f \colon c\rightarrow l $ and $ f^\prime \colon c'
\rightarrow l' $ such that $\Lm =\lm f$ and $\Lm^\prime
=\lm^\prime f^\prime$.

In the case in which $\Lm=\lm$ and $\Lm^\prime =\lm^\prime$, the
unique $f$ and $f^\prime$ satisfying $\lm=\lm f$ and $\lm^\prime
=\lm^\prime f$ are $f=1_l$ and $f^\prime =1_{l^\prime}$.

In the case in which $\Lm=\lm^\prime$ and $\Lm^\prime =\lm$ we
obtain unique morphisms $g\colon l^\prime\rightarrow l$ and $g^\prime\colon l\rightarrow
l^\prime$ such that $\lm^\prime =\lm g$ and $\lm =\lm^\prime
g^\prime$.

Thus, $\lm =\lm^\prime g^\prime =\lm gg^\prime$. But, we have
already seen that the unique morphism $f$ such that $\lm =\lm f$ is
$1_l$. So, $gg^\prime =1_l$. Similarly, $\lm^\prime =\lm
g=\lm^\prime g^\prime g$ and a similar uniqueness argument made above shows
us that $g^\prime g=1_{l^\prime}$.

So, $g$ and $g^\prime$ are inverse isomorphisms between $l$ and $l^\prime$ as
required.

\end{proof}

\end{document}
