\documentclass[main.tex]{subfiles}

\begin{document}

\paragraph{}
\begin{exercise}
Give a second proof of Proposition 3.1.7 by using the universal properties of
each of a pair of limit cones $ \lambda \colon l \Rightarrow F $
and $ \lambda' \colon l' \Rightarrow F $ to directly construct the unique
isomorphism $ l \cong l' $ between their apexes.
\end{exercise}
\paragraph{Proof.}

\begin{quote}
	{\small Note that this proof is not mine---Dr. Pardue mentioned it in class; I'm just rewriting it.}
\end{quote}

Since $ \lambda $ and $ \lambda' $ are limit cones, for any cones $ \Lambda \in
\mathsf{Cone}(l, F) $ and  $ \Lambda' \in \mathsf{Cone}(l', F) $ there are
unique natural transformations $ f \colon \Lambda \Rightarrow \lambda $ and $ g
\colon \Lambda' \Rightarrow \lambda' $. Furthermore, there are unique natural
transformations $ \eta \colon \lambda \Rightarrow \lambda' $ and $ \epsilon
\colon \lambda' \Rightarrow \lambda $. The following commutative diagrams
illustrate these transformations:
\begin{tikzcd}
	& & l \arrow[rd, "\lambda", bend left] \arrow[rd, "\Lambda"', bend right] &  & l' \arrow[ld, "\Lambda'", bend left] \arrow[ld, "\lambda'"', bend right] & & & 	\Lambda \arrow[d, "f"] & \Lambda' \arrow[d, "g"'] \\
	& & & F & & & & \lambda \arrow[r, "\eta", bend right] & \lambda' \arrow[l, "\epsilon"', bend right]
\end{tikzcd}
where the left diagram describes the various cones over $ F $ and the
right diagram describes the transformations between those cones.

Because these diagrams commute, we can compose $ g\Lambda' = \lambda' = \eta f
\Lambda $ and $ f \Lambda = \lambda = \epsilon g \Lambda' $. We can substitute
in both of these compositions to obtain $\lambda' = \eta (\epsilon g \Lambda') $
and $ \lambda = \epsilon (\eta f \Lambda) $, and from those obtain $ \lambda' =
\eta \epsilon \lambda' $ and $ \lambda = \epsilon \eta \lambda $. So we can
conclude that $ \epsilon \eta = 1_\lambda $ and $\eta \epsilon = 1_{\lambda'} $,
which means that for any particular $ m \in \lambda, m' \in \lambda' $, (that
is, for any leg of $ \lambda: l \Rightarrow F $ or $ \lambda': l' \Rightarrow F
$) $ \epsilon_m \eta_m m = m \epsilon_m \eta_m $ and $ \eta_{m'} \epsilon_{m'}
m' = m' \eta_{m'} \epsilon_{m'} $. So $ \epsilon $ and $ \eta $ form a natural
isomorphism, illustrated in part by the following diagram:

\[\begin{tikzcd}
	&&&& l \arrow[r, "\eta_l"'] \arrow[d, "\lambda_x"] \arrow[ldd, "\lambda_y" description] \arrow[llddd, "\lambda_{...}"'] & l' \arrow[l, "\epsilon_{l'}"', bend right] \arrow[d, "\lambda'_x"'] \arrow[rdd, "\lambda'_y" description] \arrow[rrddd, "\lambda'_{...}"] &  &  \\
	&&&& x \arrow[r, "\eta_x"', bend right] & x \arrow[l, "\epsilon_x"'] &  &  \\
	&&& y \arrow[rrr, "\eta_y"', bend right] &  &  & y \arrow[lll, "\epsilon_y"] &  \\
	&&... \arrow[rrrrr, "\epsilon_{...}"'] &  &  &  &  & ... \arrow[lllll, "\eta_{...}", bend left]
\end{tikzcd}\]
where the $x,y,\dots$ are objects in $ F $. The most relevant parts of this diagram are the morphisms $ \eta_l $ and $ \epsilon_l' $, (the top set of arrows) denoting the components of $ \eta $ and $ \epsilon $ taking $ l $ to $ l' $ and vice versa, respectively. As we determined earlier, $ \eta $ and $ \epsilon $ are a natural isomorphism, so $ \eta_l $ and $ \epsilon_l' $ must also be an isomorphism. Finally, recall that for a given $ \Lambda, \Lambda'; \eta $ and $ \epsilon $ are unique. So we have constructed a canonical isomorphism between $ l $ and $ l' $, given by the component morphisms $ \eta_l: l \to l' $ and $ \epsilon_l': l' \to l. $ $ \Box $

\end{document}
