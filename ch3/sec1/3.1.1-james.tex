\documentclass[main.tex]{subfiles}

\begin{document}

\paragraph{}
\begin{exercise}
For a fixed diagram $F \in \sC^\sJ$, describe the actions of the cone functors $
\Cone(-,F)\colon \sC^{op} \rightarrow \Set$ and $ \Cone(F,-)\colon \sC
\rightarrow \Set$ on morphisms in $\sC$.
\end{exercise}

\begin{proof}
In the contravariant case, the domain of the cone functor $ \Cone(-,F)$
on morphisms $f\colon c \rightarrow c'$ in $\sC^{op}$, is a morphism that takes the
constant functor $c$ to $c'$, and thus is a natural transformation $f\colon c
\To c'$. The contravariant cone funtor then sends $f$ to the mapping
$f\colon \Cone(c',F)\rightarrow \Cone(c,F)$. From this mapping, for a cone
in the domain of $f^*$, which is a natural transformation $\lambda\colon c' \To
F$, the cone functor results in the precomposition of $f$ with $\lambda$,
namely, $\lambda f$.

Likewise for the covariant case, the cone functor $ \Cone(F,-)\colon \sC
\rightarrow \Set$ sends morphisms from $c$ to $c'$ to postcomposition of
the natural transfomations comprising the sets cones under $c$ with $f$,
namely, $f\lambda$.

\end{proof}

\end{document}
