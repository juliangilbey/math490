\documentclass[main.tex]{subfiles}
\begin{document}

\settheorem{3}{2}{16}
\begin{example}
	An \textbf{idempotent} is an endomorphism $\func{e}{A}{A}$ of some object so
	that $e\cdot e = e$. The limit of an idempotent in $\Set$ is the set of
	cones with summit $1$, i.e., the set of $a\in A$ so that $ea=a$. This is the
	set of \textbf{fixed points} for the idempotent $e$, often denoted by $A^e$.

	Alternatively, applying Theorem 3.2.13 in the simplified form in Exercise
	3.2.ii, the limit $A^e$ is constructed as the equalizer
	$$
	\xymatrix{
		A^e\ar@{>->}[r]^s & A\ar@<-.5ex>[r]_e \ar@<.5ex>[r]^1 & A
	}
	$$
	The universal property of the equalizer implies that $e$ factors through $s$
	along a unique map $r$.
	$$
	\xymatrix{
		A\ar@{-->}[d]_r\ar[dr]^e\\
		A^e\ar@{>->}[r]^s & A\ar@<-.5ex>[r]_e \ar@<.5ex>[r]^1 & A
	}
	$$
	The factorization $e=sr$ is said to \textbf{split} the idempotent. Now $srs
	= es = s$ implies that $rs$ and $1_{A^e}$ both define factorizations of the
	diagram
	$$
	\xymatrix{
		A^e\ar[dr]^s\ar@<-.5ex>@{-->}[d]_{rs} \ar@<.5ex>@{-->}[d]^{1_{A^e}}\\
		A^e\ar@{>->}[r]^s & A\ar@<-.5ex>[r]_e \ar@<.5ex>[r]^1 & A
	}
	$$
	Uniqueness implies $rs=1_{A^e}$ so $A^e$ is a retract of $A$. Conversely,
	any retract diagram
	$$
	\xymatrix{
		B\ar@{>->}[r]^s & A\ar@{->>}[r]^r & B & rs=1_B
	}
	$$
	gives rise to an idempotent $sr$ on $A$, which is split by $B$.
\end{example}
\popthm

\begin{exercise}
	Dualize the construction of Example 3.2.16 to express the splitting of an
	idempotent as a coequalizer. Explain why these colimits (or limits) are
	preserved by any functor and conclude that splittings of idempotents commute
	with both limits and colimits of any shape.
\end{exercise}
\begin{proof}
	As in the example, the colimit $A_e$ can be constructed as a coequalizer:
	$$
	\xymatrix{
		A\ar@<-.5ex>[r]_e \ar@<.5ex>[r]^1 & A\ar@{->>}[r]^s & A_e
	}
	$$
	And the universal property implies that $e$ factors through $rs$ where $r$ is unique:
	$$
	\xymatrix{
		A\ar@<-.5ex>[r]_e \ar@<.5ex>[r]^1 & A\ar[dr]^e\ar@{->>}[r]^s & A_e\ar@{-->}[d]^r \\
		& & A\\
	}
	$$
	where $s$ has become an epimorphism, and the factorization $e=rs$ splits the idempotent (note the change in order of the factorization: $rs$, from the original category's $sr$). Then we have $srs = se = s$ implies that $sr$ and $1_{A_e}$ are factorizations of
	$$
	\xymatrix{
		A\ar@<-.5ex>[r]_e \ar@<.5ex>[r]^1 & A\ar[dr]_s\ar@{->>}[r]^s & A_e\ar@<.5ex>@{-->}[d]^{1_{A_e}}\ar@<-.5ex>@{-->}[d]_{sr} \\
		& & A_e\\
	}
	$$
	And uniqueness implies that $sr = 1_{A_e}$, so $A_e$ is a retract of $A$. Conversely, any retract of the form
	$$
	\xymatrix{
		B\ar@{>->}[r]^r & A\ar@{->>}[r]^s & B
	}
	$$
	gives an idempotent $sr$ on $A$, which is split by $B$. (This gives a
	construction of Example 1.2.9 and its dual, of split monomorphisms and split
	epimorphisms.)

	To see that these limits and colimits are preserved by any functor $F$, note
	that we have for any idempotent $\func{e}{A}{A}$, that $Fe\circ Fe =
	F(e\circ e) = Fe$. That is, functors take idempotents to idempotents. Then,
	since all splittings of idempotents can be expressed as an equalizer (with a
	limit $A^e$) or a coequalizer (with a colimit $A_e$), it must be that any
	functor preserves these limits (or colimits).

	Finally, since these splittings of idempotents can be expressed as both
	limits and colimits, by Theorem 3.8.1 (that limits commute with limits) and
	its dual (for colimits), they commute with limits and colimits of any shape.
\end{proof}

\end{document}
