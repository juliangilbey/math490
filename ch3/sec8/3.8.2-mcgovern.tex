\documentclass[main.tex]{subfiles}

\begin{document}

\paragraph{}

\begin{exercise}
 Show that if $G$ and $H$ are groups whose orders are coprime, $\cat{B}G$-indexed limits commute with $\cat{B}H$-indexed colimits in $\Set$.
\textit{Note: The proof strategy is based on that of Lemma 3.1 in the cited paper}

\begin{proof}
We know that by definition, $\cat{B}G$-indexed limits are the sets $X^{G} =
\{x \in X \mid gx = x \text{ for all } g \in G\}$, and that  $\cat{B}H$-indexed
colimits in $\Set$ are the sets of orbits of the group action induced by $H$.
We must show that the fixed "points" under $G$ in the set of orbits of $H$ are
equivalent to the orbits of $H$ that consist of fixed points under $G$. We also
know that $\lim_{G}\colim_{H} F $ and $\colim_{H}\lim_{G} F$ are defined for a
functor $F\colon \cat{B}G \times \cat{B}H \rightarrow \Set$. So we can
interpret $gx$ as $(g,1_H)x$ and $hx$ as $(1_G, h)x$ and see that the actions
of $G$ and $H$ commute.
First, consider an orbit in  $X$ that is fixed under the action of $G$. This
means that for all $x \in X$ and $g \in G$, $gx = hx$ for some $h \in H$.
Because $g = (g,1_H)$ and $h = (1_G,H)$ commute, we have that $g^{i}x =
h^{i}x$. We see that the set of $g^{i}x$ forms an orbit of the cyclic group
generated by $g$ acting on $X$ and similarly for the set of element $h^{i}x$.
By this, we know that the  smallest $n > 0$ such that $g^{n}a = a$ divides
$|g|$ and that the  smallest $m > 0$ such that $h^{m}a = a$ divides $|h|$ and
that $n = m$ by the equality $g^{i}x = h^{i}x$ that we stated before. Since
$|G|$ and $|H|$ are coprime and therefore so are $|g|$ and $|h|$, $n = 1$ and
$gx = x$. This holds for all $g \in G$. So every $x$ in an orbit under the
action of $H$ that is fixed by the action of $G$ is a fixed point of $G$ and we
have equality between the fixed orbits of $H$ under $G$ ($\lim_{G}\colim_{H} F
$) and the orbits under $H$ of the fixed points of $G$ ($\colim_{H}\lim_{G}
F$).  
\end{proof}
\textbf{Reference:}

M. Bjerrum, P. Johnstone, T. Leinster, W.F. Sawin, Notes on Commutation of Limits and Colimits, Theory and Applications of Categories, 30, 527-532 (2015).
\end{document}
