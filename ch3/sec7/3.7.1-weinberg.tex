\documentclass[main.tex]{subfiles}
\begin{document}

\paragraph{}
\begin{exercise}
Complete the proof of Lemma 3.7.1 by showing that an initial object defines a
limit of the identity functor $1_\sC: \sC \rightarrow \sC$.
\end{exercise}

\begin{proof}
Call the initial object we are considering $i$.  We are looking for a universal
cone, $\alpha\colon i \Rightarrow 1_\sC$.  We note that $i$ has unique morphisms to
arbitrary elements $c,d$ in $\sC$ as it is initial.  Let $f: i \rightarrow c$,
$g\colon i \rightarrow d$ be the unique morphisms, and let $h\colon c \rightarrow d$ be
arbitrary.  We note then that $g = hf$, as there is only one morphism from $i$
to $d$ in $\sC$.  This establishes that we have a cone over $1_\sC$.


Now we must show it is universal.  We consider another object $n$ as the apex
that forms a valid cone $\beta\colon n \Rightarrow 1_\sC$.  Because $i$ is initial, we
know there exists a unique morphism $h: i \rightarrow n$.  Considering $h$ as a
natural transformation between constant functors, we now must show that $\alpha
= \beta\cdot h$ (vertical composition), noting that both sides are $i
\Rightarrow 1_\sC$.
We look at any leg $t$ of the cone.  We see that $(\beta \cdot h)_t = \beta_t *
h$ is a morphism with domain $i$ that shares a codomain with $\alpha_t$, and
thus must be the same as $\alpha_t$ because $i$ is initial and morphisms from
$i$ are unique.  Thus, as all of the legs are the same, $\alpha$ must equal
$\beta \cdot h$, which guarantees that every cone $\beta$ factors through
$\alpha$.

\end{proof}

\end{document}
