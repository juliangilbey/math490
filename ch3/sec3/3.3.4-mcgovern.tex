\documentclass[main.tex]{subfiles}
\begin{document}
\paragraph{}
\begin{exercise}
Prove that $F\colon\sC \rightarrow \sD$ creates limits for a particular class of diagrams if both of the following hold:
\begin{enumerate}
    \item $\sC$ has those limits and $F$ preserves them.
    \item $F\colon \sC \rightarrow \sD$ reflects isomorphisms. 
\end{enumerate}
\end{exercise}

\begin{proof}
The first condition above means that for every $K\colon \sJ \rightarrow \sC$, $K$ has a
limit whose limit cone is represented by $\lambda\colon\lim K \Rightarrow K$ and
that $F\lambda\colon F\lim K \Rightarrow FK$ is a limit cone for FK. The second
condition means that for every morphism $f$ in $\sC$, if $Ff$ is an isomorphism
in $\sD$, then so is $F$. 
We must show that whenever $FK\colon \sJ \rightarrow \sD$ has a limit in $\sD$, there is
some limit cone over $FK$ that can be lifted to a limit cone over $K$ and that
$F$ reflects all limits. To see the first requirement, we see that for any $K$,
there exists a limit cone $F\lambda$ as defined before by condition 1 that can
be lifted to a limit cone $\lambda$ over $K$. We now must show that $F$
reflects all limits. 
Now suppose that $\nu\colon c \Rightarrow K$ is a cone over $K$ and that $F\nu\colon Fc
\Rightarrow FK$ is a limit cone. Note that we also have the limit cone
$F\lambda\colon F\lim K \Rightarrow FK$. Consider the unique morphism $f:Fc
\rightarrow F\lim K$ such that $(F\lambda)f=F\nu$. We know that $f$ is an
isomorphism, becuause $F\nu$ is also a limit cone.  We must now show that this
ismorphism is in the image of $F$, that is that $f = Fg$ for some $g: c
\rightarrow \lim K$. We see that this morphism is the image of the unique
morphism that factors the legs of $\nu$ through $\lambda$ by the uniqueness of
$f$. 
So we see that $Fg = f$ is an canonical isomorphism in $\sD$, and therefore by
our second condition that $F$ reflect isomorphisms, $g$ is a canonical
isomoprhism between $c$ and $\lim K$, so we see that $\nu$ is also a limit cone
over $K$. Therefore $F$ reflects limits and we have shown that $F$ creates
limits. 
\end{proof}
\end{document}
