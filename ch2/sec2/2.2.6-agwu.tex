\documentclass[main.tex]{subfiles}
\begin{document}

\paragraph{}
\begin{exercise}
	Do there exist any non-identity natural endomorphisms of the category of
	spaces? That is, does there exist any family of continuous maps
	$X\rightarrow X$, defined for all spaces $X$ and not all of which are
	identities, that are natural in all maps in the category $\Top$?
\end{exercise}

\begin{proof}
	Consider the topological space with one element $\{x\}$. Suppose $\al$ is a
	natural endomorphism of $1_{\Top}$, then for any topological space $Y$ we
	have the following commutative diagram:
	\[\xymatrix{\{x\} \ar[d]^{\al_{\{x\}}} \ar[r]_{f_{y}} & Y\ar[d]^{\al_{Y}}\\
	\{x\} \ar[r]^{f_{y}} &Y  }\] where $f_{y}(x) = y$ for all $y \in Y$.
	$\al_{\{x\}} = 1_{\{x\}}$ trivially, thus from the commutative diagram we
	get that $f_{y} = \al_{Y}\cdot f_{y}$. From this, it is clear that $y =
	a_{Y}\qty{y}$ for all $y \in Y$. Thus $\al_{Y}$ is the identity function on
	$Y$. Therefore, the only natural endomorphism of $1_{\Top}$ is the identity
	natural endomorphism.
\end{proof}

We have shown for $\Top$, that its identity functor only has the identity
natural transformation. For the rest of the paper, we will generalize our result
above to a concrete category $\sC$.

\begin{definition}
	Suppose some $\sC$ has a terminal object $\om$, we definition the following
	three terms
	\begin{enumerate}
		\item any morphism $\func{f}{\om}{c}$ for $c \in \ob \sC$ is called a
			{\bf global element} of $c$
		\item $\sC$ is {\bf well-pointed} if and only if for any two morphisms
			$\func{f,g}{c}{d}$ of $\sC$, if $f \cdot h = g \cdot h$ for every
			global element $h$ of $c$, then $f = g$
		\item We will call an object $\al_{s}$ {\bf weakly initial} if and only
			if $\sC\qty{\al_{s},c}$ contains at most one morphism for all
			objects $c$
	\end{enumerate}
\end{definition}

Now we will show the following:

\begin{lemma}
	If $\sC$ has a terminal object $\om$ which is not weakly initial, is well
	pointed and all of its  objects that aren't weakly initial have global
	elements, then the functor $1_{\sC}$ only has the identity natural
	endomorphism.
\end{lemma}

\begin{proof}
	Suppose $\be$ is a natural endomorphism of $1_{\sC}$. apply the natural
	endomorphism to $\func{f}{\om}{c}$ where $c$ is  not weakly initial. This
	results in the diagram $$\xymatrix{ & \om \ar[d]_{\be_{\om}} \ar[r]^{f} &
	c\ar[d]^{\be_{c}}  \\   &\om \ar[r]_{f} &c  } $$ Thus $f \cdot \be_{\om} =
	\be_{c} \cdot f$. $\be_{\om} = 1_{\om}$ since $\om$ has a unique
	endomorphism, so we get $1_{c} \cdot f  = \be_{c} \cdot f$. This holds for
	every global element of $c$ due to naturality. Since $\sC$ is well-pointed,
	$\be_{c} = 1_{c}$. Since every non weakly initial object $c$ has a global
	element, then $\be_{c} = 1_{c}$. For weakly initial object $\al_{s}$, it has
	a unique endomorphism by definition. So, $\be_{\al_{s}} = 1_{\al_{s}}$. Thus
	$\be$ is the identity natural endomorphism on $1_{\sC}$. Therefore,
	$1_{\sC}$ has only the identity natural endomorphism.
\end{proof}

From now on we will be working with a concrete category $\sC$ with terminal
object $\om$, which is not weakly initial, where $\func{U}{\sC}{\Set}$ is a
faithful functor. We will generalize the last two properties in the definition
above using the following lemma:

\begin{lemma}
	Suppose we have concrete category $\sC$ with corresponding faithful functor
	$U$ and terminal object $\om$ which is not weakly initial , then the
	following holds:
	\begin{enumerate}
		\item If $U$ is a subfunctor of $\sC\qty{\om,-}$ then $\sC\qty{\om,-} \cong U$
		\item If $\sC\qty{\om,-} \cong U$ then $\sC$ is well-pointed, all not
			weakly initial objects of $\sC$ have global elements
	\end{enumerate}
\end{lemma}

Before we prove this, we will prove another lemma

\begin{lemma}
	Suppose we have concrete category $\sC$ with corresponding faithful functor
	$U$,  a terminal object $\om$ which is not weakly initial and $\al_{s}$ is a
	weakly initial object in $\sC$ , we have the following properties:
	\begin{enumerate}
		\item $U$ reflects weakly initial objects
		\item $\sC\qty{\omega,\al_{s}} = \emptyset$
		\item $\Set\qty{\{x\},A } \cong A$ for all sets $A$
		\item $\Set$ is well-pointed
		\item $\sC\qty{\om,-}$ has a unique endofunctor
	\end{enumerate}
\end{lemma}

\begin{proof}
	Part 1 follows from $U$ being faithful. The only weakly initial set is the
	empty set. If there exist some object $c$ in $\sC$ such that $Uc =
	\emptyset$, then for any object $d$ in $\sC$, there is an injection between
	$\sC\qty{c,d}$ and $\Set\qty{\emptyset,Ud}$. Since $\Set\qty{\emptyset,Ud}$
	has exactly one element, then $\sC\qty{c,d}$ has at most one element. Thus
	$c$ is weakly initial.

	For Part 2, suppose $f \in \sC\qty{\omega,\al_{s}} $, then $f$ is an
	isomorphism since $\om$ and $\al_{s}$ have unique endomorphisms. Then
	$\sC\qty{\omega,-} \cong \sC\qty{\al_{s},-}$ making $\om$ a weakly initial
	object, which is a contradiction. Therefore, $\sC\qty{\omega,\al_{s}} =
	\emptyset$.

	For part 3,  we can define a function from $\Set\qty{\{x\},A }$ to $A$ by
	taking a function $g$ from $\Set\qty{\{x\},A }$ and mapping it to
	$g\qty{x}$. This function is clearly injective since each function in
	$\Set\qty{\{x\},A }$ only has one element in its domain, and is clearly
	surjective since each element $y$ in $A$ corresponds to the function mapping
	$x$ to $y$. Thus we have a bijection, therefore $\Set\qty{\{x\},A } \cong
	A$.

	For part 4, take functions $\func{f,g}{A}{B}$ and suppose $f \cdot h = g
	\cdot h$ for every global element $h$ of $A$. By the proof in part 3, this
	means that $f\qty{x} = g\qty{x}$ for all $x$ in $A$. Thus $f = g$.
	Therefore, $\Set$ is well-pointed.

	For part 5 consider an endofunctor $\gm$ of $\sC\qty{\om,-}$, which gives
	the following commutative diagram for global element $f$ of non weakly
	initial object $c$:
	\[\xymatrix{\sC\qty{\om,\om} \ar[d]_{\gm_{\om}}
			\ar[r]^{f_{*}} & \sC\qty{\om,c}\ar[d]^{\gm_{c}}  \\
	\sC\qty{\om,\om}\ar[r]_{f_{*}} &\sC\qty{\om,c}  } \]
	taking
	the unique endomorphism of $\om$ one gets that $f = \gm_{c}\qty{f}$. Thus
	$\gm_{c}$ is the identity morphism. for weakly initial object $\al_{s}$,
	$\sC\qty{\om,\al_{s}} = \emptyset$, thus $\gm_{\al_{s}} = 1_{\emptyset}$.
	Thus $\gm$ is the identity natural endomorphism. Therefore, $\sC\qty{\om,-}$
	has a unique natural endomorphism.
\end{proof}

Now we will prove lemma 2.2.3

\begin{proof}
	Part 1:
	By Yoneda lemma, $\Hom\qty{\sC\qty{\om,-},U} \cong U\om$. Since $U$ reflects
	weakly initial objects, there is at least one natural transformation $\de$
	from $\sC\qty{\om,-}$ to $U$. By our assumption, there is a monic natural
	transformation $\be$ from $U$ to $\sC\qty{\om,-}$. Since $\sC\qty{\om,-}$
	has a unique endomorphism, $\be \cdot \de = 1_{\sC\qty{\om,-}}$. This means
	that $\be$ is also epic. Thus every morphism $\be_{c}$ for $c \in \ob \sC$
	is injective and surjective, and thus bijective. So $\be$ is a natural
	isomorphism. Therefore, $\sC\qty{\om,-} \cong U$.

	Part 2: Suppose for $\func{f,g}{c}{d} \in \mor \sC$ we have that $f \cdot h
	= g \cdot h$ for every global element $h$ of $c$. Then $Uf \cdot Uh = Ug
	\cdot Uh$ .  Since $U \cong \sC\qty{\om,-}$,$U\om$ has only one element
	thus$Uh$ is a global element of $Uc$. Since we have that $\sC\qty{\om,c}
	\cong Uc \cong \Set\qty{U\om,Uc}$, then $Uh$ range over every global element
	of $Uc$. Since $\Set$ is well pointed, then $Uf = Ug$. Since $U$ is
	faithful, $f = g$. Therefore, $\sC$ is well-pointed.

	For any non weakly initial object $c$ of $\sC$, $Uc$ is non-empty since $U$
	reflects weakly initial objects. Thus $\sC\qty{\om,c}$ is non empty, thus
	every non weakly initial object $c$ has a global element.
\end{proof}

Now let us take an equivalence of categories $\func{T}{\sC}{\sC}$. This is our main result:

\begin{theorem}
	Suppose we have concrete category $\sC$ with corresponding faithful functor
	$U$ and terminal object $\om$ which is not weakly initial. If $U $ is a
	subfunctor of  $\sC\qty{\om,-}$ and $\func{T}{\sC}{\sC}$ is an equivalence
	of  then $\Hom\qty{T,T} \cong \Hom\qty{C\qty{\om,-},U }$
\end{theorem}

\begin{proof}
	Since $U$ is a subfunctor of $\sC\qty{\om,-}$, $U \cong \sC\qty{\om,-}$. By Yoneda
	Lemma and since $U$ preserves terminal objects, there is a unique natural
	isomorphism $\be$ from $\sC\qty{\om,-}$ to $U$. Take a natural endomorphism
	$\de$ from $\Hom \qty{T,T}$. Since $T$ is an equivalence of categories, there exist a
	functor $T_{0}$ such that $T_{0}\cdot T \cong 1_{C}$. define the natural transformation $T_{0}\de$
	where its components are the functor $T_{0}$ applied to the components of $\de$. The
	fact that $T_{0}\de$ is a natural transformation follows from the naturality of $\de$ and
	the fact that $T_{0}$ is full and faithful. Thus $T_{0}\de$ is a natural endomorphism of
	$T_{0}\cdot T$. Since $T_{0}\cdot T \cong 1_{C}$, there exist an natural isomorphism $\gm$ from
	$T_{0}\cdot T$ to $1_{C}$. composing $\gm$ and $\inv{\gm}$ to $T_{0}\de$ gives us $\gm \cdot T_{0}\de \cdot \inv{\gm}$ which is an endomorphism of $1_{C}$ We shall refer to $\gm \cdot T_{0}\de \cdot \inv{\gm}$ as $\ep$. We will horizontal composition on $\be$ and $\ep$ to obtain $\be * \ep$. The diagram for the horizontal composition is a follows: $$\xymatrix{ &\sC\qty{\omega,c} \ar@{-->}[dr]^{\qty{\be * \ep}_{c}} \ar[d]_{\be_{c}} \ar[r]^{\sC\qty{\omega,\epsilon_{c}}} & \sC\qty{\omega,c}\ar[d]^{\be_{c}}  \\   &Uc \ar[r]_{U\ep_{c}} &Uc  } $$ This shows us that $\qty{\be * \ep}_{c} = U\ep_{c} \cdot \be_{c}$. Define a function $\func{t}{\Hom\qty{T,T}}{\Hom\qty{\sC\qty{\omega,-},U}}$ taking $\de$ to $\be * \ep$. We will show that $t$ is injective. Suppose that there was a $\de '$ such that $\ep ' = \gm \cdot T_{0}\de ' \cdot \inv{\gm}$ and $\be * \ep = \be * \ep'$. Then, $U\ep_{c} \cdot \be_{c} = U\ep_{c}' \cdot \be_{c}$. Since $\be_{c}$ is an isomorphism, it is also epic. Thus $U\ep_{c} = U\ep_{c}'$ and also $\ep_{c} = \ep_{c}'$ since $U$ is faithful. Thus $\ep = \ep'$. Since $\gm$ and $\inv{\gm}$ are isomorphic, they are both monic and epic, so $T_{0}\de = T_{0}\de'$. Since $T_{0}$ is faithful, $\de_{c} = \de_{c}'$ for every $c \in \ob \sC$. Thus $\de = \de'$. So $t$ is an injection from $\Hom \qty{T,T}$ to $\Hom\qty{C\qty{\om,-},U }$. Since $\Hom\qty{C\qty{\om,-},U }$ has only one element and $\Hom \qty{T,T}$ has at least one element (the identity natural transformation), $t$ is also surjective. Therefore, $\Hom\qty{T,T} \cong \Hom\qty{C\qty{\om,-},U }$.
\end{proof}

An immediate consequence of the theorem is that every equivalence $T$ has only
the identity natural transformation.

This completes our generalization of problem 2.2.vi.

\end{document}
