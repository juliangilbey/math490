\documentclass[main.tex]{subfiles}
	\begin{document}

\begin{lemma}[2.2.vii]

Use the Yoneda lemma to explain the connection between homeomorphisms of the
standard unit interval $I = [0,1] \subset R$ and natural automorphisms of the
path functor $\bold {Path} : \bold {Top} \rightarrow \bold {Set}$.

\end{lemma}

\begin{proof}
Viewing the path functor, $F = \bold {Path}: \bold{Top} \rightarrow
\Set$, in light of the Yoneda lemma, there is a bijection $Hom(
\bold{Top} (I,-),F) \cong FI$, such that $(\alpha : \bold{Top} (I,-)
\Rightarrow F) \mapsto \alpha_I (1_I)$, and for $x \in FI$ with $y \in ob
\Top$, $x \mapsto (\psi (x) : \Top(I,-) \Rightarrow F)$, with
components of the natural transformation $\psi(x)_y : \bold{Top} (I,y)
\rightarrow Fy$. 

Now, for paths in the unit interval, with continuous functions $f$ being mapped
to $Ff(x)$, from the Yoneda lemma again, there is a bijection $Hom(\bold{Path},
\bold{Path}) \cong \bold{Path} (I)$, with $I \mapsto \Top(I,-) =
\bold{Path} (I)$. The covariant functor $\Top^{op} \rightarrow
\bold{Set}^{\Top}$ then gives an isomorphism between endomorphisms of the
unit interval and natural transformations of the path funtor. Thus by the
Yoneda embedding theorem, endomorphisms on $I$, viewed as mophisms between
objects in $End(I)^{\op}$ correspond to natural transformations of the path
funtor. 

Since these endomorphisms can be viewed as monoids, then the isomorphism is also between monoids, and those natural endomorphisms that are natural automorphisms of the path functor, are isomorphic to the homeomorphisms of the unit interval.


\end{proof}	
	
	\end{document}
