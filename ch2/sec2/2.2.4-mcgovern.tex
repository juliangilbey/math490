\documentclass[../../main]{subfiles}

\begin{document}

\paragraph{}
\begin{exercise}
Prove the following strengthening of Lemma 1.2.3, demonstrating the equivalence between an isomorphism in a category and a representable isomorphism between the corresponding co- or contravariant functors: the following are equivalent:
\end{exercise}

\begin{enumerate}[(i)]
\item $f\colon x \rightarrow y$ is an isomorphism in $\sC$.
\item $f_*\colon \sC(-,x) \Rightarrow \sC(-,y)$ is a natural isomorphism. 
\item $f^{*}\colon \sC(y,-) \Rightarrow \sC(x,-)$ is a natural isomorphism. 
\end{enumerate}

\begin{proof}
We prove this as a consequence of the Yoneda embedding theorem. We note that by 
the Yoneda embedding theorem, we have an a fully faithful functor from $\sC$ to 
$\Set^{\sC^{\op}}$ and from 
$\sC^{\op}$ to $\Set^{\sC}$ and that this means that there are bijections 
between $\sC(x,y)$ and $\Set^{\sC^{\op}}(\sC(-,x),\sC(-,y))$ and between  
$\sC^{\op}(x,y)$ and 
$\Set^{\sC}(\sC(y,-),\sC(x,-))$. To see that (i) implies (ii) and (iii), we 
note that if there exists an isomorphism $f\colon x \rightarrow y$, there 
exists at least 
one natural 
transformation $\sC(-,x) \Rightarrow \sC(-,y)$ and $\sC(y,-) \Rightarrow 
\sC(x,-)$ by the bijections previously noted. We also see that the components 
of $f_*$ and 
$f^{*}$ are 
defined by post and pre-composition with $f$, repsectively, and that post and 
pre composition by an isomorphism creates another isomorphism. So all 
components of $f_*$ 
and $f^{*}$ are isomorphisms and therefore $f_*$ and $f^{*}$ are natural 
isomorphisms. 

Now, we show that (ii) implies (i) and that (iii) implies (i). To do this, we 
recall that full and faithful functors create isomorphisms. Since we have that 
$\sC(-,x)$ 
and $\sC(-,y)$ are isomorphic by $f_*$, we know that $x$ and $y$ are also 
isomorphic by $f$, since $f_*$ is the image of $f$. Likewise, if $\sC(y,-)$ and 
$\sC(x,-)$ are 
isomorphic by $f^{*}$, than so are $x$ and $y$ by $f^{*}$, since $f^{*}$ is the 
image of $f$. So we have show that both (ii) and (iii) independently imply (i) 
and that shows the equivalence of all three statements.

\end{proof}

\end{document}
