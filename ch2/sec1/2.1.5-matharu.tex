\documentclass[main.tex]{subfiles}
\begin{document}

\paragraph{}
\begin{exercise}
	The functor of Example 2.1.5(xi) that sends a category to its collection of
	isomorphisms is a subfunctor of the functor of Example 2.1.5(x) that sends
	a category to its collection of morphisms. Define a functor between the
	representing categories $ \mbb{I} $ and $ \mbb{2} $ that induces the
	corresponding monic natural transformation between these representable
	functors.
\end{exercise}

\begin{proof}
	Recall that $\mbb{I}$ and $\two$ are defined as
	\[\xymatrix{A \ar@(ld,lu) \ar@<1ex>[r]^f & B \ar@<1ex>[l]^g \ar@(ur,dr)&}
	\qtextq{and} \xymatrix{&0 \ar@(ld,lu) \ar[r]^h & 1 \ar@(ur,dr)}\]
	respectively.
	We need to construct a functor $ \func{F}{\mbb{2}}{\mbb{I}}.$
	Since functors must preserve isomorphisms, there are only a few reasonable
	options. Define $ F $ as: $ F(0)=A,F(1)=B, F(1_0)=1_A,F(1_1)=1_B,\AND F(h)=
	f.$ Let $\Func{\io}{\iso}{\mor}$ be defined by precomposition by $F,$ that is $
	F\mapsto F\iota. $ This map is a subfunctor of the functor we
	get from the Yoneda lemma
	$$
	\iso \simeq \xymatrix{\Cat(\mbb{I},-)\ar@{=>}[r]^{\io_*}&\Cat(\mbb{2},-)} \simeq \mor.
	$$
	Also $ \iota $ is clearly monic since it is an inclusion map.
\end{proof}

\end{document}
