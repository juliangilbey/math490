\documentclass[../../main]{subfiles}
\begin{document}

\paragraph{}
\begin{exercise}
	Suppose $\func{F}{\sC}{\Set}$ is equivalent to $\func{G}{\sD}{\Set}$ in the sense that there is an equivalence of categories $\func{H}{\sC}{\sD}$ so that $GH$ and $F$ are naturally isomorphic.
	\begin{enumerate}
		\item[(i)] If $G$ is representable, then is $F$ representable?
		\item[(ii)] If $F$ is representable, then is $G$ representable?
	\end{enumerate}
\end{exercise}

Before we begin the main proof, we must proof a lemma:

\begin{lemma}
	Suppose we have isomorphic pairs of functors $\func{F,G}{\sC}{\sD}$,
	$\func{H_{0}, H_{1}}{\sD}{\sE} $, and $\func{K_{0},K_{1}}{\sB}{\sC}$ then $$
	H_{0}F \cong H_{1}G\AND FK_{0} \cong GK_{1}.$$
\end{lemma}

To prove this consider an object $c \in \ob \sC$, then we have the isomorphism
$\func{\al_{c}}{Fc}{Gc}$ such that $\al$ is the natural transformation between
$F$ and $G$. Let $\func{\be}{H_{0}}{H_{1}}$ be a natural isomorphism. Then we
have the commutative diagram.
$$\xymatrix{ &H_{0}Fc \ar@{-->}[dr]^{\gm_{c}} \ar[d]_{\be_{Fc}}
	\ar[r]^{H_{0}\al_{c}} & H_{0}Gc\ar[d]^{\be_{Gc}}  \\   &H_{1}Fc
\ar[r]_{H_{1}\al_{c}} &H_{1}Gc  }
$$ Since functors preserve isomorphisms and
$\be$ is a natural isomorphism, then every morphism in the diagram is an
isomorphism. Thus $H_{0}Fc \cong H_{1}Gc$ for all $c \in \ob \sC$. Setting
$\gm_{c}  =  \be_{Gc} \cdot H_{0}\al_{c}$, we can construct the natural
isomorphism $\func{\gm}{H_{0}F}{H_{1}G}$, and therefore $H_{0}F \cong H_{1}G$.
The second part of the lemma follows from duality.

\begin{proof}
	Now we prove that the answer to part (i) is indeed yes. If $G$ is
	representable, then $G \cong D\qty{d,-}$ for $d \in \ob \sD$. By our lemma,
	$F \cong GH \cong D\qty{d,-}H $. Since $H$ is an equivalence of categories,
	$H$ is essentially surjective, meaning there is a $c \in \ob \sC$ such that
	$Hc \cong d$. For such a $c$ we want to show that
	$$D\qty{d,-}H \cong C\qty{c,-}. $$
	For any $c' \in \ob \sC$, we  want to show that a function
	$\func{f_{c'}}{C\qty{c,c'}}{D\qty{d,Hc'}}$  defined as
	$$f_{c'}(h) = Hh\cdot\pi,$$
	where $\func{\pi}{d}{Hc}$ is an isomorphism, is a bijection.
	To show injectivity,  suppose for $h,k \in C\qty{c,c'}$, $f_{c'}(h) =
	f_{c'}(k)$. Then, $Hh\cdot\pi = Hk\cdot\pi$. Since $\pi$ is isomorphic, we
	have that $Hh = Hk$. Since $H$ is an equivalence of categories, $H$ is a
	fully faithful functor, thus $h = k$. To show surjectivity, take $l \in
	D\qty{d,Hc'}$ and compose on the right with $\inv{\pi}$. Then we get
	morphism $\func{l \cdot \inv{\pi}}{Hc}{H'c}$. Since $H$ is full, there
	exists $m \in C\qty{c,c'}$ such that $Hm = l \cdot \inv{\pi}$. Right
	composing with $\pi$ results in
	$Hm\cdot\pi = l $, thus confirming surjectivity. A function $f_{c'}$ for
	all $c' \in \ob \sC$ is a bijection. All of the functions $f_{c'}$ form a
	natural isomorphism $\func{f}{C\qty{c,-}}{D\qty{d,-}H}$ showing that
	$D\qty{d,-}H \cong C\qty{c,-} $. Thus $F \cong C\qty{c,-}$, showing that
	$F$ is representable.

	Now we prove that the answer to part (ii) is also yes. If $F$ is
	representable, then $F \cong C\qty{c,-}$ for $c \in \ob \sC$. Thus,
	$C\qty{c,-} \cong F \cong GH  $. Since $H$ is an equivalence of categories,
	there exists a functor $\func{H'}{\sD}{\sC}$ such that $HH' \cong 1_{\sD}$
	and $H'H \cong 1_{\sC}$. By our lemma, $C\qty{c,-}H' \cong FH' \cong GHH'
	\cong G  $ Since $H'$ is an equivalence of categories, $H'$ is essentially
	surjective, meaning there is a $d \in \ob \sD$ such that $H'd \cong c$. For
	such a $d$ we want to show that $$C\qty{c,-}H' \cong D\qty{d,-}. $$ For any
	$d' \in \ob \sD$, we  want to show that a function
	$\func{g_{d'}}{D\qty{d,d'}}{C\qty{c,H'd'}}$  defined as $$g_{d'}(h) =
	H'h\cdot\phi $$ where $\func{\phi}{c}{H'd}$ is an isomorphism, is a
	bijection. To show injectivity,  suppose for $h,k \in D\qty{d,d'}$,
	$g_{d'}(h) = g_{d'}(k)$. Then, $H'h\cdot\phi = H'k\cdot\phi$. Since $\phi$
	is isomorphic, we have that $H'h = H'k$. Since $H'$ is an equivalence of
	categories, $H'$ is a fully faithful functor, thus $h = k$. To show
	surjectivity, take $l \in C\qty{c,H'd'}$ and compose on the right with
	$\inv{\phi}$. Then we get morphism $\func{l \cdot \inv{\phi}}{H'd}{H'd'}$.
	Since $H'$ is full, there exists $m \in D\qty{d,d'}$ such that $H'm = l
	\cdot \inv{\phi}$. Right composing with $\phi$ results in
	$H'm\cdot\phi = l $, hence $ f_{c'} $ is surjective. A function $g_{d'}$ for
	all $d' \in \ob \sD$ is a bijection. All of the functions $g_{d'}$ form a
	natural isomorphism $\func{g}{D\qty{d,-}}{C\qty{c,-}H'}$ showing that
	$C\qty{c,-}H' \cong D\qty{d,-} $. Therefore since $G \cong D\qty{d,-},$
	$G$ is representable.

	The dual cases of (i) and (ii) follow from changing $\sC$ and $\sD$ to their
	respective opposite category.
\end{proof}

\end{document}
