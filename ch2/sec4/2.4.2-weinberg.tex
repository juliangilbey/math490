\documentclass[../../main]{subfiles}
\begin{document}

\paragraph{}

\begin{exercise}
	Characterize the terminal objects of $\sC/c$.
\end{exercise}

\begin{proof}
We claim that one terminal object in $\sC/c$ is the identity morphism $1_c$.
To see this, we look at the diagram for a morphism in $\sC/c$.

\xymatrix{x \ar[rr]^h\ar[dr]_g && c\ar[dl]^{1_c}\\ &c & }

We must show that there is a unique $h$ for any given object $g\colon x \rightarrow c$.  But by the commutative diagram above, we see $g = 1_c \circ h = h$.

Thus, this $h$ is uniquely defined, and there is a unique morphism that takes us from $f$ to $1_c$. This makes $1_c$ a terminal object.

 By corollary 2.3.2, any two terminal objects in $\sC/c$ are uniquely isomorphic.  

Suppose $f\colon x\rightarrow c$ is isomorphic to $1_c$.  We look at the composition diagram:

\xymatrix{x\ar[r]^f\ar[dr]_f &c\ar[d]^{1_c}\ar[r]^j&x\ar[dl]^f\\ &c & }

We can see that this diagram requires $fj = 1_c$ and the top row requires that $jf = 1_x$ for the composition to be the identity.  Similarly, these criteria make the composition hold in the other order:

\xymatrix{c\ar[r]^j\ar[dr]_{1_c} &x\ar[d]^{f}\ar[r]^f&c\ar[dl]^{1_c}\\ &c & }

Thus, for an isomorphism to exist between $1_c$ and arbitrary $f$, we must have
that $f$ is an isomorphism $x \rightarrow c$.  Thus, the terminal objects are
the isomorphisms $x \rightarrow c$ in $\sC$.  This can be seen in the diagram
below.

\xymatrix{x \ar[rr]^h\ar[dr]_g && c\ar[dl]^{f}\\ &c & }

Note that if $f$ is an isomorphism, we can let $h = jg$, where $j$ is the unique inverse of $f$.  Thus, $fh = fjg = g$, and the diagram commutes.
\end{proof}

\end{document}
