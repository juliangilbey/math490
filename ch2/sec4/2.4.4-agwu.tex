\documentclass[../../main]{subfiles}

\begin{document}
	
\paragraph{}

	\begin{exercise}
		
	Explain the sense in which the Sierpinski space is the universal topological
	space with an open subset.
		
	\end{exercise}
	
	
	\begin{proof}
		
		Let $\func{\mathcal{O}}{\Top^{\op} }{\Set}$ be a functor that
		maps each topological spaces to its set of open sets and each
		continuous function $\func{f}{A}{B}$ to function
		$\func{g}{\mathcal{O}B}{\mathcal{O}A}$ where $g(U) = \inv
		f(U)$. Let $\cS$ be the Sierpinski space and $\{x\}$ be the
		only non-trivial open set of $\cS$. Since $\cS$ represents
		$\mathcal{O}$, then for each topological space $T$, each
		continuous function from $T$ to $\cS$ corresponds bijectively
		with each open set of $T$. In fact the bijection would be
		defined by mapping each continuous function $\func{f}{T}{\cS}$
		to $\inv f\qty{\{x\}} = \mathcal{O}f\qty{\{x\}}$. In the
		category of elements $\int \mathcal{O} $, this means that for
		object $\qty{T,U}$, there exists a unique morphism with domain
		$\qty{T,U}$ and codomain $\qty{\cS, \{x\}}$ since there is only
		one continuous function $\func{f}{T}{\cS}$ such that
		$\mathcal{O}f\qty{\{x\}} = U$. Thus the Sierpinski space is the
		universal topological space with an open subset in the sense
		that $\qty{\cS,\{x\}}$ is terminal in $\int \mathcal{O} $.
	
		
	\end{proof}

\end{document}
