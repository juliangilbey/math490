\documentclass[../../main]{subfiles}

\begin{document}

\paragraph{}

\begin{exercise}
For a locally small category $\sC$, regard the two-sided represented functor
$\Hom(-,-)\colon \sC^{op} \times \sC \rightarrow \Set$ as a covariant functor of its
domain $\sC^{\op} \times \sC$.  The category of elements of Hom is called the
twisted arrow category.  Justify this name by describing its objects and
morphisms.
\end{exercise}

\begin{proof}
We begin by describing the objects of the category of elements.  As in the
definition, the objects are pairs $(c,x)$.  In this case, $c$ is an object in
$\sC^{op} \times \sC$, which in this context is a pair of elements in $\sC$,
$(c_1,c_2)$.  We also know $x \in \Hom(c_1,c_2)$, and thus $x$ is a morphism
between $c_1$ and $c_2$.  But we note that the objects can be described by just
$x$, as $c$ is merely the domain and range of $x$.  Thus, the objects are
morphisms, or "arrows."

Next, we consider the morphisms.  A morphism h in $\Hom(-,-)$ will take
$(f,g)$, where $f\colon c_1 \rightarrow c_2$ and $g\colon c_3 \rightarrow c_4$ are
morphisms in $\sC$, to a function that takes morphism $x\colon c_2 \rightarrow c_3$
to $gxf$.  This can be more clearly seen below:

$\xymatrix{ c_2\ar[r]^{x} & c_3 \ar[d]_{g} \\ c_1 \ar[u]^{f}  & c_4 }\\$ 

Then we note that the requirement that $Fh(x) = x'$ for morphisms in the category of elements.  Thus we are given that $gxf = x'$, and get the following diagram.

$\xymatrix{ c_2\ar[r]^{x} & c_3 \ar[d]_{g} \\ c_1 \ar[u]^{f} \ar[r]^{x'}  & c_4 }\\$ 

Note how the arrows for $f$ and $g$ face opposite directions.  This is the "twisted" part of the twisted arrow diagram.
\end{proof}

\end{document}
