\documentclass[../../main]{subfiles}
\begin{document}

\paragraph{}
\begin{exercise}\leavevmode
	\begin{enumerate}[(i)]
		\item Show that a morphism can have at most one inverse isomorphism.
			\begin{proof}
				Let \(\func{f}{x}{y}\) be an arbitrary morphism, and let
				\(\func{g}{y}{x}\) and \(\func{h}{y}{x}\) be two inverse
				isomorphisms of \(f\). That is to say, \(gf=hf=1_{x}\) and
				\(fg=fh=1_{y}\). Consider the composition \(gfh\). This is a
				valid composition, since the domain of \(g\) is equal to the
				codomain of \(f\), and the domain of \(f\) is equal to the
				codomain of \(h\). Composition is associative, so \((gf)h =
				g(fh)\).

				Evaluate each of these expressions independently:
				\begin{itemize}
					\item Evaluated as \((gf)h\), we find that \(gf=1_{x}\), so
						\((gf)h=1_{x}h=h\).
					\item Evaluated as \(g(fh)\), we find that \(fh=1_{y}\), so
						\(g(fh)=g1_{y}=g\).
				\end{itemize}

				Since both expressions are equal, we can conclude that \(h=g\).
				So any two inverse isomorphisms of \(f\) must be equal. Since
				\(f\) was arbitrary, we can generalize to conclude that any
				morphism can have at most one (distinct) inverse isomorphism.
			\end{proof}

		\item Consider a morphism \(\func{f}{x}{y}\). Show that if there exist a
			pair of isomorphisms \(\func{g,h}{y}{x}\) so that \(gf=1_{x}\) and
			\(fh=1_{y}\), then \(g=h\) and \(f\) is an isomorphism.\\
			\begin{proof}
				Let \(\func{f}{x}{y}\) be an arbitrary morphism, and let
				\(\func{g,h}{y}{x}\) be morphisms such that \(gf=1_{x}\) and
				\(fh=1_{y}\). Similarly to above, evaluate the composition
				\(gfh\) as \((gf)h=1_{x}h=h\) and as \(g(fh)=g1_{y}=g\). Due to
				associativity, we have \((gf)h=g(fh)\). So we can conclude that \(g=h\).
				Since \(fh=1_{y}\) was given, using our previous conclusion, we can
				substitute \(g\) for \(h\) to obtain \(fg=1_{y}\). Since we were
				also given \(gf=1_{x}\), we can conclude that \(f\) is an
				isomorphism.
			\end{proof}
	\end{enumerate}
\end{exercise}
\end{document}
