\documentclass[main.tex]{subfiles}
\begin{document}

\settheorem{1}{1}{11}
\begin{definition}
	A \tbf{groupoid} is a category in which every morphism is an isomorphism.
\end{definition}
\popthm

\settheorem{1}{1}{2}
\begin{exercise}
	Let \(\sC\) be a category. Show that the collection of isomorphisms in
	\(\sC\) defines a subcategory, the \tbf{maximal groupoid} inside \(\sC\).
\end{exercise}

\begin{proof}
	We want to show that, for a category \(\sC\), restricting the class
	\(\mor\sC\) to a class of morphisms that are only the isomorphisms, call it
	\(\mor\sC_{iso}\), while preserving all of the objects of \(\sC\), gives us
	a subcategory \(\sC_{iso}\) of \(\sC\).

	We start by showing that the identity morphisms in \(\sC_{iso}\) are
	isomorphisms. For an identity morphism \(\func{1_x}{x}{x}\), since the
	composition of \(1_x1_x=1_x=1_x1_x\) shows that \(1_x\) is both a right and
	left inverse of itself, then \(1_x\) is an isomorphism. Thus, the identity
	morphisms of \(\sC_{iso}\) are indeed isomorphisms.

	We now want to show that compositions of isomorphisms in \(\sC_{iso}\) yield
	isomorphisms. Take two morphisms \(\func{f}{x}{y}\) and \(\func{g}{u}{x}\)
	in \(\sC_{iso}\). Since \(f\) is an isomorphism, then there is a morphism
	\(h\in\mor\sC_{iso}\) with \(\func{h}{y}{x}\), such that \(fh=1_y\) and
	\(hf=1_x\). Likewise, since \(g\) is an isomorphism, then there is a
	morphism \(j\in\mor\sC_{iso}\) with \(\func{j}{x}{u}\), such that \(gj=1_x\)
	and \(jg=1_u\). We can take the composition \(fg\), since
	\(\dom(f)=\cod(g)\). We also have the composition \(jh\), since
	\(\dom(j)=\cod(h)\). And again, respecting domains and codomains, we have
	the composition \((fg)(jh)\), since \(\dom(fg)=\cod(jh)\). From the
	associativity of the parent category C, then
	\((fg)(jh)=f(gj)h=f(1_x)h=fh=1_y\). Thus \(jh\) is the right inverse of the
	composition \(fg\). Similarly, since \(\cod(fg)=\dom(jh)\), we have the
	composition \((jh)(fg)\), which again from the associativity of the category
	C, \((jh)(fg)=j(hf)g=j1_xg=jg=1_u\). So, \(jh\) is the left inverse of
	\(fg\), and \(fg\) is an isomorphism.

	We have shown that \(\sC_{iso}\) is a category, having all of the objects of
	\(\sC\), restricted to the isomorphisms of \(\sC\). So the groupoid \(\sC_{iso}\) is a
	subcategory of \(\sC\). Presented with any other subcategory \(D\), of \(\sC\), that is
	strictly larger than \(\sC_{iso}\), there must be a morphism in D that is
	not in \(\sC_{iso}\). Then this morphism must not be an isomorphism, and
	hence, D cannot be a groupoid. So, the category \(\sC_{iso}\) is a maximal
	groupoid that is a subcategory of C.
\end{proof}
\end{document}
