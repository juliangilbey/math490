\documentclass[../../main]{subfiles}
\begin{document}

\paragraph{}
\begin{exercise}
	What is a natural transformation between a parallel pair of functors between
	groups, regarded as one-object categories?
\end{exercise}

\begin{proof}
	For the abstract, one-object categories that are defined by groups, we can
	take any single object $C_*$ and $D_*$ as the specials objects of the
	respective groups. We then have the class of morphisms as the elements of
	the groups under their respective group operations. Let the operations here
	be group multiplication. We will start by finding what the functors are
	between these categories, and then find natural transformation in this
	context.

	Given two such categories, $\sB C$ and $\sB D$ with their respective groups $C$
	and $D$, any functors $F$ and $G$ between $\sB C$ and $\sB D$ must map the object
	$C_*$ to the object $D_*$. Since functors are functions, looking at the
	functor $F$, the role of $F$ acting on the morphisms of $\sB C$ is the same as
	a function acting on the elements of $C$ under group multiplication. So, $F:
	C_* \rightarrow D_*$, for the special elements $C_* \in \sB C$ and $D_* \in \sB D$.
	Taking any two elements $c, c'$ in $C$, $cc'$ is the composition in the
	category $\sB C$, and since the both the domain codomain of $\sB C$ are equal to $ C $, then any pair
	$cc'$ is composable. As functors respect the functoriality axioms, $F(cc') =
	F(c)F(c')$ in $D$, and $F(1_C) = 1_{F_C}$ in $D$, then the functor $F$
	behaves as a group homomorphism between the groups $C$ and $D$.

	To find a natural transformation $\Func{\al}{F}{G}$ between $F$ and $G$,
	we can let $\alpha_{C_*}\colon FC_* \rightarrow GC_*$ be a class of morphisms,
	with $f\colon C_* \rightarrow C_*$, such that $Gf\alpha_{C_*} = \alpha_{C_*}Ff$.
	In our case, we have $C_*$ as the single object of $BC$, and the morphism
	$f$ as an element $c$ in $C$. Also, we have that $FC_* = D_*$ and $GC_* =
	D_*$, so our morphism is now $\alpha_{C_*}\colon D_* \rightarrow D_*$.
	Thus, for the object $C_*$, the natural transformation gives the equality
	$\alpha_{C_*}Ff = Gf \alpha_{C_*}$.

	Since $\alpha_{C_*}$ is a morphism from the object $D_*$ to itself, this
	endomorphism (and hence automorphism) consists of the elements of the group
	$D$. Because each element has an inverse, likewise $\alpha_{C*}$ has an
	inverse, $\alpha_{C*}^{-1}$. Thus $\alpha_{C_*}Ff = Gf \alpha_{C_*}$ implies
	$\alpha_{C*}^{-1} \alpha_{C_*}Ff = \alpha_{C*}^{-1} Gf \alpha_{C_*}$,
	implies $Ff = \alpha_{C*}^{-1} Gf \alpha_{C_*}$, for all $f$. Noting that
	$Ff$ and $Gf$ are morphisms in the category $\sB D$, and hence is an element of
	the group $D$, then $Ff$ and $Gf$ are in $\Aut(D)$, and $\alpha_{C_*}$ forms
	a conjugacy class for these automorphisms.
\end{proof}
\end{document}

