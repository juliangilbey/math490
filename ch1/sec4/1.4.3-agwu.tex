\documentclass[main.tex]{subfiles}
\begin{document}

\paragraph{}
\setexercise{1}{4}{3}
\begin{exercise}
	What is a natural transformation between a parallel pair of functors between
	preorders, regarded as categories?
\end{exercise}

\begin{proof}
	Let $\sC$ and $\sD$ be preorder categories and $\pfunc{F,G}{\sC}{\sD}$ be
	parallel functors. Let us consider the properties of the natural
	transformation $\func{\al}{F}{G}$. We will show that you need only find
	morphisms $\func{f_{c}}{Fc}{Gc}$ in $\sD$ for all $c \in \ob \sC$ to produce
	a natural transformation from $F$ to $G$. This will assist in our
	characterization of natural transformations.

	Suppose we have morphisms $\func{f_{c}}{Fc}{Gc}$ in $\sD$ for $c \in \ob
	\sC$.  We can define $\func{\al}{F}{G}$ such that $\al(c) = f_{c}$. Take
	morphism $\func{g}{c}{c'}$ in $\sD$, we have that the diagram
	\[\xymatrix{ Fc \ar[d]_{\al(c)} \ar[r]^{Fg} & Fc'\ar[d]^{\al(c')}  \\
	Gc\ar[r]_{Gg} &Gc'   }\]
	commutes since in a preorder category, there is at most one
	morphism between objects. Thus, our $\al$ defines a natural
	transformation from $F$ to $G$. Thus it is sufficient to find morphisms
	$\func{f_{c}}{Fc}{Gc}$ in $D$ to define a natural transformation from
	$F$ to $G$.

	Seeing the functors $F$ and $G$ as monotone maps (i.e. order preserving
	maps) between preorders $\sC$ to $\sD$, this allows us to characterize a
	natural transformation $\al$ as a relation over $\sD$ containing only the
	pairs $\qty{Fc,Gc}$.
\end{proof}
\end{document}

