\documentclass[main.tex]{subfiles}
\begin{document}

\paragraph{}
\setexercise{1}{4}{6}
\begin{exercise}
	Given a pair of functors $ F\colon\sA\times\sB\times\sB^\op \to \sD\AND
	G:\sA\times\sC\times\sC^\op\to \sD $ a family of morphisms
	\[\al_{a,b,c}\colon F(a,b,b)\to G(a,c,c)\]
	in $ \sD $ defines the components
	of an \tbf{extranatural transformation} $ \al:F\implies G $ if for any $
	\func{f}{a}{a'},\;\func{h}{c}{c'} $ the following diagrams commute in $
	\sD$:
\begin{center}
\[
\xymatrix{ F(a,b,b)\ar[r]^{\al_{a,b,c}}\ar[d]_{F(f,1_b,1_b)}& G(a,c,c)\ar[d]^{G(f,1_c,1_c)}\\
	F(a',b,b)\ar[r]_{\al_{a',b,c}}&G(a',c,c)
}
\xymatrix{ F(a,b,b')\ar[r]^{F(1_a,1_b,g)}\ar[d]_{F(1_a,g,1_{b'})}& F(a,b,b)\ar[d]^{\al_{a,b,c}}\\
	F(a,b',b')\ar[r]_{\al_{a,b',c}}&G(a,c,c)
}
\xymatrix{ F(a,b,b)\ar[r]^{\al_{a,b,c'}}\ar[d]_{G(1_a,1_{c'},1_h)}& G(a,c',c')\ar[d]^{G(f,1_c,1_c)}\\
	G(a,c,c)\ar[r]_{\al_{a',b,c}}&G(a,c',c)
}\]
\end{center}
The left-hand square asserts the at the components $\al_{-,b,b}\colon
F(-,b,b)\implies G(-,c,c) $ define a natural transformation in $ a $ for each $
b\in\sB $ and $ c\in\sC.$ The remaining squares assert that the components $
\al_{\al_{a,-,-}}\colon F(a,-,-)\implies G(a,c,c) $ and $
\al_{a,b,-}\colon F(a,b,b)\implies G(a,-,-) $ define transformations that
are respectively extranatural in $ b $ and in $ c. $ Explain why functors $
F $ and $ G $ must have a common target category for this this definition
to make sense.
\end{exercise}

Notice that the definition of natural transformation does not actually have
anything to do with the question. This exercise is nothing more than a sanity
check. If $ F\AND G $ do not have the same target category, and if we try and
write down any of the three above diagrams, we will see that they are simply
not defined if $ F\AND G $ do not have the same target category.
\end{document}

