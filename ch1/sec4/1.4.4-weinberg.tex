\documentclass[main.tex]{subfiles}
\begin{document}

\paragraph{}
\setexercise{1}{4}{4}
\begin{exercise}
	In the notation of Example 1.4.7, prove that distinct parallel morphisms
	$f,g\colon c\rightrightarrows d$ define distinct natural transformations
	\begin{align*}
		f_*,g_*\colon \sC(-,c) \Rightarrow \sC(d,-) \\
		f^*,g^*\colon \sC(c,-) \Rightarrow \sC(d,-)
	\end{align*}
\end{exercise}

\begin{proof}
	These being natural transformations is shown in Example 1.4.7, so the
	primary concern of this problem is whether they are distinct.  First, we
	consider $f_*,g_*$ as natural transformations from $\sC(-,c) \Rightarrow
	\sC(-,d)$.  To differentiate, consider the natural transformation defined by
	$f_*$ to be $\alpha$ and $g_*$ to be $\beta$.  We need to show the
	transformations are different in at least some component.  For a natural
	transformation, we can choose arbitrary $h\colon c_1 \rightarrow c_2$ in $\sC$ to
	look at the functions for.  In this case, take $h = 1_c$.  We thus in our
	diagram have $h^*$ as our $Fh$ and $Gh$, which is precomposition by $1_c$.
	This must be the case as a functor preserves identities.  We then consider
	what this transformation does to the morphism $j =1_c \in \sC(c,c)$ to see
	what would happen to it if we put it trough the transformation.  Both
	directions take us to $fj1_c = fj = f$.  Similarly, constructing the same
	diagram except with $g$, we get $gj1_c = gj = g$.  Each takes us to a
	different morphism in $\sC(c,d)$ and thus the two transformations are
	different.  Essentially the same construction works with $f^*$ and $g^*$.
	\begin{equation*}
		\xymatrix{\sC(c,c) \ar[d]_{f_*} \ar[r]^{1_c^*} & \sC(c,c) \ar[d]^{f_*} \\
		\sC(c,d) \ar[r]^{1_c^*} & \sC(c,d) }
	\end{equation*}
\end{proof}
\end{document}
