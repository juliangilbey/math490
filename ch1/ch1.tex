\documentclass[main.tex]{subfiles}
\begin{document}

\chapter{Categories, Functors, Natural Transformations}
\section{Abstract and concrete categories}
\subfile{ch1/sec1/1.1.1-chernak.tex}
\subfile{ch1/sec1/1.1.2-james.tex}
\subfile{ch1/sec1/1.1.3-matharu.tex}

\section{Duality}
\subfile{ch1/sec2/1.2.1-pardue.tex}
\subfile{ch1/sec2/1.2.2-mcgovern.tex}
\subfile{ch1/sec2/1.2.3-agwu.tex}
\subfile{ch1/sec2/1.2.4-laczin.tex}
\subfile{ch1/sec2/1.2.5-kousoulas.tex}
\subfile{ch1/sec2/1.2.6-kousoulas.tex}
\subfile{ch1/sec2/1.2.7-kousoulas.tex}

\section{Functoriality}
\subfile{ch1/sec3/1.3.1-weinberg.tex}
\subfile{ch1/sec3/1.3.2-weinberg.tex}
\subfile{ch1/sec3/1.3.3-pardue.tex}
\subfile{ch1/sec3/1.3.4-laczin.tex}
\subfile{ch1/sec3/1.3.5-matharu.tex}
\subfile{ch1/sec3/1.3.6-agwu.tex}
%\subfile{ch1/sec3/1.3.7-james.tex}
\subfile{ch1/sec3/1.3.8-matharu.tex}
\subfile{ch1/sec3/1.3.9-mcgovern.tex}
\subfile{ch1/sec3/1.3.10-chernak.tex}

% \section{Naturality}
% \subfile{ch1/sec4/1.4.1-
\subfile{ch1/sec4/1.4.2-james.tex}
\subfile{ch1/sec4/1.4.3-agwu.tex}
\subfile{ch1/sec4/1.4.4-matharu.tex}
% \subfile{ch1/sec4/1.4.5-
% \subfile{ch1/sec4/1.4.6-

% \section{Equivalence of categories}
% \subfile{ch1/sec5/1.5.1-pardue.tex}
% \subfile{ch1/sec5/1.5.2-
% \subfile{ch1/sec5/1.5.3-chernak.tex}
% \subfile{ch1/sec5/1.5.4-mcgovern.tex}
% \subfile{ch1/sec5/1.5.5-
% \subfile{ch1/sec5/1.5.6-
% \subfile{ch1/sec5/1.5.7-
% \subfile{ch1/sec5/1.5.8-
% \subfile{ch1/sec5/1.5.9-
% \subfile{ch1/sec5/1.5.10-
% \subfile{ch1/sec5/1.5.11-

% \section{The art of the diagram chase}
% \subfile{ch1/sec6/1.6.1-
% \subfile{ch1/sec6/1.6.2-
% \subfile{ch1/sec6/1.6.3-
% \subfile{ch1/sec6/1.6.4-
% \subfile{ch1/sec6/1.6.5-
% \subfile{ch1/sec6/1.6.6-

% \section{The 2-category of categories}
% \subfile{ch1/sec7/1.7.1-
% \subfile{ch1/sec7/1.7.2-
% \subfile{ch1/sec7/1.7.3-
% \subfile{ch1/sec7/1.7.4-
% \subfile{ch1/sec7/1.7.5-
% \subfile{ch1/sec7/1.7.6-
% \subfile{ch1/sec7/1.7.7-
\end{document}
