\documentclass[main.tex]{subfiles}
\begin{document}

\paragraph{}
%\settheorem{1}{2}{3}
%\begin{lemma}
	%Recall that a group as a category has a single object $x$, and that each element of the group is a morphism in the category.  All domains and codomains are that object $x$.  There is one identity morphism $1_x$, which is the identity element in the group.  Composition is the same as multiplication in this context.
%\end{lemma}
%\popthm

\setexercise{1}{3}{2}
\begin{exercise}
	What is a functor between preorders, regarded as categories?
\end{exercise}

\begin{proof}
	Recall that a preorder regarded as a category has objects that are the
	elements of the underlying set of the preorder, and has morphisms that are
	the related pairs.  Identities are the unique morphisms $(x,x)$, which exist
	based on the reflexivity of the relation.  Note that if $(a,b)$ and $(b,c)$
	are in the relation, the composition will be $(b,c)(a,b) = (a,c)$.\\ \\ What
	do the properties of a functor between preorders $C$ (with relation $R$) and
	$D$ (with relation $S$) tell us?  First, we know that $F(1_x) = 1_{F(x)}$
	for all $x \in obC$.  This implies that the morphism $(x,x)$ must be brought
	to the morphism $(F(x),F(x))$.  This becomes redundant with the next step.
	\\ \\ We also know that $F(\dom(f)) = \dom(F(f))$ for all $f \in morC$.  If
	$f = (a,b)$, then $F(\dom(f)) = F(a)$ and thus $F(f)$ must be a pair $(F(a),
	z_1)$ for some $z_1 \in obD$.  Similarly, $F(\cod(f)) = \cod(F(f))$ implies
	that if $f = (a,b)$, then $F(\cod(f)) = F(b)$ and $F(f)$ must be a pair
	$(z_2,F(b))$.  Combining these, we get that $F(f)$ for $f = (a,b)$ must be a
	pair $(F(a),F(b))$.  This means that if $(a,b) \in R$  then $(F(a),F(b)) \in
	S$.  \\ \\ Thus, $F$ provides us a preorder homomorphism, as $F$ preserves
	related pairs.  The final property to check for a functor is composable
	pairs.  If two morphisms $f$ and $g$ are composable, then $F(fg) =
	F(f)F(g)$.  This means $F((b,c)(a,b)) = F((a,c)) = (F(b),F(c))(F(a)(F(b)) =
	(F(a),F(c))$, which was already confirmed by the previous property.  Thus,
	the functor is a preorder homomoprhism.
\end{proof}

\end{document}
