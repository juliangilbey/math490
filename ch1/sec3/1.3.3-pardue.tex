\documentclass[main.tex]{subfiles}
\begin{document}

\paragraph{}
\setexercise{1}{3}{3}
\begin{exercise}
Find an example to show that the objects and morphisms in the image of a functor
\(\func{F}{\cat{C}}{\sD}\) do not necessarily define a subcategory of
\(\cat{D}\).
\end{exercise}

At first, I was suspicious of this exercise since it seemed to me that the proof
that the image of a group (or monoid, or ring, \dots) homomorphism is a subgroup
(or submonoid, or subring, \dots) carries through without any change. Perhaps
the author meant for the exercise to be something different?

So, I asked her, and she pointed out a straightforward example that also exposed
the error in my reasoning. I will give below a simplification of the example
that she sent me. First, here is the error in my original argument.

Let \(a\) and \(b\) be morphisms in the image of \(F\) such that \(\dom a=\cod
b\) so that we can form \(ab\) in \(\sD\). Since \(a\) and \(b\) are in the
image of \(F\), there are morphisms \(f\) and \(g\) in \( bC\) such that
\(a=Ff\) and \(b=Fg\). Then \[ab=FfFg=F(fg)\] so that \(ab\) is in the image of
\(F\).

Right? Wrong!! In order to compose \(f\) and \(g\) we need that \(\dom f=\cod
g\). All we know for sure is that \(\dom Ff=\cod Fg\). If \(F\) is injective on
objects, then the argument above is valid. But perhaps \(F\) is not injective.

\begin{center}{\bf The Example.}\end{center} Now, I provide the example
requested in this exercise. Let \(\sC={\mathbb 2}\) be the ordinal category
pictured as so: \[\xymatrix{0\ar[r]^f&1.}\] Let \(g\) be an endomorphism of some
object \(x\) in some category \(\sD\) such that \(gg\) is equal to neither
\(1_x\) nor \(g\). For example, we could take \(x\) to be the unique object in
\(B\NN\) and \(g=1\), so that \(gg=g+g=2\ne 1,0\).

Then we have a functor \(F:{\mathbb 2}\rightarrow\sD\) given by \(F0=F1=x\),
\(F1_0=F1_1=1_x\) and \(Ff=g\). There are only four possible compositions of the
three morphisms in \({\mathbb 2}\), \(1_01_0\), \(f1_0\), \(1_1f\) and
\(1_11_1\) and it is easy to see that \(F\) preserves all four of these
compositions. Thus, \(F\) is a functor.

However, the image of \(F\) has only the two morphisms \(1_x\) and \(g\). Since
\(gg\) is not in the image, the image of \(F\) is not a subcategory of \(\sD\).

\end{document}
