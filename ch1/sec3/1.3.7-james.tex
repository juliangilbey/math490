\documentclass{article}
\usepackage{mathtools,array,tikz,amsfonts,amsthm,amssymb,bm,mathrsfs,braket}
\usepackage{xypic}
\newcommand{\CC}{\mathbb{C}}
\newcommand{\NN}{\mathbb{N}}
\newcommand{\RR}{\mathbb{R}}
\newcommand{\ZZ}{\mathbb{Z}}
\newcommand{\cF}{\mathcal{F}}
\newcommand{\cP}{\mathcal{P}}
\newcommand{\op}{{op}}



\DeclareMathOperator{\Hom}{Hom}
\DeclareMathOperator{\id}{id}
\DeclareMathOperator{\End}{End}
\DeclareMathOperator{\Sym}{Sym}
\DeclareMathOperator{\Aut}{Aut}


\newtheorem{definition}{Definition}
\newtheorem{theorem}{Theorem}
\newtheorem{lemma}{Exercise}

\def\addpunct#1{%
	\relax\ifhmode
	\ifnum\spacefactor>1000 \else#1\fi
\fi}
\renewenvironment{proof}[1][\proofname]{\par
  \pushQED{\qed}%
  \normalfont \topsep4pt plus 2pt\relax
  \list{}{\leftmargin=1em
          \rightmargin=1em%\leftmargin
          \settowidth{\itemindent}{\itshape#1}%
          \labelwidth=\itemindent
          \parsep=9pt plus3pt minus2pt
		  \listparindent=\parindent 
  }
  \item[\hskip\labelsep
        \itshape
    #1\addpunct{.}]\ignorespaces
}{%
  \popQED\endlist
}



	\begin{document}
	
\pagenumbering{gobble}

	\begin{center}
	
{\bf\Large Math 490: Category Theory}\\
\medskip
{\bf Re-Draft Homework 2}\\ 
\medskip
{\bf Author: Bruce James}\\ 
\medskip
{\bf 10/24/2018}\\ 

	\end{center}

\medskip


	\begin{lemma}[1.3.vii]
Define functors to construct the slice categories $c/C$ and $C/c$ as special cases of comma categories. What are the projection functors?	


	\end{lemma}
	
	

	\begin{proof}
	
We want to choose functors in a comma category, so that the comma category behaves like a slice category. Since the slice category treats morphisms like objects within a single category, and comma categories are defined generally with three categories in mind, we have some room to reduce the structure of the comma category as we construct a slice category.

Using the notation in the text's definition of the comma category, let F$\downarrow$G be our comma category. Also, let $D = C$ and let the functor $F$ be the identity functor, $\bold 1_C$. With this choice of functor, the square in the definition of F$\downarrow$G still commutes.


At this point, the objects of the comma category are $(d \in C, e\in E, f: d \rightarrow Ge)$. The morphisms are $(h: d\rightarrow d', k: e\rightarrow e')$, such that $(d,e,f) \rightarrow (d',e',f')$. 

The above choice of the identity functor collapses the amount of data represented. Yet there remain extra data at this point in the construction to qualify as a slice category. We desire to keep one of the morphisms $h$ and $k$, while reducing the object-triple $(d \in C, e\in E, f: d \rightarrow Ge$) to a suitable object-morphism pair, that allows us to fix an element in C, and take morphisms as objects.

To this end, let E be the ordinal category, $\bold 1$, with one object represented as $\emptyset$, and only the identity morphism. The objects in the comma category now are the triples $(d \in C, \emptyset, f: d \rightarrow c \in C)$. With the functor $G$ acting on the one object of $\bold 1$, then $G$ sends $\emptyset$ to one object in $C$. Since, in $E = \bold 1$, $k = id_{\bold 1}$, then $k$ can only send $\emptyset$ to $\emptyset$. So the objects in the comma construction are rendered as pairs, with the relevant data $c$ and $f$, represented as $(d \in C, f: d \rightarrow c)$.  

Letting the morphism $h$ take $d$ to $d'$, while the morphism $k = id_{\bold 1}$ sends $\emptyset$ to $\emptyset$, we have a codomain of the comma morphism $(h,k)$ represented as $(d' \in C, f': d' \rightarrow c)$. This is a class of morphisms of C taken as objects, and the comma category is reduced to the $C/c$ slice category.

\end{proof}	
	
	\end{document}