\documentclass[main.tex]{subfiles}
\begin{document}
\setexercise{1}{3}{8}
\begin{exercise}
For any group $G$, we may define other groups: 
\begin{itemize}
    \item the \textbf{center} $Z(G) = \{h \in G | hg = gh \forall g\in G\}$
    \item the \textbf{commutator subgroup} $C(G)$, the subgroup generated by the element $ghg^{-1}h^{-1}$ for any $g,h \in G$, and
    \item the \textbf{automorphism group} Aut(G), the group of isomorphisms $\phi: G \rightarrow G$ in $\Group$.
\end{itemize}
Trivially, all three constructions define a functor from the discrete category of groups (with only indentity morphsims) to \Group. Are these constructions functorial in 
\begin{itemize}
    \item the isomorphisms of groups? That is, do they extend to functors $\Group_{iso} \rightarrow \Group$?
    \item the epimorphisms of groups? That is, do they extend to functors $\Group_{epi} \rightarrow \Group$?
    \item the homomorphisms of groups? That is, do they extend to functors $\Group \rightarrow \Group$?
\end{itemize}
\end{exercise}
\begin{proof}


First, consider the functor $F_Z: \Group_{id}\rightarrow \Group$, where $G \rightarrow Z(G)$. We will now show that there exists a similar functor $F_Z: \Group_{epi} \rightarrow \Group$. We define $F_Z$ as the following: $F_Z(G) = Z(G)$ and if $f: G \rightarrow H$, then $F_{Z} f = f|_{Z(G)}$. We must show that this functor satisfies the properties of a functor.
\begin{enumerate}
    \item Because each group has a unique center and each morphism $f$ with $dom(f) = G$ has a unique restriction to $Z(G)$, this functor is well defined and satisfies the first two properties (0 and 1).
    \item We see that $F_{Z} 1_{G} = 1_{G}|_{Z(G)} = 1_{Z(G)} = 1_{F_{Z}G}$, so the functor preserves identities. 
    \item We easily see that by definition of function restriction, if $f: G \rightarrow H$, then $dom(F_{Z}f) = F_{Z}dom(f)$. We choose $cod(F_{Z}f) = Z(H) = F_Z(cod(f))$.  To see that our morphisms are still well-defined from when the domain and codomain are restricted by this functor, we show that if $g \in Z(G)$, then $f(g) \in Z(H)$. To do this, consider, for any $k \in H$, that since $f$ is an epimorphism and therefore surjective, that $k = f(h)$ for some $h \in G$. So \[f(g)k = f(g)f(h) = f(gh)\]Since $g \in Z(G)$, \[f(gh) = f(hg) = f(h)f(g) = kf(g)\] So $f(g) \in Z(H)$ and therefore we have a well-defined morphism from $F_ZG$ to $F_ZH$. 
    \item If $f:G \rightarrow H$ and $G: H \rightarrow K$, we see that $F_Z(gf) = gf|_{Z(G)}$. By the property we proved in the previous part \[gf|_{Z(G)} = g|_{Z(H)}f|_{Z(G)} = F_ZgF_Zf.\] So this functor also preserves morphism composition. 
\end{enumerate}

We have seen that $F_Z$ satisfies all properties of a functor. We also note that $F_Z$ will be a functor from $\Group_{iso} \rightarrow \Group$. 

To show that there is no such functor between $Group$ and $Group$, consider the 
composition of the homorphism $sgn: S_n \rightarrow \{ \pm 1 \} $ and $\iota: 
\{1, (1 \,\, 2) \} \rightarrow S_4$. We say $ g(x) = sgn(\iota(x))$. We see 
that $g$ is an isomorphism, and so $F_Zg$ should also be an isomorphism. 
However, this is not possible under any function of morphism, as $S_4$ has a 
trivial center and so any morphism from $Z(\{1, (1\,\,2)\} \rightarrow Z(S_4) 
\rightarrow Z(\{\pm 1\})$ must be trivial. So $F_Z$ cannot be a functor from 
$\Group \rightarrow \Group$.  

Now consider the functor $F_C:  \Group_{id} \rightarrow \Group$. We will show there exists a similarly constructed functor from $Group \rightarrow Group$ defined as the following: for $G \in ob(Group)$, $F_CG = C(G)$, where C(G) is the commutator subgroup of $G$. If $f : G \rightarrow H$, $F_Cf = f|_{C(G)}$, We will show that this satisfies all the properties of a functor.
\begin{enumerate}
    \item Because each group has a unique commutator subgroup and each morphism $f$ with $dom(f) = G$ has a unique restriction to $C(G)$, this functor is well defined and satisfies the first two properties (0 and 1).
    \item We see that $F_{C} 1_{G} = 1_{G}|_{C(G)} = 1_{C(G)} = 1_{F_{C}G}$, so the functor preserves identities. 
    \item We easily see that by definition of function restriction, if $f: G \rightarrow H$, then $dom(F_{Z}f) = F_{Z}dom(f)$. We choose $cod(F_{C}f) = C(H) = F_C(cod(f))$.To see that our morphisms are well defined when we restrict the domain and codomain, we show that if $g \in C(G)$, then $f(g) \in C(H)$. If $g \in C(G)$, $g = \Pi_{i = 1}^{n} a_i$, where each $a_i = hkh^{-1}k^{-1}$ for some $h,k \in G$. So $$f(g) = f(\Pi_{i = 1}^{n} h_ik_ih_i^{-1}k_i^{-1}) = \Pi_{i = 1}^{n} f(h_ik_ih_i^{-1}k_i^{-1}) =$$ $$  \Pi_{i = 1}^{n} f(h_i)f(k_i)f(h_i^{-1})f(k_i^{-1}) = \Pi_{i = 1}^{n} f(h_i)f(k_i)f(h_i)^{-1}f(k_i)^{-1}$$ But since $f(k_i), f(h_i) \in H$, this is an element of $C(H)$. So if $g \in C(G)$, $f(G) \in C(H)$ and therefore we have well-defined morphisms from our restricted domain to our restricted co-domain. 
    \item If $f:G \rightarrow H$ and $G: H \rightarrow K$, we see that $F_C(gf) = gf|_{C(G)}$. By the property we proved in the previous part \[gf|_{C(G)} = g|_{C(H)}f|_{C(G)} = F_CgF_Cf.\] So this functor also preserves morphism composition.
\end{enumerate}

So we have shown that this is a functor for $\Group \rightarrow \Group$, and we note that this implies that it is also a functor for $\Group_{iso} \rightarrow \Group$ and $\Group _{epi} \rightarrow \Group$. 

Next, we show that there is a functor $F_A: \Group_{iso} \rightarrow \Group$, defined as follows: If $G$ is a group, then $F_AG = Aut(G)$ and if $\phi$ is a morphism between two groups $G$ and $H$, then $(F_A\phi)(f) = \phi f \phi^{-1}$. We now show that this definition satisfies the properties of a functor. 
\begin{enumerate}
    \item Each group has uniquely defined automorphism group, and each morphism $\phi$ conjugates elements of $Aut(G)$in a unique way, so the functor is well defined. 
    \item $F_A(1_G)(f) = 1f1 = 1_{Aut(G)}(f)$, so the functor preserves identities. 
    \item $F_A(dom \phi) = F_A(G) = Aut(G) = dom F_A(\phi)$.
    \item By definiton of $F_A$, $cod(F_A\phi) = Aut(H) = F_A(cod(\phi))$.We see that $\phi f \phi^{-1} \in Aut(H)$ for any $f \in Aut(G)$ because it is a compostition of ismorphisms and therefore also an isomorphism. So when we restrict the domain and codomain of our morphisms using this functor, then they are still well-defined. 
    \item For two composable morphisms $\phi$ and $\tau$, \[F_A(\phi\tau)(f) = \phi\tau f (\phi\tau)^{-1} = \phi \tau f \tau^{-1} \phi^{-1} = F_A(\phi)(\tau f \tau^{-1}) = F_A(\phi)F_A(\tau)(f),\] so this functor preserves composition.
    
\end{enumerate}
So $F_A$ satisfies all the properties of a functor, and there exists a functor from $\Group_{iso} \rightarrow \Group$ of the desired form. 

It is unclear whether or not there is a functor from  $\Group_{epi}$to $\Group$ and from $\Group$ to $\Group$ defined in the manner, but I believe that this is not the case, although I am having trouble finding a counterexample. 
\end{proof}
\end{document}
