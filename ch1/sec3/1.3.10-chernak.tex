\documentclass[main.tex]{subfiles}
\begin{document}

% \begin{flushleft}
% 	{\small Note: In this exercise, I use the notation `\([x]\)' to denote the
% 	conjugacy class of an element \(x\) and `\(\hat{s}\)' to  denote the set of the
% conjugacy classes of the elements of a group \(s\). Furthermore, I will use F
% instead of Conj, to increase readability.}
% \end{flushleft}

\paragraph{}
\begin{exercise}
	Show that the construction of the set of conjugacy classes of elements of a
	group is functorial, defining a functor
	\(\mathrm{Conj}\colon\mathsf{Group\to Set}\). Conclude that any pair of
	groups whose sets of conjugacy classes of elements have differing
	cardinalities cannot be isomorphic.
\end{exercise}

\begin{proof}
	Let the functor \(\Conj\colon\mathsf{Group\to Set}\) represent the construction of
	the set of conjugacy classes of elements of a group, defined as follows:
	\begin{itemize}
		\item For any group \(s\), \(\Conj s = \hat{s}\).

		\item For any groups \(s\) and \(t\) and any group homomorphism
			\(f\colon s\to t\), define the morphism \(\Conj f\colon \hat{s}\to \hat{t}\)
			such that for each \([x]\in \hat{s}\), \(\Conj f([x]) = [f(x)]\).%
			\footnote{
			% \textit{\footnotesize (The proof that every \(\Conj f\) is well-defined
			% is non-trivial; see Subproof 1.3.x.0, below.)}
				In order for each \(\Conj f\) to be well-defined,
				it must send each \([x]\in \hat{s}\) to a single \([f(x)]\in \hat{t}\). But there
				may be more than one element in \([x]\), and since the definition does not
				mention which \(x\) should be used as a 'representative,' it might seem that
				there could be cases in which there were multiple possible \([f(x)]\) (and
				therefore, multiple possible \(\Conj f([x])\)) for a single \([x]\). So in order
				for \(\Conj f\) to be well-defined, \([f(x)]\) must be the same for every
				possible choice of \(x\in [x]\). In other words, we need to show that
				\([f(a)] = [f(b)]\) for any \(a, b\) in the same conjugate class.

				So, let \(a, b\) be arbitrary members of the same conjugate class. Recall
				that this means that there is some \(n\in s\) such that \(b = nan^{-1}\).
				Furthermore, recall that group homomorphisms (like \(f\)) preserve inverses.
				With this in mind, \[f(b) = f(nan^{-1}) = f(n)f(a)f(n^{-1}) =
				f(n)f(a)f(n)^{-1}\]

				\(n\in s\), so \(f(n)\in t\). This means that there is some \(m\in t\) such
				that \(mf(a)m^{-1} = f(b)\). So \(f(a)\) and \(f(b)\) are conjugates, (and
				therefore \([f(a)] = [f(b)]\) by definition,) which makes
				\(\Conj f\)
			well-defined.}
		\end{itemize}

		First we will prove that \(\Conj \) is functorial by showing that it fulfills
		both functoriality axioms:
		\begin{itemize}
			\item Let \(f\) and \(g\) be group homomorphisms such that \(gf\) is a
				valid composition, and let \([x]\in(\textrm{dom}(f))^*\) be
				arbitrary. \[\Conj g\Conj f([x])=\Conj
				g([f(x)])=[g(f(x))]=[gf(x)]=\Conj (gf([x])).\]
				Since \([x]\), f, and g were arbitrary, \(\Conj g\Conj f = \Conj
				(gf)\). So \(\Conj \)
				fulfills the first functoriality axiom.

			\item For an arbitrary group \(s\) and element \(x\in s\),
				\[\Conj 1_s([x])=[1_s(x)]=[x]=1_{\hat{s}}([x])=1_{\Conj s}([x]).\] Since s and x
				were arbitrary, \(\Conj 1_{s}=1_{\Conj s}\). So \(\Conj \) fulfills the second
				functoriality axiom.
		\end{itemize}
		Conj fulfills both axioms, so we can conclude that it is indeed
		functorial.

		Let \(s\) and \(t\) be two isomorphic groups, and let \(f\colon s\to t\) be
		an isomorphism. Functors preserve isomorphisms, (as per the `first lemma in
		category theory') so \(\Conj f\colon \hat{s}\to \hat{t}\) must also be an isomorphism.
		This makes \(\hat{s}\) and \(\hat{t}\) isomorphic, which in turn means they must
		have the same cardinality. So we can conclude the contrapositive: that any
		pair of groups whose sets of conjugacy classes have \textit{different}
		cardinalities \textit{cannot} be isomorphic.

	\end{proof}
	\end{document}
