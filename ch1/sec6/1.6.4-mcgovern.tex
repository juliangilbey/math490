\documentclass[main.tex]{subfiles}
\begin{document}

\paragraph{}
\begin{exercise}
	Find a example to show that a faithful functor need not preserve
	epimorphisms. Argue by duality, or by another counterexample, that a
	faithful functor need not preserve monomorphisms.
\end{exercise}
\begin{proof}
	Consider the category $\Ring$ and the unique morphism $\phi\colon \ZZ \rightarrow
	\QQ$. Exercise 1.2.v shows that $\phi$ is an epimorphism, however, it is
	easy to see that $\phi$ is not surjective. Now, consider a functor $F\colon \Ring
	\rightarrow \Group$ that takes a ring $R$ to it's additive group and a
	morphism $f\colon R \rightarrow S$ to the corresponding group homomorphism on the
	additive group. We see that this functor is faithful, because for fixed
	rings $R$ and $S$ and morphisms $f,g \colon R \rightarrow S$, $Ff = Fg$ implies
	that $Ff(x) = Fg(x)$ for all $x \in G_R$, the additive group of $R$, which
	has the same elements as $R$. But since by our definition $Ff(x) = f(x)$ and
	$Fg(x) = g(x)$, this implies that $f(x) = g(x)$ for all $x \in R$ and so $f =
	g$. Therefore, for all $x,y \in \Ring$, there is a injection from
	$\Ring(x,y) \rightarrow \Group (Fx,Fy)$ and therefore $F$ is faithful.  Now,
	note that in $\Group$, epimorphisms correspond exactly to surjective
	homomorphisms. But it is clear that $F\phi$ is not surjective, as its
	behavior on the elements of $\ZZ$ and $\QQ$ is identical to that of $\phi$.
	So $F\phi$ is not an epimorphism. Therefore, $F$ does not preserve
	epimorphisms.

	Now, we consider $F\colon \Ring^{\op} \rightarrow \Group^{\op}$, where $F$ acts on
	objects and morphisms as before. By 1.3.v, there is no difference between a
	functor from $\Ring$ to $\Group$ and a functor from $\Ring^{\op}$ to
	$\Group^{\op}$, so $F$ is still faithful is this setting. Now, we note that
	the epimorphisms in $\Ring$ and $\Group$ are precisely the monomorphisms in
	$\Ring^{\op}$ and $\Group^{\op}$. So $\phi$ is a monomorphism in $\Ring^{\op}$,
	but not in $\Group^{\op}$. Therefore, $F$ does not preserve monomorphisms. So
	a faithful functor need not preserve monomorphisms.
\end{proof}
\end{document}
