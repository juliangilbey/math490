\documentclass[main.tex]{subfiles}
\begin{document}

\paragraph{}
\setexercise{1}{6}{6}
\begin{exercise}
	A {\bf coalgebra} for an endofunctor $\func{T}{\sC}{\sC}$ is an object $C
	\in \sC$ equipped with a map $\func{\gm}{C}{TC}$. A morphism
	$\func{f}{\qty{C,\gm}}{\qty{C',\gm'}}$ of coalgebras is a map
	$\func{f}{C}{C'}$ so that the following square commutes.
	\[\xymatrix{ &C \ar[d]_{\gm} \ar[r]^{f} & C'\ar[d]^{\gm'}  \\  &
	TC\ar[r]_{Tf} &TC'}\]
	Prove that if $\qty{C,\gm}$ is a {\bf terminal coalgebra}, that is
	a terminal object in the category of coalgebras, then the map
	$\func{\gm}{C}{TC}$ is an isomorphism.
\end{exercise}
\begin{proof}
	Suppose that $\func{T}{\sC}{\sC}$ is an endofunctor and $\qty{C,\gm}$ a
	terminal coalgebra. Let $\func{\chi}{TC}{C}$ be the unique morphism so that
	the following diagram commutes.
	\[\xymatrix{ TC \ar[d]_{T\gm} \ar[r]^{\chi} & C\ar[d]^{\gm} \\
	TTC\ar[r]_{T\chi} &TC}\]
	Then we
	have that: $$ \gm  \chi = T\qty{\chi \gm}    $$ Now we will
	to show that the diagram $$\xymatrix{ &C \ar[d]_{\gm} \ar[r]^{\chi
	\gm} & C\ar[d]^{\gm}  \\  & TC\ar[r]_{T\qty{\chi  \gm}} &TC   } $$
	commutes. Since $ \gm  \chi = T\qty{\chi  \gm}    $, then
	composing $\gm$ on the right will give $$ \gm  \chi  \gm =
	T\qty{\chi  \gm}  \gm,   $$ which shows that the diagram above
	commutes. This means that the morphism $\chi  \gm$ uniquely allows
	the diagram above to commute. The identity morphism $1_C$ gives an
	endomorphism of the terminal coalgebra $\qty{C,\gm}$ since $\gm  \
	1_{C} = T1_{C}  \gm = 1_{TC}  \gm$ by the properties of
	identity morphisms, thus $\chi  \gm = 1_{C}$. This will give us
	$$\gm  \chi = T\qty{\chi  \gm}  = T1_{C} = 1_{TC}.$$ Thus
	$\chi$ and $\gm$ are inverses, therefore $\gm$ is an isomorphism.
\end{proof}

An example of a co-algebra is one defined by the endofunctor
$\func{P_{\fin}}{\Set}{\Set}$ where $P_{\fin}$ is the functor mapping a set $X$ to
the set of finite subsets of $X$, and maps a morphism $\func{f}{X}{Y}$ to
$\func{P_{\fin}f}{P_{\fin}X}{P_{\fin}Y}$ where for finite subset $S$ of $X$,
$P_{\fin}f(S) = f(S)$.
\end{document}
