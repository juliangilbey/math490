\documentclass[main.tex]{subfiles}
\begin{document}

\paragraph{}
\begin{exercise}
	Show that any map from a terminal object in a category to an initial one is
	an isomorphism.  An object that is both initial and terminal is called a
	zero object.
\end{exercise}

\begin{proof}
	Let $f\colon t \rightarrow i$, where $t$ is terminal and $i$ is initial. As
	$i$ is initial, there exists exactly one morphism $g\colon i \rightarrow t$.
	We must show this $g$ is the inverse of $f$ such that $fg = 1_i$ and $gf =
	1_t$. We know that $f$ and $g$ are composable, and we know $fg\colon i
	\rightarrow i$ and $gf\colon t \rightarrow t$ based on the composition law.
	As $i$ is initial, there exists exactly one morphism $i \rightarrow c$ for
	any object $c$, and thus there exists only one morphism $i \rightarrow i$,
	the identity morphism $1_i$. Thus, $fg$ must be $1_i$. Similarly, as $t$
	is terminal there exists exactly one morphism $c \rightarrow t$, and thus
	there exists only one morphism $t \rightarrow t$, the identity morphism
	$1_t$. Hence $gf$ must be $1_t$, and $f$ is an isomorphism.
\end{proof}
\end{document}
