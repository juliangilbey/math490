\documentclass[main.tex]{subfiles}
\begin{document}
%
%\settheorem{1}{2}{3}
%\begin{lemma}
	%Recall that a group as a category has a single object $x$, and that each element of the group is a morphism in the category.  All domains and codomains are that object $x$.  There is one identity morphism $1_x$, which is the identity element in the group.  Composition is the same as multiplication in this context.
%\end{lemma}
%\popthm

\paragraph{}
\setexercise{1}{6}{1}
\begin{exercise}
	Show that any map from a terminal object in a category to an initial one is
	an isomorphism.  An object that is both initial and terminal is called a
	zero object.
\end{exercise}

\begin{proof}
	Let $f\colon t \rightarrow i$, where $t$ is terminal and $i$ is initial.  As $i$
	is initial, there exists exactly one morphism $g\colon i \rightarrow t$.  We must
	show this $g$ is the inverse of $f$ such that $fg = 1_i$ and $gf = 1_t$.  We
	know that $f$ and $g$ are composable, and we know $fg\colon i \rightarrow i$ and
	$gf\colon t \rightarrow t$ based on the composition law.    As $i$ is initial,
	there exists exactly one morphism $i \rightarrow c$ for any object $c$, and
	thus there exists only one morphism $i \rightarrow i$, the identity morphism
	$1_i$.  Thus, $fg$ must be $1_i$.  Similarly, as $t$ is terminal there
	exists exactly one morphism $c \rightarrow t$, and thus there exists only
	one morphism $t \rightarrow t$, the identity morphism $1_t$.  Thus $gf$ must
	be $1_t$.  Thus, $f$ is an isomorphism.
\end{proof}
\end{document}
