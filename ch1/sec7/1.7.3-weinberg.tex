\documentclass[../../main]{subfiles}
\begin{document}

\paragraph{}
\begin{exercise}
	Redefine the horizontal composition of natural transformations introduced
	in Lemma 1.7.4 using vertical composition and whiskering.
\end{exercise}

\begin{proof}

Consider categories C, D, and E, with functors $F,G\colon \sC \rightarrow \sD$ and
$H,K\colon \sD \rightarrow \sE$ with natural transformations $\alpha\colon F \Rightarrow
G$ and $\beta\colon H \Rightarrow K$.  This is the same as in the construction for
horizontal composition given in Lemma 1.7.4.  Now consider a whiskering such
that the first functor is the identity functor, or in other words a natural
transformation $H\alpha\colon HF \Rightarrow HG$ defined by $(H\alpha)_c =
H\alpha_c$.  Then consider a second whiskering such that the last functor is an
identity functor, or in other words a natural transformation $\beta G\colon HG
\Rightarrow KG$ defined by $(\beta G)_c = \beta_{Gc}$.  Now we can vertically
compose these two natural transformations to get $\beta G \cdot H \alpha\colon HF
\Rightarrow KG$.  This is definitionally the same as the horizontal composition
natural transformation:
\begin{equation*}
(\beta*\alpha)_c = \beta_{Gc}H\alpha_c = (\beta G\cdot H\alpha)_c.
\end{equation*}

\end{proof}


\end{document}
