\documentclass[main.tex]{subfiles}
\begin{document}

\paragraph{}
\begin{exercise}
	Prove Lemma 1.7.7.
\end{exercise}

\tikzcdset{
	arrow style=math font
}

\settheorem{1}{7}{7}
\begin{lemma}[middle four interchange]
	Given functors and natural transformation
	\[\begin{tikzcd}
			\sC \ar[r, "G"{name=G, description}]
			\ar[r, out=55, in=125, looseness=1.6, "F"{name=F}]
			\ar[r, out=-55, in=-125, looseness=1.6, "H"{name=H, below}] &
			\sD \ar[r, "K"{name=K, description, scale=1.3}]
			\ar[r, out=55, in=125, looseness=1.6, "J"{name=J, scale=1.3}]
			\ar[r, out=-55, in=-125, looseness=1.6, "L"{name=L, below}] & \sE
			\ar[phantom, from=F, to=G, "\DTo\al"]
			\ar[phantom, from=G, to=H, "\DTo\be"]
			\ar[phantom, from=J, to=K, "\DTo\gm"]
			\ar[phantom, from=K, to=L, "\DTo\de"]
	\end{tikzcd}\]
	the natural transformation \(JF\To LH\) defined by first composing
	vertically and then composing horizontally equals the natural transformation
	defined by first composing horizontally and then composing vertically:
	\[\begin{tikzcd}
			\sC \ar[r, out=55, in=125, looseness=1.6, "F"{name=J}]
			\ar[r, out=-55, in=-125, looseness=1.6, "H"{name=L, below}]
			\ar[r, phantom, "\DTo\be\cdot\al"] &
			\sD \ar[r, out=55, in=125, looseness=1.6, "J"{name=J}]
			\ar[r, out=-55, in=-125, looseness=1.6, "L"{name=L, below}]
			\ar[r, phantom, "\DTo\de\cdot\gm"] &
			\sE \ar[r, phantom, "="{description, scale=1.3}] &
			\sC \ar[rr, "KG"{name=KG, description}]
			\ar[rr, out=45, in=135, looseness=1.2, "JF"{name=JF}]
			\ar[rr, out=-45, in=-135, looseness=1.2, "LH"{name=LH, below}] &&
			\sE
			\ar[phantom, from=JF, to=KG, "\DTo\gm*\al"{scale=0.9}]
			\ar[phantom, from=KG, to=LH, "\DTo\de*\be"{scale=0.9}]
	\end{tikzcd}\]
\end{lemma}
\popthm

Before proceeding with the proof, let us elaborate on exactly how horizontal and
vertical composition work. Vertical composition is the simpler of the two since
it is just composition of morphisms in the target category of our functors.
Precisely, given parallel functors and accompanying natural transformations
we can compose the component maps \(\al_c\) and \(\be_c\) for any object \(c\)
in \(\sC\). That this defines defines a new natural transformation
\(\Func{\be\cdot\al}{F}{H}\) follows immediately.
\[\begin{tikzcd}
		\sC \ar[r, "G"{name=G, description}]
		\ar[r, out=55, in=125, looseness=1.6, "F"{name=F}]
		\ar[r, out=-55, in=-125, looseness=1.6, "H"{name=H, below}] & \sD
		\ar[phantom, from=F, to=G, "\DTo\al"]
		\ar[phantom, from=G, to=H, "\DTo\be"]
		\end{tikzcd}\quad\leadsto\quad\begin{tikzcd}
		Fc \ar[r, "\al_c"] \ar[d, "Ff"] \ar[dr, dashrightarrow]  &
		Gc \ar[r, "\be_c"] \ar[d, "Gf"] \ar[dr, dashrightarrow]  &
		Hc \ar[d, "Hf"] \\
		Fd \ar[r, "\al_d"] &
		Gd \ar[r, "\be_d"] &
		Hd
\end{tikzcd}\]

Horizontal composition is more involved. Note that one way to think of natural
transformations is as mapping objects to morphisms and morphisms to commutative
squares in the target category. Given \(\Func{\al}{F}{G}\)
where \(\pfunc{F,G}{\sC}{\sD}\) and \(\func{f}{c}{d}\), the object \(c\) is
taken to a morphism \(\al_c\) and the arrow \(f\) is taking to a square of
morphisms connecting \(Fc\) to \(Gd\). Because this square commutes there is a
unique composite arrow from \(Fc\) to \(Gd\). Horizontal composition is then
applying the natural transformation \(\gm\) to \(Fc\), \(Gc\), and the map
\(\al_c\) between them. Given \(\Func{\gm}{J}{K}\) and \(\pfunc{J,K}{\sD}{\sE}\)
By naturality, there is again a unique composite arrow denoted \((\gm*\al)_c\)
which will form the \(c\) component of the new natural transformation.
\[\begin{tikzcd}
		c \ar[d, phantom, "\downmapsto"] &
		c \ar[r, "f", ""{below, phantom, name=f}] &
		d \\
		Fc \ar[d, "\al_c"{left}] &
		Fc \ar[r, "Ff"{name=Ff}] \ar[dr, dashrightarrow] \ar[d, "\al_c"{left}] &
		Fd \ar[d, "\al_d"] \\
		Gc &
		Gc \ar[r, "Gf"{below}] &
		Gd
		\ar[d, phantom, from=f, to=Ff, "\downmapsto"]
		\end{tikzcd}\quad\leadsto\quad\begin{tikzcd}
		Fc \ar[r, phantom, "\mapsto"] &
		JFc \ar[r, "\gm_{Fc}"] &
		KFc \\
		Fc \ar[d, "\al_c"{left}, ""{name=f}] &
		JFc \ar[r, "\gm_{Fc}"] \ar[dr, "(\gm*\al)_c"{description}]
		\ar[d, "J\al_c"{left, name=Ff}] &
		KFc \ar[d, "J\al_c"] \\
		Gc &
		JGc \ar[r, "\gm_{Gc}"{below}] &
		KGc
		\ar[d, phantom, from=f, to=Ff, "\mapsto"]
\end{tikzcd}\]

\begin{proof}
	At a first pass this lemma is telling us that
	\((\de\cdot\gm)*(\be\cdot\al)=(\gm*\al)\cdot(\de*\be)\). For a specific
	object \(c\) in \(\sC\) this means the following diagrams are equivalent:
	\[\xymatrix{
			JFc \ar[r]^{\gm_{Fc}} \ar[d]_{J\al_c} \ar@{-->}[dr]|{(\gm*\al)_c} &
			KFc \ar[d]^{K\al_c} \\
			JGc \ar[r]_{\gm_{Gc}} &
			KGc \ar[r]^{\de_{Gc}} \ar[d]_{K\be_c} \ar@{-->}[dr]|{(\de*\be)_c} &
			LGc \ar[d]^{L\be_c} \\ &
		KHc \ar[r]_{\de_{Hc}} & LHc }
		\xymatrix{ \\ = }
		\xymatrix{ % & Jd \ar[r]^{\gm_d} & Kd \ar[r]^{\de_d} & Ld \\
			% Fc \ar[d]_{\al_c} &
			JFc \ar[r]^{\gm_{Fc}} \ar[d]^{J\al_c}
			\ar@{-->}@/_2em/[dd]_{J(\be\cdot\al)_c}
			\ar@{-->}@/^2em/[rr]^{(\de\cdot\gm)_{Fc}}
			\ar@{-->}[ddrr]|{(\de\cdot\gm)*(\be\cdot\al)} &
			KFc \ar[r]^{\de_{Fc}} &
			LFc \ar[d]_{L\al_c} \ar@{-->}@/^2em/[dd]^{L(\be\cdot\al)_c} \\
			% Gc \ar[d]_{\be_c} &
			JGc \ar[d]^{J\be_c} &&
			% KGc &
			LGc \ar[d]_{L\be_c} \\
			% Hc &
			JHc \ar[r]_{\gm_{Hc}} \ar@{-->}@/_2em/[rr]_{(\de\cdot\gm)_{c}} &
	KHc \ar[r]_{\de_{Hc}} & LHc \\}\]
	We may extend the left diagram to the following without disturbing
	commutativity, implying that the diagonal map in each case is the same.
	\begin{equation*}\xymatrix{
			JFc \ar[r]^{\gm_{Fc}} \ar[d]_{J\al_c} \ar@{-->}[dr]|{(\gm*\al)_c} &
			KFc \ar[r]^{\de_{Fc}} \ar[d]^{K\al_c} &
			LFc \ar[d]^{L\al_c} \\
			JGc \ar[r]_{\gm_{Gc}} \ar[d]_{J\be_c} &
			KGc \ar[r]^{\de_{Gc}} \ar[d]_{K\be_c} \ar@{-->}[dr]|{(\de*\be)_c} &
			LGc \ar[d]^{L\be_c} \\
			JHc \ar[r]_{\gm_{Hc}} &
	KHc \ar[r]_{\de_{Hc}} & LHc }\end{equation*}
	In particular note that the top right and bottom left quadrants themselves
	are the diagram associated with the horizontal compositions \(\de*\al\) and
	\(\gm*\be\) and for the object \(c\), which thus must commute, making the
	entire diagram commute.
\end{proof}
\end{document}
