\begin{document}

\begin{exercise}
	Prove that a bifunctor $F\colon \sC \times \sD \rightarrow \sE$ determines and is
	uniquely determined by:
	\begin{enumerate}
		\item A functor $F(c,-)\colon\sD \rightarrow \sE$ for each $c \in \sC$.
		\item A natural transformation $F(f,-)\colon F(c,-) \implies F(c',-)$
			for each $f\colon c \rightarrow c'$ in $\sC$.
	\end{enumerate}
	In other words, prove that there is a bijection between functors $\sC \times
	\sD \rightarrow \sE$ and functors $\sC \rightarrow \sE^\sD $. By symmetry of the product
	of categories, these classes of functors are also in bijection with functors
	\(\sD\to\sE^\sC\).
\end{exercise}

\begin{proof}
	From the category $\sC$, fix an object $c$ in the bifunctor $F\colon \sC \times
	\sD
	\rightarrow \sE$. To find a functor $F(c,-)\colon\sD \rightarrow \sE$, define
	$F(c,-)(d) = F(c,d)$. With $c$ fixed, and $g\colon d \rightarrow d'$, then
	$F(c,-)(g) = F(1_c, g)$. For a natural transformation $F(f,-)\colon F(c,-)
	\implies F(c',-)$, take as components of the natural transformation
	$F(f,-)_d = F(f, 1_d) \colon F(c,d) \rightarrow F(c',d)$. Checking the
	naturality, $F(1_{c'}, g) F(f, 1_d) = F(f, 1_{d'}) F(1_c, g)$, whenever
	$(1_{c'}, g)(f, 1_d) = (f, 1_d')(1_c,g)$. Thus $F(f,-)\colon F(c,-) \implies
	F(c',-)$ is a natural transformation, and conditions (i) and (ii) are
	determinied by the bifunctor $F\colon \sC \times \sD \rightarrow \sE$.

	Similarly, with the definitions that for a fixed $c$ in $\sC$, $F(c,d) =
	F(c,-)(d)$ and $F(1_c, g) = F(c,-)(g)$, along with the natural tranformation
	$F(f,-)_d = F(f, 1_d) \colon F(c,d) \rightarrow F(c',d)$, such that $F(1_{c'}, g)
	F(f, 1_d) = F(f, 1_{d'}) F(1_c, g)$, whenever $(1_{c'}, g)(f, 1_d) = (f,
	1_d')(1_c,g)$, the bifunctor $F\colon \sC \times \sD \rightarrow \sE$ is determined.
\end{proof}

\end{document}
