\documentclass[main.tex]{subfiles}
\begin{document}
\begin{exercise}
	Characterize the categories that are equivalent to
	discrete categories. A category that is connected and essentially
	discrete is called \textbf{chaotic}.
\end{exercise}
\begin{proof}
	We will show that a category $ \sC $ is equivalent to a discrete category $ \sD $ if and only $ \sC $ is preorder and a groupoid.
	Consider any category $ \sC $ that is equivalent to a discrete category
	$ \sD $. By theorem 1.5.9 we have a full, faithful, and essentially
	surjective functor $ F $ from $ \sC $ to $\sD. $ Since $ F $ is both
	full and
	faithful, we have for each $ x,y\IN\ob\sC $ the map $
	\sC(x,y)\to\sD(Fx,Fy) $ is a bijection. Because $ \sD $ is a discrete
	category we know that $ \sD(Fx,Fy) $ is empty if $ Fx\AND Fy $
	are distinct or, if they are the same, consists of only the identity
	map. Since $ \sC(x,x)\to \sD(Fx,Fx) $ is a bijection and $ \sD(Fx,Fx) $
	only contains the identity map we can conclude that $ \sC(x,x) $ also
	only consists of the identity map. Now let $\sC(x,y)$ be inhabited by $
	f, $ so $ \sD(Fx,Fy) $ is inhabited as well and we must have $ Fx=Fy.$
	Hence there must be exactly one morphism between $ x\AND y\IN C. $
	Using similar reasoning we can conclude $ \sC(y,x) $ has at most one
	element. Since both $ \sC(x,y) \AND \sC(y,x)$ are mapped to the
	identity of $ Fx $ we must have $fg=gf=1_x .$ Hence between any objects
	$ x\AND y\IN\sC $ $ \sC(x,y) $ has exactly one member or is empty. Since
	every morphism in $ \sC $ is invertible for any two objects $ x\AND
	y\IN\sC $ there is at most one element in the $ \sC(x,y),$ $ \sC $ is
	a groupoid and preorder as desired.

	Conversely consider any category $ \sC $ that is both a groupoid and a
	preorder. For insurance reasons we will work in a universe $ V, $
	where $ \sC $ is a small category.
	Now look at the skeleton category of $\sC, $ the category whose
	objects are exactly one element from each isomorphism classes (which we obtain by using the axiom of choice, and hence the reason for the insurance policy) of $ \sC,$ denoted
	as  \skull. By definition we know that $ \skull\simeq \sC.$ Because
	$ \sC $ is a groupoid, if $ \sC(x,y) $ is nonempty for any two objects
	$ x\AND y\IN\sC,$ we must have that $ x $ is isomorphic to $ y$ and
	therefore in the same isomorphism class. Also since $ \sC $ is a
	preorder there is at most one element in $ \sC(x,y).$ So all the
	morphisms of $ \sC $ are collapsed into the identity morphisms of the
	appropriate isomorphism classes. Hence the only morphisms of $ \skull $
	are the identities of each isomorphism class, so $ \skull $ is discrete
	category. Since equivalence is transitive $ \sC $ is equivalent to a
	discrete category, and this completes the proof.
\end{proof}

\end{document}
