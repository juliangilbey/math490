\documentclass[main.tex]{subfiles}
\begin{document}

\paragraph{}
\begin{center}
	\textit{{\small For legibility, I use \(\alpha\) to denote the isomorphism
	\(\cong\colon a'\to a\) (with \(\alpha^{-1}\) denoting its inverse) and
	\(\beta\) to denote the isomorphism \(\cong\colon b'\to b\) (with
	\(\beta^{-1}\) denoting its inverse.) Furthermore, I refer to the
	'first/second/third/fourth' diagrams, counting from the left.}}
\end{center}

\begin{exercise}
	Finish the following proof of Lemma 1.5.10:
\end{exercise}

	Any morphism \(f\colon a\to b\) and fixed isomorphisms \(a \cong a'\) and
	\(b \cong b'\) determine a unique morphism \(f'\colon a'\to b'\) so that any
	of the following four diagrams commute.

	\[\xymatrix{
	& a\ar[d]_{f} & a'\ar[l]_{\alpha}\ar[d]^{f'} & a\ar[d]_{f}\ar[r]^{\alpha^{-1}} & a'\ar[d]^{f'} & a\ar[d]_{f} & a'\ar[l]_{\alpha}\ar[d]^{f'} & a\ar[d]_{f}\ar[r]^{\alpha^{-1}} & a'\ar[d]^{f'}\\
	& b\ar[r]_{\beta} & b' & b\ar[r]_{\beta} & b' & b & b'\ar[l]^{\beta^{-1}} & b & b'\ar[l]^{\beta^{-1}}
	}\]

The first diagram defines \(f'\). The commutativity of the remaining diagrams
will be proven below.

\begin{proof}
 The prompt implies that the first diagram, at least, is
commutative; so we know that \(f' = \beta f \alpha \).

	At this point, it might be tempting to immediately right-compose both sides
	of that expression with \(\alpha^{-1}\), to obtain \(f'(\alpha^{-1}) = \beta
	f \alpha(\alpha^{-1})\). Then we could simplify to obtain \(f' \alpha^{-1} =
	\beta f\), and \textit{voil\'{a}!} The second diagram commutes! ......Right?

	Well, no, not quite. The logic above is missing a crucial step---it first
	assumes that we actually \textit{can} right-compose both sides of \(f' =
	\beta f \alpha \) with \(\alpha^{-1}\). It is true that since an isomorphism
	can always be composed with its inverse, the validity of the composition
	\(\beta f \alpha (\alpha^{-1})\) is trivial. But the validity of \(f'
	(\alpha^{-1})\) is decidedly non-trivial, and this must be proven before the
	logic above can be applied.
	
First, it will be useful to explicitly identify the domains and codomains of the
morphisms included in each diagram, to make further references more concise.
These can easily be inferred from the diagrams: \(f \colon a \to b; f' \colon a'
\to b'; \alpha \colon a' \to a;\) and \(\beta \colon b \to b';\) while the
inverses of each morphism go between the same objects, but with the domain and
codomain reversed. Now the remainder of the proof becomes almost trivial:

As stated above, \(\cod(\alpha^{-1}) = \dom(\alpha) = a' = \dom (f')\). So
\(\dom(f') = \cod(\alpha^{-1})\), which means \(f' \alpha^{-1}\) is a valid
composition. With that in mind, we are now able to apply the logic quoted in the
note above to show that \(f' \alpha^{-1} = \beta f\); and this is sufficient to
show that the second diagram commutes.

To show that the remaining diagrams commute, we must prove that \(\beta^{-1} f'
= f \alpha\) and that \(\beta^{-1} f' \alpha^{-1} = f\), for the third and
fourth diagrams respectively. Again, it is easy to obtain these expressions by
composing \(\beta^{-1}\) with the expressions we have already determined --
specifically, by taking the compositions \((\beta^{-1})f' = (\beta^{-1})\beta f
\alpha = f \alpha\) for the third diagram and \((\beta^{-1})f' \alpha^{-1} =
(\beta^{-1})\beta f = f\) for the fourth. The 'difficult' part is to show that
these compositions are valid.

Fortunately, this is still fairly easy: the validity of \(\beta^{-1} \beta\) is
trivial, and \(\dom(\beta^{-1}) = \cod(\beta) = b' = \cod(f')\), so \(\beta^{-1}
f'\) is valid. This means that the compositions mentioned in the previous
paragraph are valid, and therefore that \(\beta^{-1} f' = f \alpha\) and
\(\beta^{-1} f' \alpha^{-1} = f\). So the third and fourth diagrams commute.
\end{proof}
\end{document}
