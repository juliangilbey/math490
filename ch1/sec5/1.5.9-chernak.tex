\documentclass[main.tex]{subfiles}
\begin{document}

\paragraph{}
\begin{exercise}
	Show that any category equivalent to a locally small category is locally
	small.
\end{exercise}

\begin{proof}
	First, we will prove a much more general statement; that if there exists a
	faithful functor \(F \colon \textsf{C} \to \textsf{D}\), where
	\(\textsf{D}\) is a locally small category, then \(\textsf{C}\) must be
	locally small. The proof follows:

\begin{quotation}
	\(F \colon \textsf{C} \to \textsf{D}\) is faithful, so for any \(x, y \in
	\textsf{C}\), the map \(F_x,_y \colon \textsf{C}(x, y) \to \textsf{D}(Fx,
	Fy)\) is injective. And since there is an injective function \(F_x,_y\) from
	\(\textsf{C}(x, y) \to \textsf{D}(Fx, Fy),\) there must be a surjective
	function \(g \colon \textsf{D}(Fx, Fy) \to \textsf{C}(x,
	y)\)\footnotemark[1].
	
	Per the Axiom of Replacement, the image of a function whose domain is a set
	must be a set, so the image of \(g\) is a set. But we just said that \(g\)
	is surjective over \(\textsf{C}(x, y)\), so its image is simply
	\(\textsf{C}(x, y)\); which means that \(\textsf{C}(x, y)\) must be a set!
	And since this holds for all \(x, y \in \textsf{C}\), \(\textsf{C}\) must be
	locally small.
\end{quotation}

In the particular case specified by the exercise, wherein \(\textsf{C} \simeq
\textsf{D}\), there is by definition a faithful functor \(F \colon \textsf{C}
\to \textsf{D}\). So this is a special case of the general statement proven
above; meaning we can immediately conclude that \(\textsf{C}\) is locally small.

\footnotetext[1]{The principle that "For two sets A and B, if there is a
surjection from A to B, then there is an injection from B to A, and vice versa"
is called the Partition Principle. While the principle has been known to be a
consequence of the Axiom of Choice for quite awhile, it's an open question
whether or not it \textit{implies} the Axiom of Choice -- in other words,
whether it is equivalent to the axiom. Bertrand Russell claimed it did, but he
never provided a proof; and while set theorists as a whole have come incredibly
close to one since then, they've never quite gotten there.}
\end{proof}

\end{document}
