\documentclass[main.tex]{subfiles}
\begin{document}
\setexercise{1}{5}{10}
\begin{exercise}
	Consider the forgetful functor $ \func{U}{\Mod{R}}{\msf{Ab}}, $ which
	forgets the scalar multiplication structure of $ \Mod{R} $ and treats
	it like an Abelian group under addition. Is this functor full, faithful,
	or essentially surjective?
\end{exercise}
\begin{proof}
	This functor is not always essentially surjective. If we let $ R=\{0\}, $
	then the only object $ \Mod{0}$ is the zero module. Thus $ F0 $ only goes to
	the trivial group. So if $ U $ were to be fully faithful in this case, every
	Abelian group must be isomorphic to the trivial group. This is clearly not
	the case since $ \ZZ/2\ZZ $ (or any finite Abelian group really) cannot be
	isomorphic to the trivial group. However $ U $ is both full and faithful as
	$ \Mod{0}(0,0)\AND \msf{Ab}(F0,F0)=\msf{Ab}(0,0) $ both only have one
	element in them. So the only map between $ \Mod{0}(0,0)\to \msf{Ab}(0,0) $
	takes the identity map in $ \Mod{0} $ to the identity map in the trivial
	group, which is clearly bijective.

	If we let $ R=\ZZ,$ recall that every abelian group can be uniquely
	expressed as a $\ZZ$-module. In this case since $ \Mod{\ZZ} $ is exactly
	$ \msf{Ab},$ the forgetful functor $ U $ becomes the identity functor, so
	$ U $ is clearly full, faithful, and essentially surjective.

	The previous two examples were both full, however, this need not always be
	the case. If we let $ R=\ZZ_2[x]/x_2+x+1,$ and look at $ R $ as a dimension
	one vector space over itself we have the following addition and
	multiplication tables.
	\begin{center}
		\begin{tabular}{|c|c c c c| }
			\hline
			+&0&1&\(\al\)&\(\al+1\)\\
			\hline
			0&0&1&\(\al\)&\(\al+1\)\\
			1&1&0&\(\al+1\)&\(\al\)\\
			\(\al\)&\(\al\)&\(\al+1\)&0&1\\
			\(\al+1\)&\(\al+1\)&\(\al\)&1&0\\
			\hline
		\end{tabular}
		\begin{tabular}{|c|c c c c|}
			\hline
			*&0&1&\(\al\)&\(\al+1\)\\
			\hline
			0&0&0&0&0\\
			1&0&1&\(\al\)&\(\al+1\)\\
			\(\al\)&0&\(\al\)&\(\al+1\)&1\\
			\(\al+1\)&0&\(\al+1\)&1&\(\al\)\\
			\hline
		\end{tabular}
	\end{center}
	Since $ R $ is rank one free module, every endomorphism on $ R $ is defined
	by scalar multiplication. Now forget the scalar multiplication on $ R,$ and
	treat $ R $ like an Abelian group. Since $ R $ is an Abelian group and in
	particular a field of characteristic two, the Frobenius endomorphism $
	a\mapsto a^p,$ where $ p $ (in this case $ p=2 $) is the characteristic of $
	R $ (this is a field endomorphism so it preserves the additive structure as
	well even though it is defined in terms of the multiplicative operation
	which we have technically forgotten) is a member of $ \msf{Ab}(UR,UR).$
	There is no module homomorphism that corresponds to the Frobenius
	endomorphism, since the Frobenius endomorphism fixes $ 0\AND 1 $ and swaps $
	\al\AND \al+1,$ but every module homomorphism in this case can only fix one
	element at a time. (We know this because every endomorphism is
	multiplication by a scalar, so they are all completely described in the
	multiplication table.) Notice that this counterexample will work for any
	finite field with characteristic $ p. $

	Even though $U $ is not always full, $ U $ is always faithful. If we say $
	f\AND g $ are distinct morphisms in $ \Mod{R}(x,y), $ then they must
	disagree on at least one element, say, $ z $ in $x.$ Because $ Uf\AND Ug $
	are exactly the same functions in $ \msf{Ab}(Ux,Uy) $ they disagree on the
	same element $ z, $ so they are distinct in $ \msf{Ab}(Ux,Uy) $ as well.
	Hence $ U $ is always faithful.
\end{proof}
\end{document}
