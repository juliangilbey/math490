\documentclass[main.tex]{subfiles}
\begin{document}

\begin{proof} {\(\Ring\To\Group\)}
	An isomorphism of groups must preserve cardinality, among other things.
	Since there are groups of any finite order (consider the cyclic groups), to
	disprove that the functor is essential surjective it suffices to show that
	no ring can have a multiplicative group of a specific order. In particular
	we will consider five.

	First, note that in any ring we may consider the multiplicative order of
	\(-1\), the additive inverse of 1. If 1 is distinct from -1, then -1 has
	order 2,\footnote{We have
		\(0=-1\cdot 0=-1(-1+1)=-1\cdot-1+-1\cdot1=-1\cdot-1+-1\) implying that
	\(-1\cdot-1\) is the additive inverse of 1, so \(-1\cdot-1=1\).} implying
	that the multiplicative group of our ring must contain a subgroup of order
	two, and thus must have even order by Lagrange's theorem. We thus need only
	consider rings where 1 does not have a distinct additive inverse, i.e.
	rings of characteristic two.

	Suppose that we have a ring \(R\) of characteristic two. If the
	multiplicative group of \(R\) has order five, then it must be isomorphic to
	\(\ZZ/5\ZZ\) and thus have some element \(\ze\) with multiplicative order
	five, i.e. \(\ze^5-1=(\ze-1)(\ze^4+\ze^3+\ze^2+\ze+1)=0\). We ruled out that
	\(\ze\) is one by saying it has order five, so the latter factor must be
	zero. % this seems to require that we have an integral domain

	Now consider the polynomial ring \(\FF_2\adj{x}\), and the evaluation map
	\(\func{ev_\ze}{\FF_2\adj{x}}{R}\) which takes \(x\) to \(z\). The
	polynomial \(f(x)=x^4+x^3+x^2+x+1\) must be in the kernel of this map, and
	further is irreducible in \(\FF_2\adj{x}\). To see this note first that it
	is neither divisible by \(x\) nor \(x+1\). So if it were divisible it would
	be so by two degree two polynomial. However, \(x^2+x+1\) is the only
	irreducible degree two polynomial in \(\FF_2\adj{x}\), and
	\((x^2+x+1)^2=x^4+x^2+1\). Since \(f\) is irreducible and of degree four,
	the quotient \(\FF_2\adj{x}/f\) is the field \(\FF_{16}\).
	
	So we may factor the evaluation map through the quotient of \(\FF_2\adj{x}\)
	by \(\ker ev_\ze\), which must then embed \(\FF_{16}\) in \(R\) meaning that
	it's multiplicative group has far more than just five elements. Thus the
	group \(\ZZ/5\ZZ\) is completely missed by our functor which fails to be
	essentially surjective.

	\subparagraph{}
	Now let us consider whether this functor is full or faithful. First,
	consider the ring of real numbers \(\RR\) with their usual operations, this
	has no non-identity homomorphisms. However, if we consider only
	\(\RR^\times\), then there are many group homomorphisms. A typical
	homomorphism is raising an element to some power. Thus our functor cannot be
	full.

	I haven't figured out faithful yet, updates to follow.
	% Next, consider the ring \(\QQ\adj{x}\). This has as units the constant
	% polynomials, \(\QQ\)'s embedding in \(\QQ\adj{x}\). Consider the
	% endomorphism defined by \(x\mapsto x^2\) while leaving the constants alone.
	% This is a homomorphism because of the of the exponent laws:
	% \(ax^{2n}+bx^{2n}=(a+b)x^{2n}\) and \(ax^{2n}\cdot bx^{2m}=

\end{proof}
\end{document}

