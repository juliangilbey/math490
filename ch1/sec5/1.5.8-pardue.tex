\documentclass[main.tex]{subfiles}
\begin{document}

\paragraph{}
The exercise below concerns affine and projective planes as incidence
geometries. For background, see Section 2.6 of Hartshorne's \emph{Geometry:
Euclid and Beyond}. I adapted the following two definitions from that
source\footnote{I added in each case that the set of points and the set of lines
	do not intersect so as to avoid annoying set-theoretic problems in the
constructions}. The two examples that I give are standard examples with
coordinates in a field $k$, but note that the definitions make no use of
coordinates and the constructions that we will use do not either.

However, the constructions do have natural linear algebra interpretations in the
examples with coordinates. I will use some basic facts from linear algebra
concerning two and three dimensional vector spaces, such as properties of
cross-product. If you do not recall these, then you may ignore the examples and
concentrate on the rest.

\begin{mdefinition} An \emph{affine plane} is a triple of sets $\AA=(A,L,I)$ with
	$A\cap L=\emptyset$ where the elements of $A$ are called \emph{points} and
	the elements of $L$ are called \emph{lines} satisfying the following
	additional requirements. $I\subseteq A\times L$ is a relation where $sI\ell$
	is read as "$s$ lies on $\ell$". $\AA$ satisfies the following axioms.
	\begin{enumerate}
		\item For any two distinct $s,t\in A$, there is a unique $\ell\in L$
			such that $s$ and $t$ lie on $\ell$.

		\item For every $\ell\in L$ there are at least two distinct $s,t\in A$
			that lie on $\ell$.

		\item There are at least three distinct $s,t,u\in A$ such that there is
			no $\ell\in L$ such that $s$, $t$ and $u$ all lie on $\ell$.

		\item Two lines $\ell$ and $m$ are said to be \emph{parallel} if either
			$\ell =m$ or there is no $s\in A$ that lies on both $\ell$ and $m$.
			Write $\ell\parallel m$ if $\ell$ and $m$ are parallel. For every
			$\ell\in L$ and $s\in A$, there is a unique $m\in L$ such that $s\in
			m$ and $\ell\parallel m$.
	\end{enumerate}
\end{mdefinition}

\begin{mdefinition} A \emph{projective plane} is a triple of sets $\PP=(P,L,I)$
	with $P\cap L=\emptyset$ where the elements of $P$ are called \emph{points}
	and the elements of $L$ are called \emph{lines} satisfying the following
	additional requirements. $I\subseteq P\times L$ is a relation where $sI\ell$
	is read as "$s$ lies on $\ell$". $\PP$ satisfies the following axioms.
	\begin{enumerate}
		\item For any two distinct $s,t\in P$, there is a unique $\ell\in L$
			such that $s$ and $t$ lie on $\ell$.

		\item For every $\ell\in L$ there are at least three distinct $s,t,u\in
			P$ that lie on $\ell$.

		\item There are at least three distinct $s,t,u\in P$ such that there is
			no $\ell\in L$ such that $s$, $t$ and $u$ all lie on $\ell$.

		\item For every $\ell,m\in L$ there is an $s\in P$ that lies on both
			$\ell$ and $m$.
	\end{enumerate}
\end{mdefinition}

Here are two easy lemmas and then two standard examples.

\begin{mlemma}
	Let $\ell$ and $m$ be two lines in an affine or projective plane and let
	$s,t$ be distinct points that lie on both $\ell$ and $m$. Then $\ell =m$.
\end{mlemma}

\begin{proof}
	The first property in each definition is that there is a unique line
	containing $s$ and $t$, so $\ell =m$.
\end{proof}

\begin{mlemma}
	In an affine plane $(A,L,I)$, $\parallel$ is an equivalence relation.
\end{mlemma}

\begin{proof}
	$\parallel$ is clearly reflexive and symmetric. To see that it is
	transitive, say that $\ell\parallel m$ and $m\parallel n$. We must see that
	$\ell\parallel n$. If there is a point $s\in A$ such that $s$ lies on both
	$\ell$ and $n$, then since there is a unique line parallel to $m$ on which
	$s$ lies, both $\ell$ and $n$ are this line and $\ell=n$. Otherwise, there
	is no point lying on both $\ell$ and $n$. Either way, $\ell\parallel n$.
\end{proof}

\begin{mexample}
	Let $k$ be a field. Then ${\mathbb A}^2(k)=(k^2,L,\in)$ where the lines are
	solution sets to equations $ax+by+c=0$ where $a,b,c\in k$ and at least one
	of $a$ and $b$ is not zero. Note that if $\lambda\in k^*$, then $ax+by+c=0$
	has the same solutions as $\lambda ax+\lambda by+\lambda c=0$. This is the
	only way that two different equations yield the same line. Linear algebra
	tells us that ${\mathbb A}^2(k)$ satisfies the first property of an affine
	plane. It is not hard to prove via a parameterization that every line has
	the same number of elements as $k$, which is at least $2$, so that the
	second property holds as well. An easy computation shows that for $(0,0)$,
	$(1,0)$ and $(0,1)$ to all be solutions to $ax+by+c=0$ then $a=b=c=0$. So,
	we have three non-collinear points. Finally, if $s=(s_1,s_2)$ is not a
	solution to $ax+by+c=0$, then there is a unique $d\in k$ such that $s$ is a
	solution to $ax+by+d=0$ and $d\ne c$. The lines defined by these two
	equations have empty intersection. So, ${\mathbb A}^2(k)$ satisfies all
	requirements of an affine plane.
\end{mexample}

Note that this example shows a way to construct finite affine planes. If
$k=\FF_q$, then ${\mathbb A}^2(\FF_q)$ has $q^2$ points and
$\frac{q(q^2-1)}{q-1}=q(q+1)$ lines. In fact, ${\mathbb A}^2(\FF_2)$ with $4$
points and $6$ lines is the smallest possible affine plane.

\begin{mexample}
	Let $k$ be a field. We will construct a projective plane for which the
	"points" are lines through the origin in $k^3$, while the "lines" are planes
	through the origin in $k^3$. More specifically, consider the equivalence
	relation $\equiv$ on $k^3\backslash\{(0,0,0)\}$ given by
	$(\alpha,\beta,\gamma)\equiv(\delta,\epsilon,\phi)$ if there is a
	$\lambda\in k^*$ such that
	$(\alpha,\beta,\gamma)=\lambda(\delta,\epsilon,\phi)$. That is, two nonzero
	vectors are equivalent if they are linearly dependent. (That is, determine
	the same line through the origin.) Denote the equivalence class of
	$(\alpha,\beta,\gamma)$ by $(\alpha:\beta:\gamma)$. For a linear polynomial
	$ax+by+cz$ with at least one of $a,b,c$ not zero, if the equation
	$ax+by+cz=0$ is satisfied by $(\alpha,\beta,\gamma)$ then it is also
	satisfied by everything in its equivalence class. Thus, it makes sense to
	say whether or not $(\alpha:\beta:\gamma)$ is a solution to $ax+by+cz=0$.

	${\mathbb P}^2(k)=((k^3\backslash\{(0,0,0)\})/\equiv,L,\in)$ where the lines
	are solution sets to equations $ax+by+cz=0$ with at least one of $a,b,c$
	nonzero. As in the previous example, for $\lambda\in k^*$ the solution sets
	of $ax+by+cz=0$ and $\lambda ax+\lambda by+\lambda cz=0$ are the same and
	this is the only way that two different equations yield the same line.

	Starting with the second property, let $ax+by+cz=0$ be the equation of a
	line $\ell$. If at least two of $a,b,c$ are not zero then $(b:-a:0), (c: 0:
	-a)$ and $(0:c:-b)$ are three distinct points on $\ell$. If $a=b=0$ so that
	$c\ne 0$, then $(1:0:0), (0:1:0)$ and $(1:1:0)$ are three distinct points on
	$\ell$. A similar construction applies if $a=c=0$ or if $b=c=0$.

	Moving to the last property, if two distinct lines have equations
	$ax+by+cz=0$ and $dx+ey+fz=0$, then there is exactly one common solution
	given by the equivalence class of the cross-product $(a,b,c)\times(d,e,f)$.
	Thus, given lines $\ell$ and $m$, either $\ell=m$ or $\ell\cap m$ has just
	one point. In either case, $\ell\cap m\ne\emptyset$.

	For the first property, given two distinct points $s$ and $t$ with
	representatives $(s_1,s_2,s_3)$ and $t=(t_1,t_2,t_3)$ if we let $s\times
	t=(a,b,c)$ then at least one of $a,b,c$ is not $0$, since $s$ and $t$ are
	linearly independent, and $s$ and $t$ are both solutions of to the equation
	$ax+by+cz=0$. This line is the unique line with this property, since as we
	have just seen any two lines that share more than one point are the same
	line.

	Finally, $(1:0:0),(0:1:0)$ and $(0:0:1)$ are not collinear since they are
	all solutions to $ax+by+cz=0$ then $a=b=c=0$. So, ${\mathbb P}^2(k)$ is a
	projective plane.
\end{mexample}

As in the previous example, this gives us a way to construct to finite
projective planes as ${\mathbb P}^2(\FF_q)$ for finite fields $\FF_q$, having
$\frac{q^3-1}{q-1}=q^2+q+1$ points and also $q^2+q+1$ lines. The smallest
possible projective plane is ${\mathbb P}^2(\FF_2)$, which has $7$ points and
$7$ lines. This projective plane is also known as the \emph{Fano plane}.

\begin{mproposition}
	Let $\PP=(P,L,I)$ be a projective plane and let $\ell_\infty\in L$. Let
	\[A=P\backslash\{s\in P|sI\ell_\infty\},\,\,\,
		L^\prime=L\backslash\{\ell_\infty\}\text{, and }I^\prime=I\cap(A\times
	L^\prime).\] Then $(A,L^\prime,I^\prime)$ is an affine plane.
\end{mproposition}

\begin{proof}
	For any two distinct $s,t\in A$ we also have that $s,t\in P$, so that there
	is a unique $\ell\in L$ such that $s$ and $t$ lie on $\ell$. Since $s,t\in
	A$, $\ell\ne\ell_\infty$ so that $\ell\in L^\prime$. This proves that
	$(A,L^\prime,I^\prime)$ satisfies the first axiom of an affine plane.

	For $\ell\in L^\prime$, $\ell\ne\ell_\infty$ so that there is exactly one
	point that lies on both $\ell$ and $\ell_\infty$.  So, exactly one fewer
	points in $A$ lie on $\ell$ than points in $P$ lie on $\ell$. Since at least
	three points in $P$ lie on $\ell$, at least two points in $A$ lie on $\ell$,
	satisfying the second axiom of an affine plane.

	Since $(P,L,I)$ is a projective plane, there are distinct $s,t,u\in P$ that
	are not collinear. In particular, at least one of them, say $s$, does not
	lie on $\ell_\infty$. Thus, $s\in A$. Let $\ell\in L$ be the line determined
	by $s$ and $t$ and let $m\in L$ be the line determined by $s$ and $u$. Note
	that $\ell\ne m$ since $s,t,u$ are not collinear. Since there are at least
	two points of $A$ lying on $\ell$ and at least two points of $A$ lying on
	$m$, there is a $v\in A$ lying on $\ell$ with $v\ne s$ and a $w\in A$ lying
	on $m$ with $w\ne s$. If $s,v,w$ all lay on a common line, then that line
	would share two points each with $\ell$ and $m$, so that it would be equal
	to both $\ell$ and $m$. Since $\ell\ne m$, $s,v,w$ do not lie on a common
	line, satisfying the third axiom of a projective plane.

	Finally, let $s\in A$ and $\ell\in L^\prime$. Let $t$ be the unique point
	that lies on both $\ell$ and $\ell_\infty$ and let $m$ be the line in $\PP$
	determined by $s$ and $t$. Since $s$ lies on $m$, $m\ne\ell_\infty$ so that
	$m\in L^\prime$. If $s$ lies on $\ell$ then $\ell=m$. If not, then $\ell\ne
	m$ and since $\ell$ and $m$ share the common point $t$ in $\PP$, they cannot
	share any points in $A$. In  either case, $\ell\parallel m$ in $\AA$.
\end{proof}

\begin{exercise}
	Klein's Erlangen program studies groupoids of geometric spaces of various
	kinds. Prove that the groupoid {\bf Affine} of affine planes is equivalent
	to the groupoid {\bf Proj\textsuperscript{l}} of projective planes with a
	distinguished line, called the \lq\lq line at infinity." The morphisms in
	each groupoid are bijections of both points and lines (preserving the
	distinguished line in the case of projective planes) that preserve and
	reflect the incidence relation. The functor $\Proj^\ell\rightarrow\Affine$
	removes the line at infinity and the points it contains. Explicitly describe
	an inverse equivalence.
\end{exercise}

That the functor $F:\Proj^\ell\rightarrow\Affine$ is well-defined on objects is
shown by the proposition above. For a morphism
$f:(P,L,I,\ell_\infty)\rightarrow(Q,M,J,m_\infty)$, the induced morphism
$Ff:(A,L^\prime,I^\prime)\rightarrow(B,M^\prime,J^\prime)$ is given by
restricting the bijections $P\rightarrow Q$ and $L\rightarrow M$ to $A$ and to
$L^\prime$. These restrictions give bijections to $B$ and to $M^\prime$ exactly
because the bijection from $L$ to $M$ takes $\ell_\infty$ to $m_\infty$. Then it
is easy to also see that $Ff$ is a morphism, that $F$ takes identities to
identities and that $F$ respects composition of morphisms. So, $F$ is a functor.

\subparagraph{}
We will now make a functor $G:\Affine\rightarrow\Proj^\ell$. First, we describe
$G$ on objects. Let $(A,L,I)$ be an affine plane. Let $\Pi$ be the set of
equivalence classes of $L$ under the relation $\parallel$. Let $P=A\cup\Pi$ and
let $\bar L=L\cup\{\ell^*_\infty\}$ for some $\ell^*_\infty$ that is not an
element of $A\cup\Pi\cup L$. For the sake of definiteness, take
$\ell^*_\infty=\{A\cup\Pi\cup L\}$. Let $\bar I\subseteq P\times\bar L$ be the
relation defined by $s\bar I\ell$ if either
\begin{enumerate}
	\item $s\in A$, $\ell\in L$ and $sI\ell$, or
	\item $s\in\Pi$ and $\ell=\ell^*_\infty$, or
	\item $s\in\Pi$ and $\ell\in s$.
\end{enumerate}

\subparagraph{}
We must see that $(P,\bar L,\bar I)$ is a projective plane. First, we mention an
annoying set theoretic issue. For this to be true, we need among other things
that $P\cap\bar L=\emptyset$, which is true so long as $\Pi\cap L=\emptyset$.
But, it is possible that one of the equivalence classes $s$ of "lines" is
already another "line" in $L$. We can avoid this by replacing each $s$ by the
Kuratowski product $s^\prime =s\times L=\{\{s\},\{s,L\}\}$. This cannot be equal
to any $\ell\in L$ for if it were then $\ell\in L\in\{s,L\}\in s^\prime =\ell$,
and such loops are impossible under the ZFC axioms. We will move forward as if
$\Pi\cap L=\emptyset$ so as to avoid obscuring the ideas under a weight of
additional notation. But, everything below can be adjusted to use $s^\prime$ in
place of $s$.

We will start by showing that any two distinct lines have exactly one common
point. If $\ell,m\in\bar L$ then at least one of them, say $\ell$, is in $L$. If
$m=\ell^*_\infty$, then from the second and third cases of the definition of
$\bar I$, we see that the only point that lies on both $\ell$ and
$\ell^*_\infty$ is the equivalence class $s$ of $\ell$ under $\parallel$. If
$m\in L$ as well then since $\ell\ne m$, either they are parallel in $(A,L,I)$,
in which case the only $s\in P$ that lies on both is their common equivalence
class, or they are not parallel, in which case they have one point in common via
case (1), but have no point in common in $\Pi$.

Now, we will see that any two points determine a unique line. Let $s,t\in P$ be
distinct points. They cannot both be on two different lines, since we have just
seen that two lines have exactly one point in common. So, it suffices to show
that they are on some line. If $s,t\in A$, then we already  know that there is
an $\ell\in L$ on which both $s$ and $t$ lie since $(A,L,I)$ is an affine plane
and $I$ is preserved in $\bar{I}$ via the first case of its definition. If
$s,t\in\Pi$, then $s$ and $t$ both lie on $\ell^*_\infty$ by case (2). We are
left with the case in which one of them, say $s$, is in $A$ and the other, $t$,
is in $\Pi$. Then by the last axiom of an affine plane, there is a unique line
$\ell\in L$ in the equivalence class $t$ on which $s$ lies. But, $t$ also lies
on $\ell$ by case (3).

Now we see that at least three distinct points lie on any line $\ell\in\bar L$.
If $\ell\in L$, then there are at least $2$ points in $A$ that lie on $\ell$.
But, the equivalence class of $\ell$ is an element of $\Pi$ that also lies on
$\ell$, giving $\ell$ at least $3$ distinct points. The remaining case is
$\ell=\ell^*_\infty$, whose points are the elements of $\Pi$. So, we must show
that there are at least $3$ equivalence classes of lines in $A$. To see that,
recall that we are guaranteed three distinct points $s,t,u\in A$ such that there
is no line on which they all lie. Let $\ell$, $m$ and $n$ be the lines in
$(A,L,I)$ determined respectively by $s$ and $t$, by $s$ and $u$ and by $t$ and
$u$. These three lines are distinct by the choice of $s,t,u$. But, each pair of
$\ell,m,n$ has a point in common, so no pair is parallel. Therefore, we have at
least three equivalence classes of parallel lines, giving at least three points
lying on $\ell^*_\infty$.

Finally, we must guarantee that we have at least three $s,t,u\in P$ that do not
lie on any common line in $(P,\bar L,\bar I)$. We may just take $s,t,u\in A$
that do not lie on a common line in $(A,L,I)$.  Then they maintain this property
in $(P,\bar L,\bar I)$.

\subparagraph{}
So, we take $G(A,L,I)=(P,L,I,\ell^*_\infty)$, the projective plane just
constructed with a distinguished line at infinity, also newly constructed.

\subparagraph{}
Say that $f:(A,L,I)\rightarrow(B,M,J)$ is a morphism in $\Affine$. We must
describe
\[Gf:(P,\bar L,\bar I,\ell^*_\infty)\rightarrow(Q,\bar M,\bar J,m^*_\infty).\]
First note that since $f$ is a bijection on both points and lines and also
$sI\ell$ if and only  if $f(s)Jf(\ell)$, it follows that $\ell\parallel m$ in
$(A,L,I)$ if and only if $f(\ell)\parallel f(m)$ in $(B,M,J)$.  So, the
bijection from $L$ to $M$ induces a bijection between parallel equivalence
classes in $L$ and in $M$. Thus, $f$ extends to a bijection from $P$ to $Q$.  We
also extend $f$ to a bijection from $\bar L$ to $\bar M$ by taking
$\ell^*_\infty$ to $m^*_\infty$. This gives a description of $Gf$ as a function.

To see that $Gf$ is a morphism in $\Proj^\ell$, it remains to be seen that
$s\bar I\ell$ in $(P,\bar L,\bar I)$ if and only if $Gf(s)\bar JGf(\ell)$ in
$(Q,\bar M, \bar J)$. Examining the three ways in which we could have $s\bar
I\ell$ in the definition of $\bar I$, we see that in each case $Gf(s)$ and
$Gf(\ell)$ are in the very same case, and conversely. So, $Gf$ is a morphism in
$\Proj^\ell$.

Now, it is easy to that $G$ takes identities to identities and preserves
composition of morphisms. So, $G$ is a functor.

\subparagraph{}
Consider the composite functor $FG:\Affine\rightarrow\Affine$. $G$ adds new
points to $A$ and a new line to $L$ on which the new points lie, but $F$ takes
away that new line and also all of those new points returning  us to $A$ and to
$L$. Also $(\bar I)^\prime =\bar I\cap(A\times L)=I$. So,  on objects $FG$ is
the identity. But it is on morphisms as well, since as a function $Gf$ is an
extension of $f$ to $P$ and $\bar L$, but $F$ just restricts $Gf$ to the
original sets, giving that $FGf=f$. So, $FG=1_\Affine$.

\subparagraph{}
However, $GF\ne 1_{\Proj^\ell}$. Indeed, if we write
$A=P\backslash\{s|sI\ell_\infty\}$ and $L^\prime =L\backslash\{\ell_\infty\}$
and $I^\prime=I\cap(A\times L^\prime)$ as above, then
\[GF(P,L,I,\ell_\infty)=(A\cup\Pi,
L^\prime\cup\{\ell^*_\infty\},\overline{I^\prime},\ell^*_\infty).\] But, there
is a correspondence between what was taken away by $F$ and what was added by
$G$. Namely, if $s\in P$ lies on $\ell_\infty$ then the other lines on which $s$
lies form an equivalence class $s^*\subset L^\prime$ for $\parallel$ in
$(A,L^\prime,I^\prime)$. Indeed, any two distinct elements of $s^*$ share $s$ in
$P$, and so cannot also share any points in $A$. Any two lines that are parallel
in  $A$ must share a point in $P$, necessarily one that lies on $\ell_\infty$.
This gives a bijection \[\{s\in
P|sI\ell_\infty\}\leftrightarrow\{s^*\in\Pi\}=\{t\in
A\cup\Pi|t\overline{I^\prime}\ell_\infty^*\}.\] We can fit these together into a
natural transformation $\eta:1_{\Proj^\ell}\Rightarrow GF$. Namely, if
$\PP=(P,L,I,\ell_\infty)$ then \[\eta_{\PP}:(P,L,I,\ell_\infty)\rightarrow
(A\cup\Pi, L^\prime\cup\{\ell^*_\infty\},\overline{I^\prime},\ell^*_\infty)\] is
given by
\[\begin{array}{l}
		s\mapsto\begin{cases} s^*\text{ if }sI\ell_\infty\\ s\text{ otherwise,
		and}\end{cases}\\
		\ell\mapsto\begin{cases} \ell^*_\infty\text{ if }\ell=\ell_\infty\\
		\ell\text{ otherwise.}\end{cases}
	\end{array}
\]
It is easy to see that $\eta_{\PP}$ is a morphism, and in particular an
isomorphism as all morphisms in $\Proj^\ell$ are.

To check that $\eta$ is a natural transformation, and thus a natural
isomorphism, let\footnote{Note that $\QQ$ does not represent the rational
numbers for the moment.} $\QQ=(Q,M,J,m_\infty)$ and $f:\PP\rightarrow\QQ$ be a
morphism. Then we need to check that \[\xymatrix{\PP\ar[r]^f\ar[d]_{\eta_\PP} &
\QQ\ar[d]^{\eta_\QQ}\\ GF\PP\ar[r]^{GFf} & GF\QQ}\] commutes. Checking along
both paths from $\PP$ to $GF\QQ$, we find that both give
\[\begin{array}{l}
		s\mapsto\begin{cases} f(s)^*\text{ if }sI\ell_\infty\\ f(s)\text{
		otherwise, and}\end{cases}\\
		\ell\mapsto\begin{cases} m^*_\infty\text{ if }\ell=\ell_\infty\\
		f(\ell)\text{ otherwise.}\end{cases}
	\end{array}
\]

Since $\eta$ is a natural isomorphism, we have that $F$ and $G$ define an
equivalence of categories as claimed.
\end{document}
