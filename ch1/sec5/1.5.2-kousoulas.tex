\documentclass[main.tex]{subfiles}
\begin{document}

\paragraph{}
\setexercise{1}{5}{2}
\begin{exercise}
	Segal defined a category \(\Gm\) as follows:
	\begin{quote}
		\(\Gm\) is the category whose objects are all finite sets, and whose
		morphisms from \(S\) to \(T\) are the maps \(\func{\th}{S}{P(T)}\) such
		that \(\th(\al)\) and \(\th(\be)\) are disjoint when \(\al\ne\be\). The
		composite of \(\func{\th}{S}{P(T)}\) and \(\func{\phi}{T}{P(U)}\) is
		\(\func{\psi}{S}{P(U)}\), where
		\(\psi(\al)=\bigcup\limits_{\be\in\th(\al)}\phi(\be)\).
	\end{quote}
	Prove that \(\Gm\) is equivalent to the opposite of the category \(\Finp\)
	of finite pointed sets. In particular, the functors introduced in Example
	1.3.2(xi) define presheaves on \(\Gm\).
\end{exercise}
\begin{proof}
	As a preliminary matter, it is worth checking that \(\Gm\) is indeed a
	category. The definition above makes clear that an arrow
	\(\func{\th}{S}{P(T)}\) has domain \(S\) and codomain \(T\), as well as
	telling us how to compose two morphisms. What is less obvious is what
	morphisms are the identities, and that composition is associative. For
	identities, let us first note that the identity on \(S\) in \(\Gm\) is a
	function \(\func{\id_S}{S}{P(S)}\) meaning the usual identity on the set
	\(S\) is not a possibility. However, there is a natural embedding of \(S\)
	in \(P(S)\) which takes every element \(\al\) to the singleton \(\{\al\}\).
	Checking composition of \(\id_S\) and \(\id_T\) thus defined with an
	arbitrary map \(\func{\th}{S}{P(T)}\) gives:
	\begin{alignat*}{4}
		\th\id_S(\al)&=
		\bigcup_{\be\in\id_S(\al)}\th(\be)&=
		\bigcup_{\be\in\{\al\}}\th(\be)&=
		\th(\al) \\
		\id_T\th(\al)&=
		\bigcup_{\be\in\th(\al)}\id_T(\be)&=
		\bigcup_{\be\in\th(\al)}\{\be\}&=
		\th(\al)
	\end{alignat*} verifying that we have defined the identities properly.
	Finally, we have to check that the composition law is associative. Let
	\(\th\) be as above along with \(\func{\phi}{T}{P(U)}\) and
	\(\func{\psi}{U}{P(V)}\) be valid morphisms in \(\Gm\). Then
	\begin{alignat*}{3}
		((\psi\phi)\th)(\al)=&&
		\bigcup_{\be\in\th(\al)}\phi\psi(\be)=&
		\bigcup_{\be\in\th(\al)}\qty{\bigcup_{\gm\in\phi(\be)}\psi(\gm)},\\
		\intertext{and}
		(\psi(\phi\th))(\al)=&&
		\bigcup_{\gm\in\phi\th(\al)}\psi(\gm)=&
		\bigcup_{\gm\in\qty{\bigcup\limits_{\be\in\th(\gm)}\phi(\be)}}\psi(\gm).
	\end{alignat*}
	At first appearance it is not at all clear that these ought to be the same
	set, but unpacking makes it clear they are in fact the same. Both assert
	that for any \(\de\in(\psi(\phi\th))\) there exists some \(\be\in\th(\al)\)
	and \(\gm\in\phi(\be)\) such that \(\de\in\psi(\gm)\). Thus our composition
	law is associative and we have indeed defined a category.

	\subparagraph{}
	An equivalence of categories consists of two functors along with two natural
	isomorphisms between their compositions and the identity functor on each
	category. We will thus begin by defining two functors
	\(\afunc{+}{\inv{(-)}}{\Gm}{\Finp^\op}\). We define \(+\) by the following
	mappings:
	\begin{align*}
		S&\mapsto\qty{S\cup\set{S},S} \\
		\func{\th}{S}{T}&\mapsto\func{+\th}{+T}{+S} \\
		\intertext{where}
		+\th(\be)&=\begin{cases}
			S        & \IF\be=T \\
			S        & \IF\be\notin\bigcup_{\al\in S}\th(\al) \\
			\al\in S & \IF\be\in\th(\al) \\
		\end{cases}.
	\end{align*}
	Note first that this is a valid map from \(+T\) to \(+S\) in \(\Finp\) (and
	thus a map from \(+S\) to \(+T\) in \(\Finp^\op\)): it takes the base point
	of \(+T\) to the base point of \(+S\), if \(\al\in\th(\gm)\) for some
	\(\gm\in S\), then this \(\gm\) must be unique since the image of distinct
	elements of \(S\) under \(\th\) must be disjoint. We must further check the
	functoriality axioms. Recall that the identity map on an object \(S\) of
	\(\Gm\) is \(\maps{\id_S}{\al}{\{\al\}}\). Applying this to the definition
	above gives:
	\[+\id_S(\al)=
		\begin{cases}
			S         & \IF\al=S \\
			S         & \IF\al\notin\bigcup_{\al'\in S}\id_S(\al')=S\\
			\al'\in S & \IF\al\in\id_S(\al')=\set{\al'} \\
		\end{cases},
	\qtext{which is clearly the identity.}\] So \(+\id_S=\id_{+S}\).

	Further, given morphisms \(\func{\th}{S}{T}\) and \(\func{\phi}{T}{U}\)
	between objects of \(\Gm\), we have that
	\begin{align*}
		+\phi(\gm)=
		\begin{cases}
			T        & \IF\gm=U \\
			T        & \IF\gm\notin\bigcup_{\be\in T}\phi(\be) \\
			\be\in T & \IF\gm\in\phi(\be) \\
		\end{cases},
		&&+\th(\be)=
		\begin{cases}
			S        & \IF\be=T \\
			S        & \IF\be\notin\bigcup_{\al\in S}\th(\al) \\
			\al\in S & \IF\be\in\th(\al) \\
		\end{cases}, \\
	\end{align*}\vskip-3em
	\begin{align*}
		\textq{and}+\phi\th(\gm)=
		\begin{cases}
			S        & \IF\gm=U \\
			S        & \IF\gm\notin\bigcup_{\al\in S}\phi\th(\al)=\bigcup_{\al\in S}\bigcup_{\be\in\th(\gm)}\phi(\be) \\
			\al\in S & \IF\gm\in\phi\th(\al)=\bigcup_{\be\in\th(\al)}\phi(\be) \\
		\end{cases}.
	\end{align*}
	Letting \(\gm\in+U\) be arbitrary there are several cases. First, if
	\(\gm=U\), then \(+\phi\th(\gm)=S\) and \(+\phi(\gm)=T\) so
	\(+\th+\phi(\gm)=S\). Alternately, if \(\gm\neq U\) so \(\gm\in U\) we again
	have two cases. If there exists \(\al\in S\) such that
	\(\gm\in\phi\th(\al)\), then \(+\phi\th(\gm)=\al\) and there exists a
	\(\be\in\th(\al)\) such that \(\gm\in\phi(\be)\) implying \(+\phi(\gm)=\be\)
	and \(+\th(\be)=\al\) so \(+\th+\phi(\gm)=\al\). Finally, if there exists no
	such \(\al\in S\), then \(+\phi\th(\gm)=S\). If there also exists no
	\(\be\in T\) such that \(\gm\in\phi(\be)\) then \(+\phi(\gm)=T\) so
	\(+\th+\phi(\gm)=S\). However, is such \(\be\in T\) does exist so that
	\(+\phi(\gm)=\be\) then it must be the case that

	\subparagraph{}
	Next we define the functor \(\inv{(-)}\) by:
	\begin{align*}
		(S,s)&\mapsto S\setminus\set{s} \\
		\func{f}{(T,t)}{(S,s)}&\mapsto\func{\inv{f}}{S\setminus\set{s}}{T\setminus\set{t}} \\
		\intertext{where}
		\inv{f}(\al)&=\set{\be\in T\mid f(\be)=\al}.
	\end{align*}
	% note T in the above should technically be T\{t}, but it doesn't matter
	% because t cannot map to anything other than s which we removed
	Note that if \(\inv{f}(\al)\cap\inv{f}(\al')\) is inhabited, then there is
	some \(\be\in T\) such that \(f(\be)=\al\) and \(f(\be)=\al'\) implying that
	\(\al=\al'\). Thus the map we have defined satisfies the disjointness
	condition sufficient to be an morphism in \(\Gm\). Again we must check
	functoriality axioms. The identity morphism on \((S,s)\) is merely the
	identity function on \(S\), so
	\begin{align*}
		\inv{\id_S}(\al)
		&=\set{\al'\in S\mid\id_S(\al')=\al} \\
		&=\set{\al'\in S\mid\al'=\al}=\set{\al}=\id_{\inv{(-)}S}(\al).
	\end{align*}
	Given morphisms \(\func{f}{(T,t)}{(S,s)}\) and \(\func{g}{(U,u)}{(T,t)}\) we
	have that
	\begin{align*}
		\inv{(fg)}(\al)&=\set{\gm\in U\mid fg(\gm)=\al} \\
		&=\bigcup_{f(\be)=\al} \set{\gm\in U\mid g(\gm)=\be} \\
		&=\bigcup_{\be\in\inv{f}(\al)}\inv{g}(\be)
		=\inv{g}\inv{f}(\al).
	\end{align*}

	\subparagraph{}
	We thus have two bona fide functors connecting \(\Gm\) and \(\Finp^\op\).
	What remains is to construct natural isomorphisms
	\(\isom{\eta}{\id_\Gm}{\inv{(-)}+}\) and
	\(\isom{\ep}{+\inv{(-)}}{\id_{\Finp^\op}}\). Looking first at \(\eta\),
	given an object \(S\) of \(\Gm\) we have that
	\[\inv{(-)}+S=\inv{(-)}(S\cup\set{S},S)=(S\cup\set{S})\setminus\set{S}=S\]
	meaning that the collection of maps which make up \(\eta\) will be
	automorphisms. Further, given \(\func{\th}{S}{T}\) and
	\(\al\in\inv{(-)}+S=S\), we have
	\[\inv{(-)}+\th(\al)=\set{\be\in T|+\th(\be)=\al}.\] Now, recall that
	\(+\th(\be)=\al\) precisely when \(\be\in\th(\al)\) and thus
	\[\set{\be\in T|+\th(\be)=\al}=\th(\al)\] so that \(\inv{(-)}+\) is the
	identity functor on \(\Gm\) and we may take our natural isomorphism to be
	the collection of identity maps \(\id_S\) for each object \(S\) of
	\(\Gamma\).

	Now for \(\ep\), given an object \((S,s)\) of \(\Finp^\op\) we have that
	\[+\inv{(-)}(S,s)=+S\setminus\set{s}=((S\setminus\set{s})
	\cup\set{S\setminus\set{s}},S\setminus\set{s})\]
	For sanity's sake we will use the notation \((S_*\cup\set{S_*},S_*)\) for
	the above set. This grotesque looking object (which amounts to replacing
	\(s\) with something more generic) is not \(S\). However, there is a based
	isomorphism \(\ep_s\) from \(S\) to it which takes \(s\) to
	\(S\setminus\set{s}\) and leaves the other elements of \(S\) be. The
	collection of such isomorphisms will form \(\ep\). Similarly, given a map
	\(\func{f}{(T,t)}{(S,s)}\) we have that
	\[+(\inv{f})(\be)=\begin{cases}
			S_* & \IF \be=T_* \\
			S_* & \IF \be\notin\bigcup_{\al\in S_*}\inv{f}(\al)=
			\bigcup_{\al\in S_*}\set{\be\in T|f(\be)=\al}\\
			\al\in S_* & \IF \be\in\inv{f}(\al)=\set{\be\in T|f(\be)=\al}
	\end{cases}.\]

	To verify that these do constitute a natural isomorphism we must check that
	the following diagram commutes:
	\[\xymatrix{(S,s)\ar[d]_{\ep_S}&(T,t)\ar[l]_f\ar[d]^{\ep_T}\\
	S_*&T_*\ar[l]^{+\qty{\inv{f}}}}\]
	Let \(\be\) be an element of \(T,t\), there are two cases to consider. If
	\(\be=t\), then
	\[\ep_Sf(t)=\ep_S(s)=S_*\qtextq{and}
	+\qty{\inv{f}}\ep_T(t)=+\qty{\inv{f}}T_*=S_*.\] If \(\be\neq t\), then
	either \(f(\be)=s\) or \(f(\be)\neq s\). In the first case, there is no element
	\(\al\in S\setminus\set{s}\) such that \(f(\be)=\al\). Since \(\be\) is not in
	any of the fibers from \(S\setminus\set{s}\), by definition it is taken to
	\(S_*\) by \(+\qty{\inv{f}}\), thus \[\ep_Sf(\be)=\ep_S(s)=S_*\qtextq{and}
	+\qty{\inv{f}}\ep_T(\be)=+\qty{\inv{f}}(\be)=S_*.\] Finally, in the
	interesting case where \(f(\be)\neq s\), we have \(f(\be)=\al\) for some
	\(\al\in S\setminus\set{s}\), so \(\be\in\inv{f}(\al)\), and thus
	\[\ep_Sf(\be)=\ep_S\al=\al\qtextq{and}
	+\qty{\inv{f}}\ep_T(\be)=+\qty{\inv{f}}(\be)=\al.\] Thus we have defined a
	natural isomorphism and we may conclude that \(\Gm\) is equivalent to
	\(\Finp^\op\).
\end{proof}
\end{document}

