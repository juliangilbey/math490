\documentclass[main.tex]{subfiles}
\begin{document}

\paragraph{}
\settheorem{1}{3}{8}
\begin{lemma}
	Functors preserve isomorphisms.
\end{lemma}
\settheorem{1}{5}{9}
\begin{theorem} \textnormal{(characterizing equivalences of categories).}
	A functor defining an equivalence of categories is full, faithful, and
	essentially surjective on objects. Assuming the axiom of choice, any functor
	with these properties defines an equivalence on categories.
\end{theorem}
\setexercise{1}{5}{6}
\begin{exercise}\ \\
	\begin{enumerate}[i]
		\item Prove that the composite of a pair of full, faithful, or
			essentially surjective functors again has the same properties.

		\item Prove that if $\cat{C}\simeq\cat{D}$ and $\cat{D}\simeq\cat{E}$,
			then $\cat{C}\simeq\cat{E}$. Conclude that the equivalence of
			categories is an equivalence relation.
	\end{enumerate}
\end{exercise}

\begin{proof}\ \\
	\begin{itemize}
		\item Let $\func{F}{C}{D}$ and $\func{G}{D}{E}$ be functors. If $F$ and
			$G$ are full, but $GF$ is not full, then there is some
			$c\in\cat{C}(x,y)$, for some $x,y$ which make sense, such that
			$GFc\notin \cat{E}(GFx, GFy)$. However, $GFc = G(Fc) = Gd \in
			\cat{E}(GFx, GFy)$ by the fullness of $G$, so $GF$ is full.\\ To
			show that $GF$ is faithful if $F$ and $G$ are faithful, by a similar
			argument, if $GF$ were not faithful, then there would be a morphism
			in $\cat{E}(GFx, GFy)$ mapped to by two morphisms of $\cat{C}(x,y)$.
			However, $GFc = G(Fc) = Gd$ for a unique $d\in\cat{D}(Fx, Fy)$,
			which is again injective by the faithfulness of $G$.\\ Finally, to
			show that $GF$ is essentially surjective if $F$ and $G$ are
			essentially surjective, we see from Lemma 1.3.8 that functors take
			isomorphisms to isomorphisms. Since $G$ is essentially surjective,
			for each $d\in\cat{D}$, there is some $e\in\cat{E}$ such that
			$Gd\cong e$. Then, since $F$ has the same property, $d$ must be
			isomorphic to some $Fc$, that is, $Fc\cong d$. So we have that $GFc
			\cong Gd \cong e$, which is the requirement for $GF$ to be
			essentially surjective.

		\item If $\cat{C}\simeq\cat{D}$ and $\cat{D}\simeq\cat{E}$, then the
			functors $F$ and $G$ are fully faithful and essentially surjective,
			and by Theorem 1.5.9, $\cat{C}\simeq\cat{E}$. We also have that
			$\cat{C}\simeq\cat{C}$ (reflexivity), if $\cat{C}\simeq\cat{D}$ then
			$\cat{D}\simeq\cat{C}$ (symmetry), and from what we just showed we
			get transitivity. Thus, equivalence of categories defines an
			equivalence relation.
	\end{itemize}
\end{proof}

\end{document}
