\documentclass[main.tex]{subfiles}
\begin{document}

\subparagraph{\(\Ring\to\Ab\)}
\begin{proof}
	Let \(F\) be the functor described above. The additive group of any ring is
	already abelian, so \(F\)  takes the additive group of each \(r \in
	\textsf{Ring}\) (in other words, every \((r, +) \in \textsf{Ring}\)) to the
	same \((r, +) \in \textsf{Ab}\); and takes every ring homomorphism \(f
	\colon r \to s \in \textsf{Ring}\) to \(f_+ \colon (r, +) \to (s, +) \in
	\textsf{Ab}\),  where we define \(f_+\) as exactly the function \(f\), but
	applied only to the additive group of \(r\) (instead of to the whole ring.)

	So, in a sense, \(F\) is an 'inclusion functor,' taking the additive groups of
	elements of \(\textsf{Ring}\) to those same groups in \(\textsf{Ab}\), and the
	ring homomorphisms between those additive group in \(\textsf{Ring}\) to the same
	homomorphisms in \(\textsf{Ab}\).

	% \paragraph{F is faithful.}

	\subparagraph{}
	When we say that \(F\) takes every morphism to 'itself', what we are really
	saying is the following: For any \(f \colon r \to s \in \textsf{Ring}\), if
	we define \(f_+ \colon (r, +) \to (s, +)\) as \(f\) applied to the additive
	group of \(r\), we can say that \(Ff = f_+\in \textsf{Ring}\). So it is
	trivial that for any \(f, g \colon r \to s\) such that \(Ff \neq Fg\), \(f_+
	\neq g_+\).

	But the ring homomorphisms \(f, g \colon r \to s\) can be equal only if
	\(f(a+b) = g(a+b)\). In other words, \(f = g \) only if \(Ff = f_+ = g_+ =
	Fg\); so if \(Ff \neq Fg\), then of course \(f \neq g \)! So \(F(r, s)\) is
	injective by definition, and since this is the case for all \(r, s \in
	\textsf{Ring}\), we can conclude that F is faithful.

	% \paragraph{F is not full.}

	\subparagraph{}
	On the other hand, consider that there are no morphisms from the zero ring,
	\(\{0\}\), to any nonzero \(r \in \textsf{Ring}\). \(F\{0\}\) is simply
	\((\{0\}, +)\); that is, the trivial group. But there is automatically a
	group homomorphism from the trivial group to any group, which means that the
	set of morphisms from \(F\{0\}\) to \(Fr\) is \textit{not} empty for any \(r
	\in \textsf{Ring}\). So \(F\) cannot be surjective over \(\{f : F\{0\} \to
	Fr\}\), which means \(F\) is not full.

	% \paragraph{F is not essentially surjective.}

	\subparagraph{}
	Suppose to the contrary that \(F\) is essentially surjective. This means
	that, for example, there must be some \(r \in \textsf{Ring}\) such that \(Fr
	= (r, +) \simeq \QQ/\ZZ \in \textsf{Ab}\). Consider that elements of
	\(\QQ/\ZZ\) take the form \(\left\{\frac{a}{b} + \ZZ \mid a, b \in
	\ZZ\right\}\); which means that for every \(n \in \QQ/\ZZ\), there is some
	positive integer \(b\) such that if you 'multiply' \(n\) by \(b\) (using the
	definition we mentioned earlier of "adding n to itself b times") you obtain
	that \(b*n = a + \ZZ = \ZZ = 0_{\QQ/\ZZ}\).

	But consider the case where \(n = 1_r\). We have just determined that there
	must be some positive integer \(b\) such that \(b*1 = 0\). This means that
	\((r, +)\) must have characteristic \(b\), which itself means that every
	element of \((r, +)\) must have order \(\leq b\). So since \((r, +) \simeq
	\QQ/\ZZ\), every element of \(\QQ/\ZZ\) must also have order \(\leq b\). But
	we know that there is some element of \(\QQ/\ZZ\) with order \(n\) for
	\textit{any} positive n; there cannot be a finite \(b\) such that every
	element of \(\QQ/\ZZ\) has order \(\leq b\). This brings us to a
	contradiction, which means that our assumption must be false -- \(F\) must
	not be essentially surjective.
\end{proof}

\end{document}
