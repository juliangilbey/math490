\documentclass[main.tex]{subfiles}
\begin{document}

\paragraph{}
\begin{exercise}
	Let \textsf{G} be a connected groupoid and let $G$ be
	the group of autmorphisms at any of its objects. The inclusion $\textsf{B}G
	\hookrightarrow \textsf{G}$ defines an equiavlence of categories. Construct an
	inverse equiavlence $\textsf{G} \rightarrow \textsf{B}G$.
\end{exercise}
\begin{proof}
	To construct the inverse equivalence between  \textsf{G} and $\textsf{B}G$,
	we must find a fully faithful and essentially surjective functor between
	these categories. First, we see that since $\textsf{B}G$ has just one
	object and any two singleton sets are isomorphic, that any functor $F\colon \textsf{G} \rightarrow \textsf{B}G$ will be
	essentially surjective. We define our functor $F$ on the objects of
	\textsf{G} by sending any object of \textsf{G} to the one object of
	$\textsf{B}G$. Before defining our functor on morphisms, we first show that
	for every trio of objects $x,y,z \in \textsf{G}$, there exists a bijection
	from $G = \textsf{G}(x,x) $ to $\textsf{G}(y,z)$.

	First, we define a class of reference morphisms for our groupoid in the
	following manner: Because $\textsf{G}$ is connected, we have at least one
	morphism in $\textsf{G} (x,y)$ for any $y \in \ob\textsf{G}$. By the Axiom of
	Choice, for each $y$, we can choose a morphism $f_y$.\footnote{Note that if
		$\textsf{G}$ is not small, we must apply the Axiom of Choice in a larger
		universe. Since $\textsf{G}$ is a groupoid, we also have a morphism
		$f_y^{-1}$ for each $y$. We also note that if $x = y$, we choose $f_y =
	1_y$.}

	Now, we use this subclass of morphisms to first determine a bijection $\rho$
	between $\textsf{G}(x,x) \rightarrow \textsf{G}(x,y)$. We define this
	bijection by sending $\gamma \in \textsf{G}(x,x)$ to $f_y\gamma \in
	\textsf{G}(x,y)$. It is easy to see that this function is injective, because
	$f_y$ is invertible, and surjective, as for any $g \in \textsf{G}(x,y)$, $g
	= \rho(f_y^{-1}g)$, where $f_y^{-1}g \in \textsf{G}(x,x)$. So we have the
	desired bijection.

	Next, we see that we can define bijections  $\sigma$ from $\textsf{G}(x,x)$
	to $\textsf{G}(y,x)$, where $\sigma(\gamma) = \gamma f_y^{-1}$ which is
	bijective by a similar argument as above. We can then compose these
	bijections to define a bijection $\phi = \sigma\rho$ from $\textsf{G}(x,x)$
	to $\textsf{G}(y,z)$, where $\phi(\gamma) = f_z\gamma f_y^{-1}$.

	Clearly, we also have an inverse bijection $\phi^{-1}\colon \textsf{G}(x,y)
	\rightarrow G$, where for a $g \in \textsf{G}(y,z)$ such that $g = f_z\gamma
	f_y^{-1}$, $\phi^{-1}(g) = \gamma$. We will use this to define our functor
	on morphisms, so that for a morphism $g\colon y\rightarrow z$, $Fg =
	\phi^{-1}(g)$. First, we note that in this setting $f_x = 1_x$ and that for
	any $y$, $f_y = f_y\gamma f_x = f_y\gamma$, so that $Ff_y = 1_x$. Also, for
	any $y$, $1_y = f_y 1_x f_y^{-1}$, so our functor preserves identities.

	Now, we show that our functor preserves composition of morphisms. To do
	this, consider $g\colon y\rightarrow z$ and $h\colon z \rightarrow w$, where $g =
	f_z\gamma_gf_y^{-1}$ and $h = f_w\gamma_hf_z^{-1}$. Now, consider $Fhg$. We
	know that $hg = f_w\gamma_hf_z^{-1}f_z\gamma_gf_y^{-1} =
	f_w\gamma_h\gamma_gf_y^{-1}$. So $Fhg = \gamma_h\gamma_g = FhFg$. So we see
	that our functor preserves composition of morphisms, and therefore we have a
	well defined functor.

	To see that our functor is fully faithful, we remember that for any $y,z \in
	\ob\textsf{G}$, $\textsf{G}(y,z)$ is in bijection with $G$. Since $Fy = Fz =
	\emptyset$, the only object of $\textsf{B}G$, and the set
	$\textsf{B}G(\emptyset, \emptyset) = G$, we have a bijection between
	$\textsf{G}(y,z)$ and $\textsf{B}G(Fy, Fz)$ and therefore a fully faithful
	functor. Therefore, we have a fully faithful functor that is essentially
	surjective on objects from $\textsf{G}$ to $\textsf{B}G$.

	Now, call the functor defined in 1.5.12  $\iota\colon \textsf{B}G \rightarrow
	\textsf{G}$. We must now define natural transformations $\tau\colon F\iota
	\Rightarrow 1_{\textsf{B}G} $ and $\eta\colon 1_{\textsf{G}}\Rightarrow \iota F$.
	For all $\gamma \in\mor\textsf{B}G$, we have that $F\iota(\gamma) =
	F(\gamma) = \gamma$ and that $F\iota(\emptyset) = Fx = \emptyset$ so $F\iota
	= 1_{\textsf{B}G}$, and the natural transformation is the identity
	transformation. Now, for each $y \in\ob\textsf{G}$, we must find $\eta_y$,
	so that the following diagram commutes for every $f\colon y \rightarrow z$.
	\[\xymatrix{
			\iota Fy = x
			\ar[r]^{\iota Ff}
			\ar[d]_{\eta_y}
			& \iota Fz = x
			\ar[d]^{\eta_z}
			\\
			y
			\ar[r]_{f}
			& z
	}\]
	We claim that if $\eta_y = f_y$, the reference morphism picked earlier, than
	we will have formed a natural transformation. First, note that since $f$ can
	be represented by $f_z\gamma f_y^{-1}$ for some automorphism $\gamma\colon x
	\rightarrow x$ and that $\gamma = f_z^{-1}ff_y$. Also, note that $\iota Ff =
	\iota(\gamma) = \gamma$. So we must show that $f_z\gamma = ff_y$. We know
	that $f_z\gamma = f_zf_z^{-1}ff_y = ff_y$, so we have the desired equality.
	So the following diagram commutes, and $\eta$ is a natural transformation.
	\begin{align}
		\xymatrix{
			x
			\ar[r]^{\iota Ff}
			\ar[d]_{f_y}
			&  x
			\ar[d]^{f_z}
			\\
			y
			\ar[r]_{f}
			& z
		}
	\end{align}
	We also see that $\eta_y$ is an isomorphism for every $f_y$, because
	$\textsf{G}$ is a groupoid. So we have shown the existence of the desired
	natural isomorphisms. Therefore, $F$ and $\iota$ define a equivalence
	between  the categories $\textsf{B}G$ and $\textsf{G}$.
\end{proof}
\end{document}
