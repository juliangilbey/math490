\documentclass[main.tex]{subfiles}
\begin{document}

\settheorem{1}{2}{7}
\begin{definition}
	A morphism \(\func{f}{x}{y}\) in a category is
	\begin{enumerate}[(i)]
		\item a \emph{monomorphism} if for any parallel morphisms
			\(\pfunc{h,k}{w}{x}\), \(fh=fk\) implies that \(h=k\); or
		\item an \emph{epimorphism} if for any parallel morphisms
			\(\pfunc{h,k}{y}{z}\), \(hf=kf\) implies that \(h=k\).
	\end{enumerate}
\end{definition}
\popthm

\setexercise{1}{2}{5}
\begin{exercise}
	Show that the inclusion \(\ZZ\emb\QQ\) is both a monomorphism and an
	epimorphism in the category \Ring of rings. Conclude that a map that is both
	monic and epic need not be an isomorphism.
\end{exercise}
\begin{proof}
	Note first that monic and epic correspond to a map being cancellable on the
	left and right, whereas an isomorphism is by definition invertible. It is
	easy to see that invertibility implies cancellability; however the converse
	need not be true. Looking at the monoid of natural numbers under addition,
	every element is cancellable; however none except zero is invertible.
	Because every monoid is a category this gives us an elementary example where
	a map that is monic and epic is not an isomorphism. However, this example
	might seem simplistic and it is worth asking whether there is an example of
	cancellability not implying invertibility in a ``larger'' category where the
	arrows represent actual maps.

	Recall that \(\QQ\) is the localisation of \(\ZZ\) with respect to its
	cancellable elements \(\ZZ\setminus\{0\}\). The immediate result of this is
	the existence of a natural embedding \(\func{\io}{\ZZ}{\QQ}\) that is an
	injective ring homomorphism. Further, this embedding has the following
	universal property: given a ring \(R\) and a homomorphism
	\(\func{\phi}{\ZZ}{R}\) such that \(\phi(q)\) has an inverse for all
	\(q\in\ZZ\), there is a unique ring homomorphism \(\func{\psi}{\QQ}{R}\)
	such that the following diagram commutes.
	\[\xymatrix{\ZZ \ar[rr]^\phi \ar[rd]_\io && R \\ & \QQ \ar[ur]_\psi}\]

	Note further that there is a unique ring homomorphism from \(\ZZ\) to any
	ring \(R\) which maps \(\ZZ\) onto the subring generated by the
	multiplicative identity of \(R\). This implies that there can be at most one
	homomorphism from \(\QQ\) to any ring \(R\). If \(\func{\psi}{\QQ}{R}\) is a
	ring homomorphism, then \(\func{\psi\io}{\ZZ}{R}\) must be the unique
	homomorphism from \(\ZZ\) to \(R\) and thus \(\psi\) is the unique
	homomorphism specified by the universal property.

	Supposing \(\pfunc{h,k}{\QQ}{R}\) are parallel homomorphisms, they are equal
	by virtue of the fact that there is at most one homomorphism from \(\QQ\) to
	\(R\), and \(\io\) vacuously fulfils the condition of an epimorphism.
\end{proof}
\end{document}

