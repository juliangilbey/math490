\documentclass[main.tex]{subfiles}
\begin{document}

\paragraph{}
\setexercise{1}{2}{7}
\begin{exercise}
	Regarding a poset \((\sP,\le)\) as a category, define the supremum of a
	subcollection of objects \(A\in\sP\) in such a way that the dual
	statement defines the infimum. Prove that the supremum of a subset of
	objects is unique, whenever it exists, in such a way that the dual proof
	demonstrates the uniqueness of the infimum.
\end{exercise}
\begin{proof}
	Given a subcollection, \(C\), of objects \(A\in\sP\), define an upper bound
	as follows: a object \(u\) is an upper bound of \(C\) if for all objects
	\(x\) in \(C\) there is a morphism \(\func{x\le u}{x}{u}\). (Recall that
	morphisms in a poset category are merely elements of the \(\le\) relation.)
	Note that this immediately gives us a dual notion of a lower bound by
	considering instead \(\sP^\op\). A lower bound of \(C\) in \(\sP\) is an
	upper bound of \(C\) in \(\sP^\op\). In other words an object \(l^\op\) such
	that for all objects \(x^\op\) in \(C\) there is a morphism \(\func{(x\le
	l)^\op}{x^\op}{l^\op}\), or equivalently \(\func{(l\le x)}{l}{x}\).

	Letting \(F\) be the collection of all upper bounds of \(C\), we define the
	supremum of \(C\), if it exists%
	\footnote{There are many cases where suprema fail to exist. Consider
		the poset category: \[\xymatrix{a&b\\c\ar[u]\ar[ur]&d\ar[u]\ar[ul]}\]
		the set \(\{c,d\}\) has as upper bounds \(\{a,b\}\). However,
		\(\{a,b\}\) has as lower bounds \(\{c,d\}\). Because these sets are
		disjoint there is no supremum of \(\{c,d\}\). Even in more common
		orderings, like the usual ordering on the rational numbers,
		subcollections can fail to have suprema. For example,
		\(\{x\in\QQ|x^2<2\}\).

		A poset with the property that any collection of elements has a supremum
		and infimum is called a complete lattice.
	}, to be a lower bound of \(F\) (as defined
	above) which is contained in \(F\). The condition of containment implies
	uniqueness. Supposing we have two lower bounds \(x\) and \(y\) of \(F\). If
	both are contained in \(F\), then there are maps \(\func{x\le y}{x}{y}\) and
	\(\func{y\le x}{y}{x}\). Since the only endomorphisms in \(\sP\) are
	identities these must compose to identities and thus be inverses, and
	because \(\sP\) is a partially ordered set (as opposed to just being a
	preordered set) the only isomorphisms are identities. In familiar terms, a
	partial order is antisymmetric. Thus \(x\) and \(y\) are the same object.

	We may thus define the infimum of \(C\) to be its supremum on \(\sP^\op\).
	This time we consider the collection \(I\) of lower bounds \(C\) (the upper
	bounds of \(C\) in \(\sP^\op\)). The infimum is then an upper bound of \(I\)
	which is contained in \(I\) (a lower bound in \(\sP^\op\)). The infimum must
	be unique because it's a supremum in the opposite category, and suprema are
	unique.
\end{proof}
\end{document}

