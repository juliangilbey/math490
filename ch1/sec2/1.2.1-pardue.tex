\documentclass[main.tex]{subfiles}
\begin{document}

\setexercise{1}{2}{1}
\begin{exercise}
	Show that \(\sC/c\cong(c/(\sC^\op ))^\op\). Defining \(\sC/c\) to be
	\((c/(\sC^\op))^\op\), deduce Exercise 1.1.iii(ii) from Exercise 1.1.iii(i).
\end{exercise}

\begin{proof}
	This exercise asks us to prove that two categories are isomorphic, which is
	a notion that we have not yet encountered. But, I will prove that the two
	categories are equal!

	This exercise uses definitions from Exercise 1.1.iii. There are so many
	layers in the present exercise that to keep things straight, it will help to
	add one more piece of notation. Recall that for an object \(c\) of a
	category \(\sC\), the slice category \(\sC/c\) of \(\sC\) over \(c\) has as
	objects the morphisms \(\func{f}{x}{c}\) in \(\sC\). A morphism from \(f\)
	to \(g\) in \(\sC/c\), where \(f\) has domain \(x\) and \(g\) domain \(y\)
	in \(\sC\), is a morphism \(\func{h}{x}{y}\) in \(\sC\) such that \(gh=f\).
	To distinguish between \(h\) as viewed in \(\sC\) and in \(\sC/c\), let's
	write \(\func{h'}{f}{g}\) when we want to consider \(h\) as a morphism in
	\(\sC/c\) and \(\func{h}{x}{y}\) when we want to consider it in \(\sC\)%
	\footnote{The notation \(h'\) can still be ambiguous since there might also
		be other \(\func{i}{c}{x}\) and \(\func{j}{c}{y}\) such that \(jh=i\) so
		that we also have \(\func{h'}{i}{j}\). This leads to different morphisms
		labeled \(h'\), but they are distinguished by their domains and
	codomains}. We can use similar notation for the slice category \(c/\sC\) of
	\(\sC\) under \(c\).

	Since we will make systematic and careful use of opposite categories, recall
	that the objects and morphisms of \(\sC\) and of \(\sC^\op\) are
	\emph{precisely the same}. Only, the assignment of domains and codomains are
	swapped, allowing order of composition to be swapped. If \(f\) is a morphism
	in \(\sC\), then \(f^\op\) is precisely the same morphism, but the \(\op\)
	reminds us that we are considering it in \(\sC^\op\) rather than in \(\sC\)
	so that we have different assignments for domain and codomain.

	Now, I claim that \(\sC/c=(c/(\sC^\op))^\op\). We must first check that they
	have the same objects, though we notate \(f\) in the first category as
	\(f^\op\) in the second. Then, for every pair of objects \(f\) and \(g\) in
	\(\sC/c\) we must see that \[(\sC/c)(f,g)=(c/(\sC^\op))^\op(f^\op, g^\op).\]
	(Note that we use this notation even when \(\sC\) is not locally small, so
	that each side of the equality might be a proper class.) Finally, we must
	see that composition of morphisms is the same in each category.

	Since the objects of a category and its opposite category are the same, the
	objects in \((c/(\sC^\op))^\op\) are the objects in \(c/(\sC^\op)\), which
	are morphisms \(\func{f^\op}{c}{x}\) in \(\sC^\op\). But, these are the same
	as morphisms \(\func{f}{x}{c}\) in \(\sC\), which is to say objects of
	\(\sC/c\) as claimed.

	For the rest, let \(\func{f}{x}{c}\), \(\func{g}{y}{c}\) and
	\(\func{h}{z}{c}\) be three morphisms in \(\sC\). A morphism
	\[\func{i^{\op\prime\op}}{f^\op}{g^\op}\] in \((c/(\sC^\op))^\op\) is just a
	morphism \[\func{i^{\op\prime}}{g^\op}{f^\op}\] in \(c/(\sC^\op)\). This in
	turn is a morphism \(\func{i^\op}{y}{x}\) such that \(i^\op g^\op=f^\op\) in
	\(\sC^\op\), together with the ordered pair \((g^\op,f^\op)\) giving the
	domain and codomain of \(i^{\op\prime}\). Unravelling one more layer, this
	is in turn a morphism \(\func{i}{x}{y}\) such that \(gi=f\) in \(\sC\)
	together with the ordered pair \((f,g)\). This in turn corresponds to a
	morphism \(\func{i'}{f}{g}\) in \(\sC/c\). Each of these correspondences is
	actually an equality of classes. So, we have argued that
	\begin{align*}
		&(c/(\sC^\op))^\op(f^\op, g^\op)\\
		=\ &(c/(\sC^\op))(g^\op, f^\op)\\
		=\ &\{i^\op\in\sC^\op(y,x)|i^\op g^\op=f^\op\}\times\{(g^\op,f^\op)\}\\
		=\ &\{i\in\sC(x,y)|gi=f\}\times\{(f,g)\}\\
		=\ &(\sC/c)(f,g)\\
	\end{align*}
		as required.

	Notice also in this correspondence that when \(f=g\) that the identities in
	each class are the same. Altogether, \(1_{f^\op}^\op=1_f\).

	Now, we must see that the composition laws are the same. We have already
	established above that \({i^{\op\prime}}^\op=i'\). Similarly, if
	\(\func{j'}{g}{h}\), then we have
	\(\func{j^{\op\prime\op}i^{\op\prime\op}}{f^\op}{h^\op}\). But, examining
	the definitions of opposite categories and of slice categories, we have
	equality of each of the following morphisms interpreted in the categories
	shown
	\begin{alignat*}{2}
		j^{\op\prime\op}i^{\op\prime\op}\colon&f^\op\to h^\op&&
		\qtextq{ in }(c/(\sC^\op))^\op\\
		i^{\op\prime}j^{\op\prime}\colon&h^\op\to f^\op&&
		\qtextq{ in }c/(\sC^\op)\\
		(i^\op j^\op)'\colon&h^\op\to f^\op&&
		\qtextq{ in }c/(\sC^\op)\\
		i^\op j^\op\colon&z\to x&&\qtextq{ in }\sC^\op\\
		ji\colon&x\to z&&\qtextq{ in }\sC\\
		(ji)'\colon&f\to h&&\qtextq{ in }\sC/c\\
		j' i'\colon&f\to h&&\qtextq{ in }\sC/c
	\end{alignat*} This proves that the two
	categories share the same composition law. Thus, they are one and the same.

	Now, looking back at Exercise 1.1.iii, we see that we we could have defined
	\(\sC/c\) as \((c/(\sC^\op))^\op\), so that the existence of the category
	\(\sC/c\) follows from the existence of \(c/(\sC^\op)\), which we have by
	the first part of Exercise 1.1.iii applied to \(\sC^\op\).
\end{proof}

\end{document}
