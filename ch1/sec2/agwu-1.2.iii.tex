\documentclass[main.tex]{subfiles}
\begin{document}

\begin{lemma}\leavevmode
	\begin{enumerate}[(i)]
		\item If \(\mono{f}{x}{y}\) and \(\mono{g}{y}{z}\) are monomorphisms,
			then so is \(\mono{gf}{x}{z}\).

		\item If \(\func{f}{x}{y}\) and \(\func{g}{y}{z}\) are morphisms so
			that \(gf\) is monic, then \(f\) is monic.
	\end{enumerate}
	Dually:
	\begin{enumerate}[(i')]
		\item If \(\epi{f}{x}{y}\) and \(\epi{g}{y}{z}\) are epimorphisms,
			then so is \(\epi{gf}{x}{z}\).

		\item If \(\func{f}{x}{y}\) and \(\func{g}{y}{z}\) are morphisms so
			that \(gf\) is epic, then \(g\) is epic.
	\end{enumerate}
\end{lemma}

\begin{exercise}
	Prove Lemma 1.2.11 by proving either (i) or (i') and either (ii) or (ii'),
	then arguing by duality. Conclude that the monomorphisms in any category
	define a subcategory of that category and dually that the epimorphisms also
	define a subcategory.
\end{exercise}

First we will show the above properties for monomorphisms, and then apply
duality, as the problem suggests, to prove the corresponding properties for
epimorphisms.

\begin{proof}
	% {\bf Proof of Lemma 1.2.11 (i)}\\
	First, we will prove that the composition of two monomorphisms is a
	monomorphism. Let \(\sC\) be a category and \(\func{f}{x}{y}\) and
	\(\func{g}{y}{z}\) be monomorphisms of \(\sC\). Let \(\func{h,k}{w}{x}\) be
	two morphisms in \(\sC\) so that: \(gf)h=(gf)k\). Since compositions of
	morphisms are associative, we have \(g(fh)=g(fk)\). Since \(g\) is monic, we
	get: \((fh)=(fk)\). Since \(f\) is monic, we ultimately get: \(h=k\). Thus,
	\(gf\) is monic. Thus, the compositions of two monomorphisms is indeed a
	monomorphism.

	% {\bf Proof of Lemma 1.2.11 (ii)}\\
	Next we will show that If the composition of two morphisms is monic, then
	the rightmost morphism is monic. Take morphisms \(\func{a}{x}{y}\) and
	\(\func{b}{y}{z}\) from category \(\sC\) where \(ba\) is monic. Take
	\(\func{h,k}{w}{x}\) so that \(ah=ak\). Left composing \(b\) on both sides
	of the equations results in: \(b(ah)=b(ak)\). By associativity we get:
	\((ba)h=(ba)k\). Applying the properties of monomorphisms results in:
	\(h=k\). Thus \(a\) is monic. So we have shown that if the composition of two
	morphisms is monic, then the rightmost morphism is monic.

	% {\bf Monomorphisms form a subcategory}\\
	Now we will show that the monomorphisms of any category forms a category.
	Suppose that \(\sD\) is a subcategory of \(\sC\) where the morphisms of
	\(\sD\) are the monomorphisms of \(\sC\) and \(\sD\) and \(\sC\) have the
	same objects. Since for any object \(x\) in \(\sC\), if we had for morphisms
	\(h\) and \(k\) of \(\sC\) with codomain \(x\) the following property:
	\(id_{x}h=id_{x}k\), then \(h=k\), since \(id_{x}\) is left cancellable,
	thus the identity morphism for every object in \(\sD\) is a monomorphism.
	Therefore, every object in \(\sD\) has a identity arrow in \(\sD\). Since
	the composition of two monomorphisms is a monomorphism, then \(\sD\)
	contains compositions of its morphisms. Obviously, the domains and codomains
	of morphisms of \(\sD\) are contained in \(\sD\) since \(\sD\) and \(\sC\)
	have the same objects. Thus \(\sD\) is a subcategory of \(\sC\).

	We have shown that:
	\begin{enumerate}
		\item the composition of two monomorphisms in \(\sC\) is a monomorphism.

		\item If the composition of two morphisms in \(\sC\) is monic, then the
			rightmost morphism is monic.

		\item The class of monomorphisms of any category \(\sC\) forms a
			subcategory of \(\sC\).
	\end{enumerate}

	% {\bf Proofs of Lemma 1.2.11 (i'), Lemma 1.2.11 (i'), and Epimorphisms forming a subcategory}\\
	Now we will use duality to show the corresponding properties for
	epimorphisms. If we have the opposite category \(\sC^{\op}\), where the
	epimorphisms of \(\sC\) are the monomorphisms of \(\sC^{\op}\), this means
	that the three properties proven for monomorphisms also work for
	\(\sC^{\op}\). The properties of monomorphisms in \(\sC^{\op}\) are the dual
	properties of epimorphisms in \((\sC^{\op})^{\op} = \sC\). Namely,
	\begin{enumerate}
		\item the composition of two epimorphisms in \(\sC\) is an epimorphism.

		\item If the composition of two morphisms in \(\sC\) is epic, then the
			leftmost morphism is epic (Since \(f^\op g^\op\) in \(\sC^\op\)
			corresponds to \(gf\) in \(\sC\)).

		\item The class of epimorphisms of any category \(\sC\) forms a
			subcategory of \(\sC\).
	\end{enumerate}
	This completes the proof.
\end{proof}

\end{document}

